% IMPORTANT: Please write various parts in different files, and then include
% them into this document.
% If you have a file called intro.tex then write: \include{intro}
% This is to avoid nasty merge conflicts, as well as to keep it tidy,
% modular, etc

\documentclass[12pt,a4paper,twoside]{book}

\usepackage{etex} % for fixing error: no room for a new \dimension

% Make bibliography appear in table of contents
\usepackage[nottoc,numbib]{tocbibind}

\usepackage{amsmath,amssymb,calc,ifthen,capt-of,mathrsfs,eufrak}
\usepackage{multirow} % for mergins cells vertically
\usepackage{makecell} % for thicker horizontal/vertical lines

%\usepackage[giveninits=true,hyperref]{biblatex}
%\bibliography{citations} % or

\usepackage[raggedright]{titlesec} % for section titles not to exceed page margins

\usepackage[nomessages]{fp}% http://ctan.org/pkg/fp
\usepackage{python} % allows to run pythonrvisors code inside latex

% \usepackage{subfig}%

\usepackage[ampersand]{easylist}

\usepackage[table,usenames,dvipsnames]{xcolor} % for coloured cells in tables

% \usepackage{auto-pst-pdf} % convert ps to pdf ... you need this if you compile with pdflatex
\usepackage{epsf}

\definecolor{blue3}{HTML}{86B7FC} % med blue
\definecolor{blue1}{HTML}{B5F1FF} % light blue
\definecolor{blue2}{HTML}{E0F9FF} % very light blue
\definecolor{blue4}{HTML}{0047bc} % dark blue


\usepackage[bottom]{footmisc} % prevent figures/floats to be placed after the footnote

% Allows us to click on links and references!
% http://tex.stackexchange.com/questions/73862/how-can-i-make-a-clickable-table-of-contents
\usepackage{hyperref}
\hypersetup{
    colorlinks,
    citecolor=blue4,
    filecolor=black,
    linkcolor=blue4,
    urlcolor=blue4
}

% Nice package for plotting graphs
% See excellent guide:
% http://www.tug.org/TUGboat/tb31-1/tb97wright-pgfplots.pdf
\usepackage{pgfplots}
\usetikzlibrary{plotmarks}
\usepackage{amsmath,graphicx}
\usepackage{epstopdf}
\usepackage{caption}
\usepackage{subcaption}
\usepackage{float}
\usepackage[table]{xcolor}

\usepackage{mathtools}
\DeclarePairedDelimiter\ceil{\lceil}{\rceil}
\DeclarePairedDelimiter\floor{\lfloor}{\rfloor}
\usepackage{arydshln} % for horiz dotted line in table
\usepackage{pdfpages} % for importing pdf images

%\usepackage[romanian]{babel} % for romanian characters
%\usepackage{combelow}

\usepackage{multicol} % double-colum list

\usepackage{listings} % for listing source code

\usepackage[toc,page]{appendix}

\usepackage{scalefnt} %^ scale font for neurological brain images

\pgfplotsset{compat = newest}

% highlight - useful for TODOs and similar
\usepackage{color}
%  \newcommand{\hilight}[1]{\colorbox{yellow}{#1}}
\newcommand{\hilight}[1]{}

\newcommand{\HRule}{\rule{\linewidth}{0.5mm}}

\usepackage{array,booktabs}

\usepackage[export]{adjustbox}

\usepackage{pifont}% for xmark http://ctan.org/pkg/pifont

% margin size
\usepackage[margin=1in,headheight=14pt]{geometry}

% \usepackage{algpseudocode} % for writing pseudocode & algorithms
\usepackage[linesnumbered]{algorithm2e}
\newcommand\mycommfont[1]{\footnotesize\ttfamily\textcolor{blue}{#1}}
\SetCommentSty{mycommfont}

\usepackage{enumitem} % for nested enumerate numbers 1 1.1 1.1.1
\usepackage{pgfgantt} % for grantt charts
\usepackage{rotating}
\usepackage[graphicx]{realboxes}

\usepackage{pdflscape}

\newganttchartelement*{mymilestone}{
mymilestone/.style={
shape=isosceles triangle,
inner sep=0pt,
draw=cyan,
top color=white,
bottom color=cyan!50
},
mymilestone incomplete/.style={
/pgfgantt/mymilestone,
draw=yellow,
bottom color=yellow!50
},
mymilestone label font=\slshape,
mymilestone left shift=0pt,
mymilestone right shift=0pt
}

\newgantttimeslotformat{stardate}{%
\def\decomposestardate##1.##2\relax{%
\def\stardateyear{##1}\def\stardateday{##2}%
}%
\decomposestardate#1\relax%
\pgfcalendardatetojulian{\stardateyear-01-01}{#2}%
\advance#2 by-1\relax%
\advance#2 by\stardateday\relax%
}

\usepackage{tikz}
\usetikzlibrary{arrows,positioning, shapes.symbols,shapes.callouts,patterns,shapes,chains,calc,backgrounds,fadings}
\tikzstyle{state}=[circle,thick,draw=black, align=center, minimum size=2.1cm,
inner sep=0]
\tikzstyle{vertex}=[circle,thick,draw=black]
\tikzstyle{vertex2}=[circle,thick,draw=black,minimum size=0.3cm]
\tikzstyle{vertex3}=[circle,thick,draw=white,minimum size=0.3cm,fill=white]
\tikzstyle{terminal}=[rectangle,thick,draw=black]
\tikzstyle{edge} = [draw,thick]
\tikzstyle{lo} = [edge,dotted]
\tikzstyle{hi} = [edge]
\tikzstyle{trans} = [edge,->]
\tikzstyle{image}=[circle,thick,draw=black]


\usepackage[section]{placeins} % prevents placing floats before a section

\usepackage{color}

\usepackage{silence} % silence warning that page 

\definecolor{mygreen}{rgb}{0,0.6,0}
\definecolor{mygray}{rgb}{0.5,0.5,0.5}
\definecolor{mymauve}{rgb}{0.58,0,0.82}

\lstset{ %
  backgroundcolor=\color{white},   % choose the background color; you must add \usepackage{color} or \usepackage{xcolor}
  basicstyle=\ttfamily,        	    % the size of the fonts that are used for the code
  breakatwhitespace=false,         % sets if automatic breaks should  happen at whitespace
  breaklines=true,                 % sets automatic line breaking
  captionpos=b,                    % sets the caption-position to bottom
  commentstyle=\color{mymauve},    % comment style
  deletekeywords={...},            % if you want to delete keywords from the given language
  escapeinside={\%*}{*)},          % if you want to add LaTeX within your code
  extendedchars=true,              % lets you use non-ASCII 
  frame=single,                    % adds a frame around the code
  keepspaces=true,                 % keeps spaces in text, useful for keeping indentation of code (possibly needs columns=flexible)
  keywordstyle=\color{blue},       % keyword style
  language=C++,                    % the language of the code
  morekeywords={*,...},            % if you want to add more keywords to the set
  numbers=left,                    % where to put the line-numbers; possible values are (none, left, right)
  numbersep=5pt,                   % how far the line-numbers are from the code
  numberstyle=\tiny\color{mygray}, % the style that is used for the line-numbers
  rulecolor=\color{black},         % if not set, the frame-color may be changed on line-breaks within not-black text (e.g. comments (green here))
  showspaces=false,                % show spaces everywhere adding particular underscores; it overrides 'showstringspaces'
  showstringspaces=false,          % underline spaces within strings 
  showtabs=false,                  % show tabs within strings adding particular underscores
  stepnumber=2,                    % the step between two line-numbers. If it's 1, each line will be numbered
  stringstyle=\color{mymauve},     % string literal style
  tabsize=2,                       % sets default tabsize to 2 spaces
  title=\lstname                   % show the filename of files included with \lstinputlisting; also try caption instead of title
}

\setcounter{secnumdepth}{3}

% % Left bar
\usepackage{framed}
\usepackage{amsthm}
\usepackage{thmtools}


\definecolor{lightgray}{gray}{0.00}
\renewenvironment{leftbar}[1][\hsize]{%
  \def\FrameCommand{%
    {\hspace{0pt}\color{lightgray}\vrule width 3pt}%
    \hspace{5pt}%
    %\fboxsep=\FrameSep\colorbox{lightgray}%
  }%
  \MakeFramed{\hsize#1\advance\hsize-\width\FrameRestore}%
}
{
  \endMakeFramed%
}
\setlength{\FrameSep}{0pt}

% Custom environments
% \theoremstyle{definition}
% \declaretheorem[shaded={bgcolor={gray}{0.93},margin=5pt}]{definition}
% \declaretheorem[name=Definition]{rawdefinition}
% \theoremstyle{definition}
% \declaretheorem{proposition}
% \theoremstyle{definition}
% \declaretheorem{lemma}
% \declaretheorem{corollary}

% \newtheorem{rawdefinition}{Proof}
% 
% \newtheoremstyle{named}{1em}{1em}{\itshape}{}{\bfseries}{}{\newline}{#3 }
% \theoremstyle{named}
% \newtheorem*{namedtheorem}{Theorem}


\newenvironment{myalgo}[1][]{%
  \begin{leftbar}%
  \vspace{-11pt}%
  \subsubsection{#1}
}
{
  \end{leftbar}%

}

\newenvironment{algotwo}[1][]{%
  \begin{leftbar}%
  \vspace{-11pt}%
  \subsubsection{#1}
  \begin{algorithm}
}
{
  \end{algorithm}
  \end{leftbar}%

}


\usepackage{tabularx, booktabs} % make width of table columns evenly distributed (see http://tex.stackexchange.com/questions/60601/evenly-distributing-column-widths)
\newcolumntype{Y}{>{\centering\arraybackslash}X}


% % Definition with left bar
% \newenvironment{mydef}[1][]{%
%   \begin{rawdefinition}[#1]%
%   \vspace{-11pt}%
%   \begin{leftbar}%
% }
% {
%   \end{leftbar}%
%   \vspace{-11pt}%
%   \end{rawdefinition}%
% }

\DeclareMathOperator*{\argmin}{arg\,min}
\DeclareMathOperator*{\argmax}{arg\,max}

% fancy headers and footers from Petr 

\usepackage{fancyhdr}

\newcommand{\FrontPageStyle}{\pagestyle{empty}}
\newcommand{\MainPageStyle}{\pagestyle{main}}

\fancypagestyle{plain}{%
  \fancyhf{}%
  \renewcommand{\headrulewidth}{0pt}%
  \renewcommand{\footrulewidth}{0pt}%
}

\fancypagestyle{main}{%
  \fancyhf{}%
  \fancyhead[RE]{\nouppercase{\leftmark}}%
  \fancyhead[LO]{\nouppercase{\rightmark}}%
  \fancyhead[LE,RO]{\thepage}
}

% \includeonly{introduction, background_research, pcaProg, perf_eval, voxelwise, dkt, tadpole, conclusion, appendix}
% \includeonly{pcaProg, appendix}
%\newcommand*{\pcaPaperFigs}{../PCA_long_paper/figures}
\newcommand*{\pcaPaperFigs}{images/pcaPaperFigs}
\newcommand{\voxDpmFolder}{../voxelwiseDPM/}

\newcommand\ci{\perp\!\!\!\perp} % perpendicular sign
\newcommand{\Mu}{M}
\newcommand\independent{\protect\mathpalette{\protect\independenT}{\perp}}
\def\independenT#1#2{\mathrel{\rlap{$#1#2$}\mkern2mu{#1#2}}}

\newcommand{\beq}{\begin{equation}}
\newcommand{\eeq}{\end{equation}}




\DeclareSymbolFont{matha}{OML}{txmi}{m}{it}% txfonts
\DeclareMathSymbol{\varv}{\mathord}{matha}{118}

% \usepackage{titling}
% 
% % set up \maketitle to accept a new item
% \predate{\begin{center}\placetitlepicture\large}
% \postdate{\par\end{center}}
% 
% % commands for including the picture
% \newcommand{\titlepicture}[2][]{%
%   \renewcommand\placetitlepicture{%
%     \includegraphics[#1]{#2}\par\medskip
%   }%
% }
% \newcommand{\placetitlepicture}{} % initialization



\begin{document}
\belowdisplayskip=12pt plus 3pt minus 9pt
\belowdisplayshortskip=7pt plus 3pt minus 4pt

% for fancyhdr
\FrontPageStyle{}

\begin{titlepage}
\begin{center}

% % Upper part of the page. The '~' is needed because \\
% % works if a paragraph has started.
% \includegraphics[width=0.3\textwidth]{./images/ucl-logo2}~\\[1cm]
% 
% \textsc{\LARGE University College London}\\[1.5cm]
% 
% \textsc{\Large MRes Project}\\[0.5cm]

{\Large 

% Title
\HRule \\[0.4cm]
{ \LARGE Modelling the Neuroanatomical Progression of Alzheimer's Disease and Posterior Cortical Atrophy\\[0.4cm] }

\HRule \\[1.5cm]

% Author and supervisor
\begin{minipage}{0.4\textwidth}
\begin{flushleft} \Large
\emph{Author:}\\
R\u{a}zvan V. \textsc{Marinescu}
\end{flushleft}
\end{minipage}
\begin{minipage}{0.5\textwidth}
\begin{flushright} \Large
\emph{Supervisors:} \\
Prof. Daniel C. \textsc{Alexander}\\
Dr. Sebastian \textsc{Crutch}\\
Dr. Neil P. \textsc{Oxtoby} 
\end{flushright}
\end{minipage}


\vfill

\includegraphics[height=7cm]{images/cortical_1.png}

\vfill
 
A dissertation submitted in partial fulfillment \\
of the requirements for the degree of\\[0.3cm]
\textbf{Doctor of Philosophy}\\[0.3cm]
of\\[0.3cm]
\textbf{University College London}\\[1cm]

Centre for Medical Image Computing, University College London
 
 \vfill
% 
% Word count: 18,847 (as measured by texcount)
% 
% \vfill

% Bottom of the page
{\Large \today}

}
\end{center}
\end{titlepage}
% \maketitle{}


\clearpage


I, R\u{a}zvan Valentin Marinescu, confirm that the work presented in this thesis is my own. Where information has been derived from other sources, I confirm that this has been indicated in the thesis.

\clearpage


\chapter*{Abstract}

In order to find effective treatments for Alzheimer's disease (AD), a devastating neurodegenerative disease affecting millions of people worldwide, we need to identify subjects at risk of AD as early as possible. To this end, disease progression models have been recently developed, which not only to perform early diagnosis, but also estimate a unique disease signature that is used to predict the subjects' disease stages and future evolution. However, these models have not yet been applied to rare neurodegenerative diseases, are not suitable to understand the complex dynamics of biomarkers, work only on large multimodal datasets, and their predictive performance has not been objectively validated.

In this work I developed novel models of disease progression and applied them to estimate the progression of Alzheimer's disease and Posterior Cortical atrophy, a rare neurodegenerative syndrome causing visual deficits. My first contribution is a study on the progression of Posterior Cortical Atrophy, using models already developed: the Event-based Model (EBM) and the Differential Equation Model (DEM). My second contribution is the development of DIVE, a novel spatio-temporal model of disease progression that estimates fine-grained spatial patterns of pathology, potentially enabling us to understand complex disease mechanisms relating to pathology propagation along brain networks. My third contribution is the development of Disease Knowledge Transfer (DKT), a novel disease progression model that estimates the multimodal progression of rare neurodegenerative diseases from limited, unimodal datasets, by transferring information from larger, multimodal datasets of typical neurodegenerative diseases. My fourth contribution is the development of novel extensions for the EBM and the DEM, and the development of novel measures for performance evaluation of such models. My last contribution is the organization of the TADPOLE challenge, a competition which aims to identify algorithms and features that best predict the evolution of AD. 

\chapter*{Impact Statement}
\thispagestyle{empty}

The work presented in this thesis furthers our understanding of the temporal evolution of Posterior Cortical Atrophy and Alzheimer's disease. The disease progression models and evaluation techniques that we developed can help towards understanding underlying disease mechanisms, aid patient stratification and drug evaluation in clinical trials for Alzheimer's disease and Posterior Cortical Atrophy, and can be used in clinical practice for predicting the future evolution of subjects that are at risk of developing Alzheimer's disease. 

I published the work in this PhD thesis in two first-author papers (DIVE and TADPOLE chapters), and will soon submit another two papers (DKT and PCA chapters). I have also communicated my results in international conferences. I have also engaged with the broader scientific community by organising the TADPOLE Challenge, as well as a couple of hackathons at the PyConUK conference and the CMIC Summer School.

%Alzheimer's disease (AD), along with related dementias, is a devastating neurodegenerative disease that affects up to 44 million people worldwide.  All clinical trials have so far failed to find disease-modifying treatments -- it is believed that treatments are administered too late, and to wrong subjects that have different underlying pathology. We thus need to administer treatments as early as possible, and to the right subjects. 



\chapter*{Acknowledgements}
\thispagestyle{empty}

%When I first started my PhD in October 2014, I attended a welcome lecture by Ivana Drobnjak who said that a PhD is a marathon and not a race. For me, it has actually been more of a triathlon. I've sometimes almost "drowned" in projects that didn't seem to lead to anywhere, while other times I've been cycling through tasks at great speed. 

There are many great people who have helped my PhD project become reality. First of all, I'd like to thank my supervisor Daniel Alexander, for his great advice, ideas and research directions. He has always encouraged me to pursue interesting ideas and supported me in developing them. Secondly, I'd also like to thank Alexandra Young and Neil Oxtoby for teaching me disease progression modelling, especially in the early years of my PhD. I'd also like to thank Sebastian Crutch, Tim Shakespeare, Keir Yong, and other DRC collaborators, for their help and advice on Posterior Cortical Atrophy and other clinical aspects of my work. I'd like to thank Marco Lorenzi, for trying to explain mathematics to a wanna-be mathematician like myself. Marco and Neil are also great guitar players, which I had the opportunity to hear a few times. I'd also like to thank Sara Garbarino, for her great spirit, for taking the time to repeatedly listen to my presentations when rehearsing them, and for reminding me that I was probably the biggest nerd in CMIC. I'd further like to thank the POND group, for the help they offered me throughout my PhD, for the great coffees we had after our meetings, and for reminding me that I can't deal with non-working technology in hotels during our trips in the Netherlands. I'd like to thank Gary Zhang for coaching me on how to present my work without putting half of the audience to sleep, as well as others in MIG and CMIC, for teaching me about diffusion MRI, machine learning and other imaging techniques. I remember coming to those meetings in early days of my PhD and not understanding what was being discussed.

In terms of the social aspect, I had a wonderful time at UCL. I'll miss the trips organised by Pawel Markiewicz around Wales and Cornwall, where we had a lot of fun surfing, playing frisbee and BBQ-ing on the beach. I'll also miss the great camping trips with the CMIC folks in Peak District and Lake district, when I attempted driving -- successfully! -- for the first time in the UK! I'll also miss the great time I had with Thore Bucking, Emma Hill and Kin Quan during the MRes year. I'll miss the dinners and lunches such as the EuroPOND celebratory lunch, when we got so excited that we each ordered 4 glasses of champagne, which got me tipsy. When we came back to UCL after lunch I realised I was actually breaking the code instead of doing anything useful.

Finally, I'd like to thank my parents, Aurora and Dan Marinescu, for their love and support, without which I wouldn't have been able to start the PhD in the first place. My brother Robert Marinescu, for his funny jokes and good spirit. My grandmother Anghelut\u{a} Constantina, for her funny and charismatic character. And my friends and housemates, in particular Vibhav Mishra, Carlos Gavidia and Mikael Brudfors, for the wonderful time spent in the Ifor residence, as well as Georgiana Ghetie, Alexandru Barbu and Oana Lang, for their light-hearted spirit, conversations and for the fun we had in the last few months of my PhD. 

\clearpage

\cleardoublepage{}
\MainPageStyle{}

\setcounter{tocdepth}{2}
\tableofcontents

\setcounter{tocdepth}{1}
\listoffigures

\listoftables

\newcommand{\pxgs}{\begin{equation}
  p(X|S) = \prod_{j=1}^J \left[ \sum_{k=0}^N p(k) \left( \prod_{i=1}^k p\left(x_{s(i),j} | E_{s(i)} \right) \prod_{i=k+1}^N p\left(x_{s(i),j} | \neg E_{s(i)}\right) \right) \right]
  \end{equation}}

\newcolumntype{C}[1]{>{\centering\let\newline\\\arraybackslash\hspace{0pt}}m{#1}}

\setlength{\tabcolsep}{0.2em}
  
\include{introduction}

\include{background_research}

\chapter[Longitudinal Neuroanatomical Progression of PCA]{Longitudinal Neuroanatomical Progression of Posterior Cortical Atrophy}
\label{chapter:pca}

This chapter outlines the clinically applied part of my PhD, which focused on modelling the progression of Posterior Cortical Atrophy using already developed methods. The content of this chapter is based on the neuroimaging results from the joint publication below, where I've re-written most of the text for this thesis. I performed all the neuroimaging work: image pre-processing, statistical analysis with EBM and DEM, and the interpretation of the results. The data from table \ref{tab:pcaDemographics} was gathered by Nicholas Firth. Splitting of PCA patients into cognitively-defined subgroups was done by Silvia Primativo. Details in section \ref{sec:pcaParticipants} regarding patient recruitment, patient numbers, clinical diagnosis and pathological confirmation along with image acquisition details from section \ref{sec:pcaImageAcq} were taken from our joint publication.


\section{Publications}

\begin{itemize}
 \item N. C. Firth*, S. Primativo*, R. V. Marinescu*, T. J. Shakespeare, A. Suarez-Gonzalez, M. Lehmann, A. Carton, D. Ocal, I. Pavisic, R. W. Paterson, C. F. Slattery, A. J. M. Foulkes, B. H. Ridha, E. Gil-Néciga, N. P. Oxtoby, A. L. Young, M. Modat, M. J. Cardoso, S. Ourselin, N. S. Ryan, B. L. Miller, G. D. Rabinovici, E. K. Warrington, M. N. Rossor, N. C. Fox, J. D. Warren, D. C. Alexander, J. M. Schott, K. X. X. Yong\^{} and S. J. Crutch\^{}, Longitudinal neuroanatomical and cognitive progression of posterior cortical atrophy, Brain, 2019. (*) joint first authors (\^{}) joint senior authors
 
 In the above manuscript, I preprocessed all the imaging data, performed the modelling and statistical analysis of all the imaging data, and created the figures, tables and diagrams (including statistical tests in the supplementary). I also drafted the section of the results which was related to the imaging results. Other authors recruited patients, collected the data, performed the analysis of cognitive tests, and helped draft the initial version of the manuscript. 

 \item R. V. Marinescu, A. L. Young, Neil P. Oxtoby, N. C. Firth, M. Lorenzi, A. Eshaghi, V. Wottschel, M. J. Cardoso, M. Modat, K. X. X. Yong, S. Primativo, N. C. Fox, M. Lehmann, T. J. Shakespeare, S. J. Crutch, D. C. Alexander, A data-driven comparison of the progression of brain atrophy in Posterior Cortical Atrophy and Alzheimer's disease, AAIC poster, 2016.

 \item R. V. Marinescu, S. Primativo, A. L. Young, N. P. Oxtoby, N. C. Firth, A. Eshaghi, S. Garbarino, J. M. Cardoso, K. Yong, N. C. Fox, M. Lehmann, T. J. Shakespeare, S. J. Crutch, D. C. Alexander, Analysis of the heterogeneity of Posterior Cortical Atrophy: Data-driven Model Predicts Distinct Atrophy Patterns for three different Cognitive Subgroups, AAIC poster, 2017 
\end{itemize}


\section{Introduction}

% Posterior Cortical Atrophy
Posterior Cortical Atrophy (PCA), already described in section \ref{sec:bckPca}, is a progressive neurodegenerative syndrome causing predominantly visuospatial and visuoperceptual impairments \cite{crutch2012posterior}. In order to understand complex disease mechanisms underlying PCA, and design efficient clinical trials for finding treatments of PCA, we need to be able to accurately estimate the temporal progression of atrophy in PCA and contrast it with typical AD (tAD). Previous neuroimaging studies of PCA have shown more atrophy in the superior parietal, occipital and posterior temporal regions as compared to typical AD \cite{lehmann2011cortical, whitwell2007imaging}. However, these studies are all cross-sectional and cannot map the continuous longitudinal progression of the disease. One longitudinal study of PCA \cite{lehmann2012global} showed widespread gray matter loss in both PCA and tAD, but the numbers were small (17 PCA and 16 tAD) and the time interval was short (1 year). Larger longitudinal studies are therefore required to robustly estimate longitudinal progression patterns of PCA as compared to tAD. Moreover, a second aspect that needs to be clarified is the heterogeneity within PCA itself. Some studies have so far reported three dominant subgroups: primary visual (the striate cortex, caudal), parietal (dorsal) and occipito-temporal (ventral) \cite{ross1996progressive,galton2000atypical}. However, evidence for the existence of these groups is mainly limited to individual case reports \cite{ross1996progressive,galton2000atypical} and no previous study looked at the temporal progression of brain atrophy in such subgroups.

The aim of this study is to estimate the progression of MRI brain volumes in PCA as compared to tAD. We used the event-based model (EBM, section \ref{sec:bckEbm}) and the differential equation model (DEM, section \ref{sec:bckDem}) to estimate the progression of brain volumes in 361 individuals (117 PCA, 106 tAD and 138 controls) from three centres in the UK, Spain and US. We also use the event-based model to estimate the progression of atrophy in three cognitively-defined PCA subgroups. Compared to previous studies, our study is the first comprehensive study of atrophy progression in PCA. We also provide the first glimpse into the early progression of atrophy within PCA subgroups. 


\section{Methods}

\subsection{Participants}
\label{sec:pcaParticipants}

% Number of patients, clinical diagnosis, pathological confirmation.
117 patients with PCA were recruited from three specialist centres: 100 from the Dementia Research Centre (DRC) UK, 9 patients from the University Hospital Virgen del Rocio (HUVR) Memory disorders Unit, Spain and 8 patients from the University of California San Francisco (UCSF) Memory and Aging Center, USA. All PCA participants met two widely-accepted Tang-Wai et al. \cite{tang2004clinical} and Mendez, Ghajarania \& Perryman \cite{mendez2002posterior} criteria. Participants had no clinical features of other neurodegenerative disorders (e.g. visual hallucinations, pyramidal signs), hence fulfilling the criteria for PCA-pure \cite{crutch2017consensus}. 106 tAD patients and 138 healthy controls recruited from the DRC UK were also used for this study. tAD subjects all met criteria for \emph{probable} AD \cite{mckhann2011diagnosis}. Available pathological and molecular analyses for the patients (45/117 = 38\% for PCA, 49/106  = 46\% for tAD) all indicated AD pathology. 

% Neuroimaging and cognitive
Of all the study participants, 270 had undergone at least one T1 MRI scan and 216 at least one cognitive assessment. Available neuroimaging and neuropsychology data, stratified by the number of visits, are shown in table \ref{tab:pcaDemographics}. PCA, tAD and healthy controls were age-matched (65.44 $\pm$ 7.51 for PCA, 65.67 $\pm$ 7.57 for tAD and 63.13 $\pm$ 5.94 for controls). The gender proportion was as follows: 39\% male for PCA, 62\% male for tAD and 50\% male for controls. PCA and tAD subjects had a similar level of impairment as measured by MMSE scores at first assessment: 20.88 $\pm$ 5.17 for PCA, 19.58 $\pm$ 5.08 for tAD and 29.02 $\pm$ 0.98 for controls. 

\newcommand{\tabwidth}{1cm}

\begin{table}
\centering
\begin{tabular}{c | c c c | c c c}
 & \multicolumn{3}{c|}{\textbf{Imaging}} & \multicolumn{3}{c}{\textbf{Neuropsychology}}\\
\textbf{Visits} & \textbf{Number} & \textbf{Age} & \textbf{Visit Interval} & \textbf{Number} & \textbf{Age} & \textbf{Visit Interval} \\
\multicolumn{7}{c}{}\\
\multicolumn{7}{c}{\textbf{PCA (n=117)}}\\
\hline
All & 89 & 63.52 $\pm$ 6.91 & N/A & 109 & 64.49 $\pm$ 7.54 & N/A \\
≥2 & 46 & 62.11 $\pm$ 6.52 & 1.03 $\pm$ 0.47 & 70 & 63.64 $\pm$ 7.32 & 1.18 $\pm$ 0.48 \\
≥3 & 31 & 62.75 $\pm$ 6.5 & 0.99 $\pm$ 0.47 & 45 & 62.73 $\pm$ 7.26 & 1.15 $\pm$ 0.45 \\ 
≥4 & 15 & 61.46 $\pm$ 4.44 & 0.86 $\pm$ 0.31 & 20 & 63.19 $\pm$ 7.00 & 1.14 $\pm$ 0.40 \\ 
≥5 & 9 & 61.73 $\pm$ 4.06 & 0.81 $\pm$ 0.33 & 7 & 59.44 $\pm$ 4.84 & 1.06 $\pm$ 0.45 \\
≥6 & 2 & 62.35 $\pm$ 1.65 & 0.83 $\pm$ 0.24 & 2 & 57.22 $\pm$ 3.49 & 1.02 $\pm$ 0.35 \\
\multicolumn{7}{c}{}\\
\multicolumn{7}{c}{\textbf{tAD (n=106)}} \\ 
\hline
All & 66 & 66.39 $\pm$ 8.58 & N/A & 58 & 65.68 $\pm$ 7.57 & N/A \\
≥2 & 37 & 66.84 $\pm$ 8.83 & 0.83 $\pm$ 1.46 & 28 & 64.58 $\pm$ 7.08 & 1.35 $\pm$ 0.56 \\ 
≥3 & 21 & 71.0 $\pm$ 6.97 & 0.53 $\pm$ 0.39 & 5 & 66.08 $\pm$ 2.78 & 1.26 $\pm$ 0.43 \\ 
≥4 & 14 & 70.89 $\pm$ 6.33 & 0.47 $\pm$ 0.33 & 0 & N/A & N/A \\
≥5 & 4 & 72.08 $\pm$ 4.81 & 0.49 $\pm$ 0.33 & 0 & N/A & N/A \\ 
≥6 & 1 & 79.9 $\pm$ 0.0 & 0.58 $\pm$ 0.40 & 0 & N/A & N/A \\
\multicolumn{7}{c}{}\\
\multicolumn{7}{c}{\textbf{Controls (n=138)}} \\
\hline
All & 115 & 61.87 $\pm$ 10.43 & N/A & 49 & 63.12 $\pm$ 5.90 & N/A \\
≥2 & 50 & 61 $\pm$ 12.01 & 0.79 $\pm$ 0.66 & 18 & 60.00 $\pm$ 5.87 & 0.91 $\pm$ 0.27 \\
≥3 & 28 & 65.75 $\pm$ 5.96 & 0.66 $\pm$ 0.52 & 0 & N/A & N/A \\
≥4 & 17 & 66.82 $\pm$ 4.88 & 0.45 $\pm$ 0.28 & 0 & N/A & N/A \\
≥5 & 8 & 66.11 $\pm$ 4.83 & 0.44 $\pm$ 0.25 & 0 & N/A & N/A \\
≥6 & 0 & N/A & N/A & 0 & N/A & N/A \\

\end{tabular}
\caption[Demographic details for participants in the PCA study]{Demographic details for participants in the PCA study. Number of participants (n), mean and standard deviation age of participants at baseline visit and mean and standard deviation of visit interval is shown per number of visits. }
\label{tab:pcaDemographics}
\end{table}

\begin{table}
\centering
\begin{tabular}{C{5cm}|c|c|c|c|c|c}
%\toprule
 & \multicolumn{2}{C{3.5cm}|}{\textbf{Vision subgroup (n=30)}} & \multicolumn{2}{C{3.5cm}|}{\textbf{Space subgroup (n=30)}} & \multicolumn{2}{C{3.5cm}}{\textbf{Object subgroup (n=29)}} \\
%\cline{2-7}
& n & mean (std) & n & mean (std) & n & mean (std)\\
\hline
MMSE &  29 &     22.38 (5.19) &  28 &    19.54 (4.10) &  29 &     21.90 (5.62) \\
Age &  30 &     64.23 (7.72) &  30 &    63.26 (7.44) &  29 &     65.05 (8.48) \\
Gender (\%male)	&	30 & 16\% &	30 &	46\% &	29 & 41\% \\
PAL &   6 &     15.00 (5.66) &   5 &     5.60 (6.18) &   8 &     10.75 (7.14) \\
Digitspan forwards &  17 &      6.76 (2.86) &  27 &     6.15 (2.46) &  24 &      7.08 (2.36) \\
Digitspan backwards &  16 &      3.88 (1.69) &  28 &     2.89 (1.42) &  23 &      3.78 (2.08) \\
GDA addition &  17 &      1.59 (1.94) &  22 &     0.55 (1.41) &  19 &      1.95 (2.84) \\
GDA subtraction &  17 &      0.76 (1.16) &  22 &     0.23 (0.67) &  19 &      1.84 (2.87) \\
\hline
\textbf{Memory} & & & & & \\
SRMT faces &  18 &     18.78 (4.13) &  25 &    18.88 (4.30) &  20 &     18.10 (3.95) \\
SRMT words &  30 &     21.70 (2.88) &  30 &    19.43 (3.45) &  29 &     20.38 (2.73) \\
\hline
\textbf{Visual processing} & & & & & \\
Shape detection (VOSP) &  28 &     14.43 (3.29) &  29 &    17.17 (2.24) &  28 &     17.18 (2.42) \\
Shape discrimination &  28 &     12.96 (3.18) &  28 &    15.21 (3.45) &  28 &     16.14 (2.68) \\
Crowding &  22 &      6.68 (4.23) &  28 &     9.07 (2.34) &  28 &      8.71 (2.23) \\
\hline
\textbf{Space Perception} & & & & & \\
Number location &  29 &      2.10 (2.66) &  29 &     2.07 (2.69) &  28 &      4.68 (2.83) \\
Dot counting (n correct) &  29 &      4.03 (3.27) &  29 &     3.93 (3.15) &  29 &      6.93 (2.64) \\
A cancellation (time) &  28 &    73.90 (20.24) &  29 &   82.25 (13.95) &  28 &    62.68 (22.65) \\
A cancellation (n misses) &  29 &      5.76 (5.33) &  29 &     4.00 (4.36) &  29 &      3.62 (4.77) \\
\hline
\textbf{Object perception} & & & & & \\
Object decision &  30 &      9.77 (4.51) &  30 &    12.63 (3.95) &  29 &     10.03 (4.13) \\
Fragmented letters &  23 &      3.57 (4.99) &  29 &     6.34 (5.25) &  27 &      5.19 (5.72) \\
Unusual views &  20 &      3.05 (3.97) &  25 &     6.16 (5.03) &  22 &      3.36 (3.95) \\
Usual views &  20 &      9.65 (6.42) &  25 &    15.16 (4.89) &  22 &     12.59 (5.58) \\
\hline
Memory -- visual processing composite score discrepancy &  30 &   -28.90 (16.82) &  30 &   14.26 (20.33) &  29 &    11.93 (14.79) \\
Space -- object composite score discrepancy  &  30 &    -1.62 (26.91) &  30 &   23.06 (16.93) &  29 &   -17.70 (16.30) \\
%\bottomrule
\end{tabular}
\caption[Baseline population demographics for PCA subgroups]{Baseline population demographics and neuropsychological data for PCA subgroups. For every neuropsychological test, we report the number of participants with available data (n) and the mean and standard deviation of the available measures.}
\label{tab:pcaSubgrDemographics}
\end{table}



% PCA subgroups
For analysing the heterogeneity within PCA, we split the dataset into three groups based on performance on a suite of cognitive tests. For each subject we computed the following composite tests by summing up scores from individual tests:
\begin{itemize}
 \item Early visual processing: shape detection, shape discrimination and crowding
 \item Visuoperceptual processing: object decision, fragmented letters, usual and unusual views.
 \item Visuospatial processing: number location, dot counting and A cancellation (time -- cut off at 90s)
 \item Episodic memory: short recognition memory test (sRMT) for words and faces
\end{itemize}

The score for each of the four categories was computed by standardising each of the sub-scores on a 0-100 scale, corresponding to the minimum and maximum values obtained by the participants, and then taking the average within each category. The subjects were then classified into three groups. The worst 1/3 of subjects (n=30) on the early visual processing tests as compared to the memory tests (i.e. difference between early visual and memory tests) were assigned to the vision subgroup. The remaining 2/3 of participants were split into two groups based on the difference between visuoperceptual and visuospatial tasks: subjects with space $<$ object performance (n=30) were assigned to the space subgroup while remaining subjects (n=29) were assigned to the object subgroup. Of all the PCA subjects selected for the subgroup analysis, only 23 (vision), 21 (space) and 18 (object) had imaging data. Demographics and neuropsychological data of subjects belonging to the PCA subgroups is shown in table \ref{tab:pcaSubgrDemographics}.


\subsection{Image Acquisition and Preprocessing}
\label{sec:pcaImageAcq}


% Images
89 PCA, 66 tAD and 115 healthy controls had at least one T1-weighted MRI scan (see Table\ref{tab:pcaDemographics}). A total of five different scanners were used: two 3T Trio (DRC and UCSF), 1.5T Intera (HUVR) and two 1.5 Signa (DRC). The scans had full brain coverage using between 124 and 208 coronal or sagittal slices of 1.0 or 1.5mm in thickness.

% Segmentation and registration
Estimation of region-of-interest (ROI) volumes was performed using the Geodesic Information Flow (GIF) \cite{cardoso2015geodesic} based on the Neuromorphometrics atlas \cite{neuromorphometrics}. The atlas produced 144 different brain ROIs across the left and right hemispheres. Segmentation failed for 6 scans belonging to 5 subjects (3 controls, 1 PCA and 1 tAD), which were subsequently removed. A total of 52 brain ROIs were removed (18 were not part of the cerebral cortex, 6 had segmentation errors, 28 were grouped into larger ROIs). After merging the left and right sides of the remaining ROIs, this resulted in a total of 46 ROIs which were further averaged into 8 ROIs corresponding to whole brain, hippocampal, occipital, frontal, entorhinal, parietal and ventricle. GIF-derived ROI volumes were corrected for total intracranial volume (TIV), age, gender, scanner type and site using a general linear model and a one-hot encoding scheme.

\subsection{Statistical Methods}
\label{sec:pcaStaMet}

\subsubsection{The Event-based Model}
\label{sec:pcaStaMetEBM}
% EBM
For finding the order in which brain regions are affected, we ran the standard Event-Based Model \cite{fonteijn2012event} (section \ref{sec:bckEbm}) independently on estimated volumes from PCA and tAD, using the shared control population in both scenarios. For the EBM event distributions, we chose a Gaussian distribution for likelihood of normal observations, and a uniform distribution for the likelihood of abnormal observations. We further assumed that the control population is well-defined, so we fit the Gaussian distribution directly on the control data. For the uniform distribution, we set the minimum and maximum of the uniform range to be equal to the smaller and largest observed biomarker values. For finding the optimal sequence, we used 25 different starting points and performed greedy ascent for 10,000 biomarker sequences (see section \ref{sec:model_est}). We chose the sequence with the maximum likelihood as the final sequence.

For estimating uncertainty within the EBM sequence, we used MCMC to take 100,000 samples of the event sequence, starting from the maximum likelihood solution. The perturbation rule used is described in detail in section \ref{sec:model_est}. 

\subsubsection{The Differential Equation Model}
\label{sec:pcaStaMetDEM}

\newcommand{\figScale}{0.5}

\begin{figure}
 \centering
 
 \begin{subfigure}{0.47\textwidth}
    \includegraphics[width=\textwidth]{images/demDiagramPlots/fig1_linReg.png}
    \caption{}
    \label{fig:pcaDemDiagramA}
 \end{subfigure}
 \begin{subfigure}{0.47\textwidth}
     \includegraphics[width=\textwidth]{images/demDiagramPlots/fig2_GP.png}
    \caption{}
    \label{fig:pcaDemDiagramB}
 \end{subfigure}
 \vspace{1em}
 
  \begin{subfigure}{0.47\textwidth}
    \includegraphics[width=\textwidth]{images/demDiagramPlots/fig3_recon.png}
    \caption{}
    \label{fig:pcaDemDiagramC}
 \end{subfigure}
 \begin{subfigure}{0.47\textwidth}
     \includegraphics[width=\textwidth]{images/demDiagramPlots/fig4_align.png}
    \caption{}
    \label{fig:pcaDemDiagramD}
 \end{subfigure}
 \caption[Diagram of the Differential Equation Model]{Diagram of the Differential Equation Model. (a) Subject-specific biomarker rates of change were measured from line of best fit, i.e. line slope. (b) Rate of change model: the slopes of each fitted line were plotted against the average biomarker value of each subject (blue crosses). A non-parametric model (Gaussian Process regression, green line) was then fitted on measurements. (c) Trajectory reconstruction: A line integral was performed on the rate of change model. (d) Anchoring process: to give an absolute time reference, the origin $t_0$ was set as the line that best separates controls from patients, which have been staged along the time axis using their biomarker data. Diagram made by me.}
 \label{fig:pcaDemDiagram}
\end{figure}


% DEM
For estimating the rate and extent of biomarker decline, we applied the Differential Equation Model \cite{villemagne2013amyloid, oxtoby2018} (section \ref{sec:bckDem}). The methodology is outlined in Fig. \ref{fig:pcaDemDiagram}. The biomarker measurements for each subject were plotted against time since baseline, and a line was fit for each subject independently. The slope of these lines was then used as a measure of the biomarker rate of change (Fig. \ref{fig:pcaDemDiagramA}). The slopes of each fitted line were then plotted against the average biomarker value of each subject (Fig. \ref{fig:pcaDemDiagramB}). A line integral is then performed on the rate of change model (Fig. \ref{fig:pcaDemDiagramC}).  We repeat the DEM fitting for 8 ROIs independently: the four main lobes, whole brain, ventricles, hippocampus and entorhinal cortex. In order to align all images on a common time frame, we staged the controls and patients based on all their data, and then set an absolute time reference $t_0$ as the line that best separated (i.e. maximised the balanced classification accuracy) the controls' and patients' stages.\footnote{The staging of subjects using all their data required an initial trajectory alignment, which we aligned by initially setting $t_0$ to be the mean biomarker value of patients at baseline.}

There were some adaptations that we performed on the DEM model to ensure a good data fit. In the estimation of the rate of change model (Fig. \ref{fig:pcaDemDiagramB}), we did not include the controls, as there was very little change in their biomarker values. We also normalised the average biomarker values to z-scores and standardised the rates of change by dividing them with the average rate of change of all patients. At the line integration step (Fig. \ref{fig:pcaDemDiagramC}), the integration limits were defined as the biomarker values where the corresponding change is zero or the average biomarker value was equal to the minimum or maximum observed in the dataset. 

After the line integration step, we aligned all trajectories on a common time axis through an anchoring process, where we set the time $t_0 = 0$ to correspond to the value of that biomarker in patients at baseline, averaged across all patinets. More precisely, we set $f_j(t_0) = avg(X_j)$ where $f_j$ is the trajectory for biomarker $j$ and $X_j$ are the values of biomarker $j$ in tAD/PCA patients at baseline visit. After this initial anchoring, we staged the subjects along their progression and re-set the $t_0 = 0$ to correspond to the threshold that best separated controls from patients (Fig. \ref{fig:pcaDemDiagramD}). After the anchoring process, we converted all biomarker values to z-scores for comparability (Fig. \ref{fig:pcaDemDiagramD}).

To estimate uncertainty in the trajectories, we sampled 20 trajectories from the posterior distribution of the GP, and then anchored them like the mean trajectory. However, the anchoring would've resulted in zero noise interval at the anchor point, so to get realistic confidence intervals we added to each trajectory an extra amount of random noise $N(0, \sigma)$ on the y-axis, with $\sigma$ set to the standard deviation of the biomarker measurements of each subject at baseline visit.

\subsubsection{Statistical Tests}
% Statistical testing

In order to find out statistically significant differences between the EBM- and DEM-estimated trajectories, we applied several non-parametric statistical tests. 

For EBM results, we tested the effect size of  biomarker $i$ becoming abnormal before another biomarker $j$ both within- and between-group. Within-group differences were assessed using Wilcoxon signed-rank one-tailed tests for all pairs of biomarkers. Between-group (PCA vs tAD) and between-subgroup (space vs object, space vs vision and object vs vision subgroups) differences were assessed using two-tailed Mann-Whitney U tests. We used these non-parametric tests due to non-gaussianity of the data (data is ordinal representing ranks). The reason for using different tests (Wilcoxon vs Mann-Whitney) is becuase in one case we compare paired samples (two events within the same sequence sample), and in the other unpaired (two events in different sequences, e.g. in a randomly sampled PCA sequence vs a different randomly sampled tAD sequence). We also thinned the MCMC samples (1/100) due to dependence between consecutive samples. 

For DEM results, we tested for differences in estimated biomarker values at different timepoints (-10, 0 and 10 years from $t_0$) both within- and between-groups. For every pair of ROIs, within-group differences were assessed using two-tailed unpaired t-tests. For all ROIs and timepoints, between-group (PCA vs tAD) differences were assessed using similar two-tailed t-tests. For rejecting null hypotheses, we applied Bonferroni-corrected significance thresholds for all tests performed on EBM and DEM results.

\section{Results}
 
% scale parameter for the circles and the gradient
%\tikzset{every picture/.append style={scale=0.5}}
% scale parameter for the upper and lower small brain images
\newcommand*{\scaleBrainImg}{0.25}

% \newcommand*{\midLateralLoc}{../../../drawImages/input/images}

\newcommand*{\snapLocationPCA}{\pcaPaperFigs/ebmSnapshotsPCA} % or ../code/figures/mriAllGaussUnifDirPCA/snapshots
\newcommand*{\snapLocationAD}{\pcaPaperFigs/ebmSnapshotsAD} 
\newcommand*{\snapLocationEAR}{\pcaPaperFigs/ebmSnapshotsEAR} 
\newcommand*{\snapLocationPER}{\pcaPaperFigs/ebmSnapshotsPER} 
\newcommand*{\snapLocationSPA}{\pcaPaperFigs/ebmSnapshotsSPA} 
%col{x}{y}{z} respresents the color for ball z from matrix x at stage y (matrix x, stage y, ball z)

\newcommand*{\snapScale}{0.6} 
\definecolor{light-gray}{gray}{0.6}

\subsection{Progression of PCA and Typical AD}
\label{sec:pcaResPcaAd}

Fig. \ref{fig:pcaSnapshots} shows the maximum likelihood progression of atrophy estimated by the EBM, for both PCA and tAD patients. Snapshots of brain atrophy were taken at model stages 4, 8, 16, 24, 32, 40 and 46 (of 46) using the template from Supplementary Fig. \ref{fig:ebmSnapLabels}. Figure \ref{fig:pcaEBMProg} shows the maximum likelihood sequence and the variance in the main sequence. PCA patients show early atrophy in  occipital areas, ventricles and the superior parietal region, while tAD patients show early atrophy in the amygdala, hippocampus and entorhinal cortex, followed by temporal areas. The ordering is largely preserved under bootstrapping (Supplementary Fig. \ref{fig:bootPosVarAllPcaAd}), and supported by statistical testing (Supplementary Fig. \ref{fig:statTestPcaAd}). Differences in abnormality sequences between PCA and tAD are also statistically significant under Bonferroni corrections (Supplementary Figure \ref{fig:ebmStatTestPcaAd}).
 
Fig. \ref{fig:pcaAdDEM} shows the DEM-estimated biomarker trajectories for PCA (left) and tAD (right). Confidence estimates of the mean trajectory are also given in Fig. \ref{fig:trajDEMPcaAdConf}. Amongst PCA patients, occipital and parietal atrophy was most evident before $t_0$, and by $t_0$ we also observe considerable atrophy in the temporal lobe. Between $t_0$ and 10 years after $t_0$, we observe a marked increase in the rate of occipital, parietal and temporal atrophy and ventricular expansion. By contrast, hippocampal, entorhinal and frontal atrophy never match the extent of tissue loss in posterior and temporal regions. After 10 years from $t_0$, atrophy rates in occipital, parietal and temporal lobes seem to slow down, but limited data in this time window prevents drawing any clear conclusions. Statistical testing within the PCA cohort also confirms our conclusions -- see Supplementary Tables \ref{tab:demStatTestVolsPcaMinus10}, \ref{tab:demStatTestVolsPca0} and \ref{tab:demStatTestVolsPcaPlus10}.

By contrast, before $t_0$ tAD patients showed most extensive tissue loss in the hippocampus, confirmed by significance tests between hippocampal volume and other regions (p $<$ 4e-05, see Supplementary Figs. \ref{tab:demStatTestVolsAdMinus10} and \ref{tab:demStatTestVolsAd0}). After $t_0$, subsequent rates of change are the highest for temporal atrophy and ventricular expansion. It is of note that within 12 years from $t_0$, model estimates of parietal and ventricular abnormality amongst tAD patients are equivalent to or exceed the relative extent of hippocampal abnormality. Comparing PCA and tAD trajectories directly (Fig. \ref{fig:trajDEMPcaAdConf}), the separation between groups at $t_0$ is greatest in parietal (PCA $>$ tAD, p $<$ 1e-6) and hippocampal (tAD $>$ PCA, p $<$ 1e-22) volumes -- see Supplementary table \ref{tab:demStatTestPcaAd} for full statistical testing. 

\subsection{Progression of PCA Subgroups}
\label{sec:pcaResPcaSub}

Fig. \ref{fig:pcaSubgrSnap} shows snapshots of the EBM-estimated atrophy sequence at early stages, in the three cognitively-defined PCA subgroups. The bottom row in the figure shows the uncertainty in the estimated atrophy sequence. The vision subgroup has initial atrophy in the inferior occipital lobe, followed by the angular gyrus, middle temporal, precuneus and superior parietal. The space subgroup shows early atrophy in the superior parietal area (dorsal pattern), followed by inferior occipital and inferior and middle temporal areas. Finally, the object subgroup shows initial atrophy in the middle and inferior occipital areas, with subsequent atrophy in the inferior and middle temporal areas (ventral pattern). Bonferroni-corrected statistically significant differences in atrophy progression have also been observed between PCA subgroups -- see Supplementary Fig. \ref{fig:ebmStatTestPcaSubgroups}. Longitudinal trajectories within PCA subgroups using the DEM could not be estimated due to lack of sufficient data. 


% EBM snapshots
\begin{figure}
  \begin{subfigure}{\textwidth}
   \centering
  % the red-to-yellow gradient on the right
  \begin{tikzpicture}[scale=\snapScale,auto,swap]
    \shade[left color=gray!30,right color=red] (0,0) rectangle (6,0.5);
    \node[inner sep=0] (corr_text) at (6,1) {abnormal};
    \node[inner sep=0] (corr_text) at (0,1) {normal};
    \node[inner sep=0] (corr_text) at (0.2,0.5) {};
  \end{tikzpicture}
  \end{subfigure}
  \vspace{1em}
  
    \textbf{\Large{Posterior Cortical Atrophy}}
    
  %\begin{subfigure}[b]{0.15\textwidth}
    \begin{tikzpicture}[scale=\snapScale,auto,swap]

    % the two brain figures on top
    \node (upper_brain) at (0,1.5) { \includegraphics*[scale=\scaleBrainImg,trim=0 0 240 0]{\snapLocationPCA/stage_4.eps}};
    \node (lower_brain) at (0,-1.5) { \includegraphics*[scale=\scaleBrainImg,trim=240 0 0 0]{\snapLocationPCA/stage_4.eps}};
    \node[above=0cm of upper_brain] (stage) {Stage 4};
    % the balls
    
    \end{tikzpicture}
  %\end{subfigure}
  % next subfigure
  \hspace{-1.5em}
  ~
    %\begin{subfigure}[b]{0.15\textwidth}
    \begin{tikzpicture}[scale=\snapScale,auto,swap]

    % the two brain figures on top
    \node (upper_brain) at (0,1.5) { \includegraphics*[scale=\scaleBrainImg,trim=0 0 240 0]{\snapLocationPCA/stage_8.eps}};
    \node (lower_brain) at (0,-1.5) { \includegraphics*[scale=\scaleBrainImg,trim=240 0 0 0]{\snapLocationPCA/stage_8.eps}};
    \node[above=0cm of upper_brain] (stage) {Stage 8};
    % the balls
    
    \end{tikzpicture}
  %\end{subfigure}
  % next subfigure
  \hspace{-1.5em}
  ~
  %\begin{subfigure}[b]{0.15\textwidth}
    \begin{tikzpicture}[scale=\snapScale,auto,swap]

    % the two brain figures on top
    \node (upper_brain) at (0,1.5) { \includegraphics*[scale=\scaleBrainImg,trim=0 0 240 0]{\snapLocationPCA/stage_16.eps}};
    \node (lower_brain) at (0,-1.5) { \includegraphics*[scale=\scaleBrainImg,trim=240 0 0 0]{\snapLocationPCA/stage_16.eps}};
    \node[above=0cm of upper_brain] (stage) {Stage 16};
    % the balls
    
    \end{tikzpicture}
  %\end{subfigure}
  % next subfigure
  \hspace{-1.5em}
  ~
  %\begin{subfigure}[b]{0.15\textwidth}
    \begin{tikzpicture}[scale=\snapScale,auto,swap]

    % the two brain figures on top
    \node (upper_brain) at (0,1.5) { \includegraphics*[scale=\scaleBrainImg,trim=0 0 240 0]{\snapLocationPCA/stage_24.eps}};
    \node (lower_brain) at (0,-1.5) { \includegraphics*[scale=\scaleBrainImg,trim=240 0 0 0]{\snapLocationPCA/stage_24.eps}};
    \node[above=0cm of upper_brain] (stage) {Stage 24};
    % the balls
    
    \end{tikzpicture}
  %\end{subfigure}
  % next subfigure
  \hspace{-1.5em}
  ~
  %\begin{subfigure}[b]{0.15\textwidth}
    \begin{tikzpicture}[scale=\snapScale,auto,swap]

    % the two brain figures on top
    \node (upper_brain) at (0,1.5) { \includegraphics*[scale=\scaleBrainImg,trim=0 0 240 0]{\snapLocationPCA/stage_32.eps}};
    \node (lower_brain) at (0,-1.5) { \includegraphics*[scale=\scaleBrainImg,trim=240 0 0 0]{\snapLocationPCA/stage_32.eps}};
    \node[above=0cm of upper_brain] (stage) {Stage 32};
    % the balls
    
    \end{tikzpicture}
  %\end{subfigure}
  % next subfigure
  \hspace{-1.5em}
  ~
  %\begin{subfigure}[b]{0.15\textwidth}
    \begin{tikzpicture}[scale=\snapScale,auto,swap]

    % the two brain figures on top
    \node (upper_brain) at (0,1.5) { \includegraphics*[scale=\scaleBrainImg,trim=0 0 240 0]{\snapLocationPCA/stage_40.eps}};
    \node (lower_brain) at (0,-1.5) { \includegraphics*[scale=\scaleBrainImg,trim=240 0 0 0]{\snapLocationPCA/stage_40.eps}};
    \node[above=0cm of upper_brain] (stage) {Stage 40};
    % the balls
    
    \end{tikzpicture}
  %\end{subfigure}
  % next subfigure
  \hspace{-1.5em}
  ~
  %\begin{subfigure}[b]{0.15\textwidth}
    \begin{tikzpicture}[scale=\snapScale,auto,swap]

    % the two brain figures on top
    \node (upper_brain) at (0,1.5) { \includegraphics*[scale=\scaleBrainImg,trim=0 0 240 0]{\snapLocationPCA/stage_46.eps}};
    \node (lower_brain) at (0,-1.5) { \includegraphics*[scale=\scaleBrainImg,trim=240 0 0 0]{\snapLocationPCA/stage_46.eps}};
    \node[above=0cm of upper_brain] (stage) {Stage 46};
    % the balls
    
    \end{tikzpicture}
  %\end{subfigure}
  % next subfigure
  \hspace{-1.5em}
  
  \textbf{\Large{Typical Alzheimer's Disease}}

%   \centering
  %\begin{subfigure}[b]{0.15\textwidth}
    \begin{tikzpicture}[scale=\snapScale,auto,swap]

    % the two brain figures on top
    \node (upper_brain) at (0,1.5) { \includegraphics*[scale=\scaleBrainImg,trim=0 0 240 0]{\snapLocationAD/stage_4.eps}};
    \node (lower_brain) at (0,-1.5) { \includegraphics*[scale=\scaleBrainImg,trim=240 0 0 0]{\snapLocationAD/stage_4.eps}};
    \node[above=0cm of upper_brain] (stage) {Stage 4};
    % the balls
    
    \end{tikzpicture}
  %\end{subfigure}
  % next subfigure
  \hspace{-1.5em}
  ~
    %\begin{subfigure}[b]{0.15\textwidth}
    \begin{tikzpicture}[scale=\snapScale,auto,swap]

    % the two brain figures on top
    \node (upper_brain) at (0,1.5) { \includegraphics*[scale=\scaleBrainImg,trim=0 0 240 0]{\snapLocationAD/stage_8.eps}};
    \node (lower_brain) at (0,-1.5) { \includegraphics*[scale=\scaleBrainImg,trim=240 0 0 0]{\snapLocationAD/stage_8.eps}};
    \node[above=0cm of upper_brain] (stage) {Stage 8};
    % the balls
    
    \end{tikzpicture}
  %\end{subfigure}
  % next subfigure
  \hspace{-1.5em}
  ~
  %\begin{subfigure}[b]{0.15\textwidth}
    \begin{tikzpicture}[scale=\snapScale,auto,swap]

    % the two brain figures on top
    \node (upper_brain) at (0,1.5) { \includegraphics*[scale=\scaleBrainImg,trim=0 0 240 0]{\snapLocationAD/stage_16.eps}};
    \node (lower_brain) at (0,-1.5) { \includegraphics*[scale=\scaleBrainImg,trim=240 0 0 0]{\snapLocationAD/stage_16.eps}};
    \node[above=0cm of upper_brain] (stage) {Stage 16};
    % the balls
    
    \end{tikzpicture}
  %\end{subfigure}
  % next subfigure
  \hspace{-1.5em}
  ~
  %\begin{subfigure}[b]{0.15\textwidth}
    \begin{tikzpicture}[scale=\snapScale,auto,swap]

    % the two brain figures on top
    \node (upper_brain) at (0,1.5) { \includegraphics*[scale=\scaleBrainImg,trim=0 0 240 0]{\snapLocationAD/stage_24.eps}};
    \node (lower_brain) at (0,-1.5) { \includegraphics*[scale=\scaleBrainImg,trim=240 0 0 0]{\snapLocationAD/stage_24.eps}};
    \node[above=0cm of upper_brain] (stage) {Stage 24};
    % the balls
    
    \end{tikzpicture}
  %\end{subfigure}
  % next subfigure
  \hspace{-1.5em}
  ~
  %\begin{subfigure}[b]{0.15\textwidth}
    \begin{tikzpicture}[scale=\snapScale,auto,swap]

    % the two brain figures on top
    \node (upper_brain) at (0,1.5) { \includegraphics*[scale=\scaleBrainImg,trim=0 0 240 0]{\snapLocationAD/stage_32.eps}};
    \node (lower_brain) at (0,-1.5) { \includegraphics*[scale=\scaleBrainImg,trim=240 0 0 0]{\snapLocationAD/stage_32.eps}};
    \node[above=0cm of upper_brain] (stage) {Stage 32};
    % the balls
    
    \end{tikzpicture}
  %\end{subfigure}
  % next subfigure
  \hspace{-1.5em}
  ~
  %\begin{subfigure}[b]{0.15\textwidth}
    \begin{tikzpicture}[scale=\snapScale,auto,swap]

    % the two brain figures on top
    \node (upper_brain) at (0,1.5) { \includegraphics*[scale=\scaleBrainImg,trim=0 0 240 0]{\snapLocationAD/stage_40.eps}};
    \node (lower_brain) at (0,-1.5) { \includegraphics*[scale=\scaleBrainImg,trim=240 0 0 0]{\snapLocationAD/stage_40.eps}};
    \node[above=0cm of upper_brain] (stage) {Stage 40};
    % the balls
    
    \end{tikzpicture}
  %\end{subfigure}
  % next subfigure
  \hspace{-1.5em}
  ~
  %\begin{subfigure}[b]{0.15\textwidth}
    \begin{tikzpicture}[scale=\snapScale,auto,swap]

    % the two brain figures on top
    \node (upper_brain) at (0,1.5) { \includegraphics*[scale=\scaleBrainImg,trim=0 0 240 0]{\snapLocationAD/stage_46.eps}};
    \node (lower_brain) at (0,-1.5) { \includegraphics*[scale=\scaleBrainImg,trim=240 0 0 0]{\snapLocationAD/stage_46.eps}};
    \node[above=0cm of upper_brain] (stage) {Stage 46};
    % the balls
    
    \end{tikzpicture}
  %\end{subfigure}
  % next subfigure
  \hspace{-1.5em}
  
\caption[Atrophy progression in PCA and tAD patients according to the event-based model]{Atrophy progression in PCA and tAD patients according to the event-based model. White regions are within the volume range of healthy controls, while red regions show abnormally low volumes by the corresponding stage, with shading indicating the probability of abnormality. By each stage, a number of biomarkers shaded in red became abnormal. Brain pictures generated using BrainPainter \cite{marinescu2019BrainPainter}}  
\label{fig:pcaSnapshots}
\end{figure}

% EBM positional variance matrices
\begin{figure}
\centering
  \begin{subfigure}{0.7\textwidth}
  \centering
 \textbf{\large{\mbox{Posterior Cortical Atrophy}}}
 \includegraphics[width=1\textwidth,trim=100 30 0 50,clip]{\pcaPaperFigs/posVarianceMatrixPCA.png} 
 \end{subfigure}
 \hspace{1em}
  \begin{subfigure}{0.7\textwidth}
  \centering
  \textbf{\large{\mbox{Typical Alzheimer's Disease}}}
 \includegraphics[width=1\textwidth,trim=100 30 0 50,clip]{\pcaPaperFigs/posVarianceMatrixAD.png}
%  \caption{}
 \end{subfigure}
 \caption[PCA and tAD positional variance diagrams estimated by the EBM]{Uncertainty in the EBM-estimated atrophy sequences for (top) PCA and (bottom) tAD from Fig \ref{fig:pcaSnapshots}. The ROIs on the Y-axis are ordered according to the timing of abnormality, from early abnormalities on the top to late abnormalities on the bottom. The X-axis shows the position of a biomarker in the abnormality sequence. Each pixel at position $(i,j)$ shows the probability of biomarker $j$ becoming abnormal at position $i$, with darker squares showing higher confidence and whiter squares showing lower confidence. The biomarker orderings are sampled from the EBM posterior distribution.}
 \label{fig:pcaEBMProg}
\end{figure}

% DEM average traj
\begin{figure}
 \centering
\begin{subfigure}{0.8\textwidth}
 \centering
 \includegraphics[width=0.8\textwidth,trim=0 300 0 0,clip]{\pcaPaperFigs/trajAlign_600_500PCA.pdf}
\end{subfigure}
 
 \begin{subfigure}{0.47\textwidth}
\includegraphics[width=\textwidth,trim=90 0 120 60,clip]{\pcaPaperFigs/trajAlign_600_500PCA.pdf}
 \caption{PCA}
 \label{trajDEMPCA} 
 \end{subfigure}
 \begin{subfigure}{0.47\textwidth}
 \includegraphics[width=\textwidth,trim=90 0 120 60,clip]{\pcaPaperFigs/trajAlign_600_500AD.pdf}
 \caption{tAD}
 \label{trajDEMAD}
 \end{subfigure}
 \caption[PCA and tAD trajectories estimated by the DEM]{(a-b) Trajectories of different ROI volumes from the differential equation model for (a) PCA progression and (b) tAD progression. The x-axis shows the number of years since $t_0$, and the y-axis shows the z-score of the ROI volume relative to controls. The trajectories of the ventricles have been flipped to aid comparison. Overlayed are histograms of subject stages based on the estimated trajectories.}
 \label{fig:pcaAdDEM}
\end{figure}

% DEM confidence estimates
\begin{figure}
 \centering
 \includegraphics[width=\textwidth, trim=50 0 50 0]{\pcaPaperFigs/subplotsPcaAd.pdf}
 \caption[PCA and tAD trajectories aligned in the same space, with samples from the posterior distribution]{Mean trajectories for ROI volumes for PCA and tAD aligned on the same temporal scale with samples from the posterior distribution showing the confidence of the mean trajectory. The axis shows the number of years since $t_0$, and the y-axis shows the z-score of the ROI volume relative to controls. The trajectories for the ventricles have been flipped to aid visual comparison. The 1 std and 0 std horizontal lines represent the limit of 1 and 0 standard deviations away from the mean values of controls.}
 \label{fig:trajDEMPcaAdConf}
\end{figure}

% subgroup EBM results
\begin{figure}
  \begin{subfigure}{\textwidth}
   \centering
  % the red-to-yellow gradient on the right
  \begin{tikzpicture}[scale=\snapScale,auto,swap]
    \shade[left color=gray!30,right color=red] (0,0) rectangle (6,0.5);
    \node[inner sep=0] (corr_text) at (6,1) {abnormal};
    \node[inner sep=0] (corr_text) at (0,1) {normal};
    \node[inner sep=0] (corr_text) at (0.2,0.5) {};
  \end{tikzpicture}
  \end{subfigure}
%   \vspace{1em}

%   \centering
  {\Large \textbf{Vision impairment group}\par}
  %\begin{subfigure}[b]{0.15\textwidth}
    \begin{tikzpicture}[scale=\snapScale,auto,swap]

    % the two brain figures on top
    \node (upper_brain) at (0,1.5) { \includegraphics*[scale=\scaleBrainImg,trim=0 0 240 0]{\snapLocationEAR/stage_1.eps}};
    \node (lower_brain) at (0,-1.5) { \includegraphics*[scale=\scaleBrainImg,trim=240 0 0 0]{\snapLocationEAR/stage_1.eps}};
    \node[above=0cm of upper_brain] (stage) {Stage 1};
    % the balls
    
    \end{tikzpicture}
  %\end{subfigure}
  % next subfigure
  \hspace{-1.5em}
  ~
  %\begin{subfigure}[b]{0.15\textwidth}
    \begin{tikzpicture}[scale=\snapScale,auto,swap]

    % the two brain figures on top
    \node (upper_brain) at (0,1.5) { \includegraphics*[scale=\scaleBrainImg,trim=0 0 240 0]{\snapLocationEAR/stage_2.eps}};
    \node (lower_brain) at (0,-1.5) { \includegraphics*[scale=\scaleBrainImg,trim=240 0 0 0]{\snapLocationEAR/stage_2.eps}};
    \node[above=0cm of upper_brain] (stage) {Stage 2};
    % the balls
    
    \end{tikzpicture}
  %\end{subfigure}
  % next subfigure
  \hspace{-1.5em}
  ~
  %\begin{subfigure}[b]{0.15\textwidth}
    \begin{tikzpicture}[scale=\snapScale,auto,swap]

    % the two brain figures on top
    \node (upper_brain) at (0,1.5) { \includegraphics*[scale=\scaleBrainImg,trim=0 0 240 0]{\snapLocationEAR/stage_3.eps}};
    \node (lower_brain) at (0,-1.5) { \includegraphics*[scale=\scaleBrainImg,trim=240 0 0 0]{\snapLocationEAR/stage_3.eps}};
    \node[above=0cm of upper_brain] (stage) {Stage 3};
    % the balls
    
    \end{tikzpicture}
  %\end{subfigure}
  % next subfigure
  \hspace{-1.5em}
  ~
  %\begin{subfigure}[b]{0.15\textwidth}
    \begin{tikzpicture}[scale=\snapScale,auto,swap]

    % the two brain figures on top
    \node (upper_brain) at (0,1.5) { \includegraphics*[scale=\scaleBrainImg,trim=0 0 240 0]{\snapLocationEAR/stage_4.eps}};
    \node (lower_brain) at (0,-1.5) { \includegraphics*[scale=\scaleBrainImg,trim=240 0 0 0]{\snapLocationEAR/stage_4.eps}};
    \node[above=0cm of upper_brain] (stage) {Stage 4};
    % the balls
    
    \end{tikzpicture}
  %\end{subfigure}
  % next subfigure
  \hspace{-1.5em}
  ~
  %\begin{subfigure}[b]{0.15\textwidth}
    \begin{tikzpicture}[scale=\snapScale,auto,swap]

    % the two brain figures on top
    \node (upper_brain) at (0,1.5) { \includegraphics*[scale=\scaleBrainImg,trim=0 0 240 0]{\snapLocationEAR/stage_5.eps}};
    \node (lower_brain) at (0,-1.5) { \includegraphics*[scale=\scaleBrainImg,trim=240 0 0 0]{\snapLocationEAR/stage_5.eps}};
    \node[above=0cm of upper_brain] (stage) {Stage 5};
    % the balls
    
    \end{tikzpicture}
  %\end{subfigure}
  % next subfigure
  \hspace{-1.5em}
  ~
  %\begin{subfigure}[b]{0.15\textwidth}
    \begin{tikzpicture}[scale=\snapScale,auto,swap]

    % the two brain figures on top
    \node (upper_brain) at (0,1.5) { \includegraphics*[scale=\scaleBrainImg,trim=0 0 240 0]{\snapLocationEAR/stage_6.eps}};
    \node (lower_brain) at (0,-1.5) { \includegraphics*[scale=\scaleBrainImg,trim=240 0 0 0]{\snapLocationEAR/stage_6.eps}};
    \node[above=0cm of upper_brain] (stage) {Stage 6};
    % the balls
    
    \end{tikzpicture}
  %\end{subfigure}
  % next subfigure
  \hspace{-1.5em}
  ~
  %\begin{subfigure}[b]{0.15\textwidth}
    \begin{tikzpicture}[scale=\snapScale,auto,swap]

    % the two brain figures on top
    \node (upper_brain) at (0,1.5) { \includegraphics*[scale=\scaleBrainImg,trim=0 0 240 0]{\snapLocationEAR/stage_7.eps}};
    \node (lower_brain) at (0,-1.5) { \includegraphics*[scale=\scaleBrainImg,trim=240 0 0 0]{\snapLocationEAR/stage_7.eps}};
    \node[above=0cm of upper_brain] (stage) {Stage 7};
    % the balls
    
    \end{tikzpicture}

    \hspace{-1.5em}


    \vspace{-1em}

    {\Large \textbf{Space perception impairment group}\par}
    \begin{tikzpicture}[scale=\snapScale,auto,swap]

    % the two brain figures on top
    \node (upper_brain) at (0,1.5) { \includegraphics*[scale=\scaleBrainImg,trim=0 0 240 0]{\snapLocationSPA/stage_1.eps}};
    \node (lower_brain) at (0,-1.5) { \includegraphics*[scale=\scaleBrainImg,trim=240 0 0 0]{\snapLocationSPA/stage_1.eps}};
    \node[above=0cm of upper_brain] (stage) {Stage 1};
    % the balls
    
    \end{tikzpicture}
  %\end{subfigure}
  % next subfigure
  \hspace{-1.5em}
  ~
  %\begin{subfigure}[b]{0.15\textwidth}
    \begin{tikzpicture}[scale=\snapScale,auto,swap]

    % the two brain figures on top
    \node (upper_brain) at (0,1.5) { \includegraphics*[scale=\scaleBrainImg,trim=0 0 240 0]{\snapLocationSPA/stage_2.eps}};
    \node (lower_brain) at (0,-1.5) { \includegraphics*[scale=\scaleBrainImg,trim=240 0 0 0]{\snapLocationSPA/stage_2.eps}};
    \node[above=0cm of upper_brain] (stage) {Stage 2};
    % the balls
    
    \end{tikzpicture}
  %\end{subfigure}
  % next subfigure
  \hspace{-1.5em}
  ~
  %\begin{subfigure}[b]{0.15\textwidth}
    \begin{tikzpicture}[scale=\snapScale,auto,swap]

    % the two brain figures on top
    \node (upper_brain) at (0,1.5) { \includegraphics*[scale=\scaleBrainImg,trim=0 0 240 0]{\snapLocationSPA/stage_3.eps}};
    \node (lower_brain) at (0,-1.5) { \includegraphics*[scale=\scaleBrainImg,trim=240 0 0 0]{\snapLocationSPA/stage_3.eps}};
    \node[above=0cm of upper_brain] (stage) {Stage 3};
    % the balls
    
    \end{tikzpicture}
  %\end{subfigure}
  % next subfigure
  \hspace{-1.5em}
  ~
  %\begin{subfigure}[b]{0.15\textwidth}
    \begin{tikzpicture}[scale=\snapScale,auto,swap]

    % the two brain figures on top
    \node (upper_brain) at (0,1.5) { \includegraphics*[scale=\scaleBrainImg,trim=0 0 240 0]{\snapLocationSPA/stage_4.eps}};
    \node (lower_brain) at (0,-1.5) { \includegraphics*[scale=\scaleBrainImg,trim=240 0 0 0]{\snapLocationSPA/stage_4.eps}};
    \node[above=0cm of upper_brain] (stage) {Stage 4};
    % the balls
    
    \end{tikzpicture}
  %\end{subfigure}
  % next subfigure
  \hspace{-1.5em}
  ~
  %\begin{subfigure}[b]{0.15\textwidth}
    \begin{tikzpicture}[scale=\snapScale,auto,swap]

    % the two brain figures on top
    \node (upper_brain) at (0,1.5) { \includegraphics*[scale=\scaleBrainImg,trim=0 0 240 0]{\snapLocationSPA/stage_5.eps}};
    \node (lower_brain) at (0,-1.5) { \includegraphics*[scale=\scaleBrainImg,trim=240 0 0 0]{\snapLocationSPA/stage_5.eps}};
    \node[above=0cm of upper_brain] (stage) {Stage 5};
    % the balls
    
    \end{tikzpicture}
  %\end{subfigure}
  % next subfigure
  \hspace{-1.5em}
  ~
  %\begin{subfigure}[b]{0.15\textwidth}
    \begin{tikzpicture}[scale=\snapScale,auto,swap]

    % the two brain figures on top
    \node (upper_brain) at (0,1.5) { \includegraphics*[scale=\scaleBrainImg,trim=0 0 240 0]{\snapLocationSPA/stage_6.eps}};
    \node (lower_brain) at (0,-1.5) { \includegraphics*[scale=\scaleBrainImg,trim=240 0 0 0]{\snapLocationSPA/stage_6.eps}};
    \node[above=0cm of upper_brain] (stage) {Stage 6};
    % the balls
    
    \end{tikzpicture}
  %\end{subfigure}
  % next subfigure
  \hspace{-1.5em}
  ~
  %\begin{subfigure}[b]{0.15\textwidth}
    \begin{tikzpicture}[scale=\snapScale,auto,swap]

    % the two brain figures on top
    \node (upper_brain) at (0,1.5) { \includegraphics*[scale=\scaleBrainImg,trim=0 0 240 0]{\snapLocationSPA/stage_7.eps}};
    \node (lower_brain) at (0,-1.5) { \includegraphics*[scale=\scaleBrainImg,trim=240 0 0 0]{\snapLocationSPA/stage_7.eps}};
    \node[above=0cm of upper_brain] (stage) {Stage 7};
    % the balls
    
    \end{tikzpicture}
  %\end{subfigure}
  % next subfigure
  \hspace{-1.5em}


\vspace{-1em}


%   \centering
  %\begin{subfigure}[b]{0.15\textwidth}
    {\Large \textbf{Object perception impairment group}\par}
    \begin{tikzpicture}[scale=\snapScale,auto,swap]

    % the two brain figures on top
    \node (upper_brain) at (0,1.5) { \includegraphics*[scale=\scaleBrainImg,trim=0 0 240 0]{\snapLocationPER/stage_1.eps}};
    \node (lower_brain) at (0,-1.5) { \includegraphics*[scale=\scaleBrainImg,trim=240 0 0 0]{\snapLocationPER/stage_1.eps}};
    \node[above=0cm of upper_brain] (stage) {Stage 1};
    % the balls
    
    \end{tikzpicture}
  %\end{subfigure}
  % next subfigure
  \hspace{-1.5em}
  ~
  %\begin{subfigure}[b]{0.15\textwidth}
    \begin{tikzpicture}[scale=\snapScale,auto,swap]

    % the two brain figures on top
    \node (upper_brain) at (0,1.5) { \includegraphics*[scale=\scaleBrainImg,trim=0 0 240 0]{\snapLocationPER/stage_2.eps}};
    \node (lower_brain) at (0,-1.5) { \includegraphics*[scale=\scaleBrainImg,trim=240 0 0 0]{\snapLocationPER/stage_2.eps}};
    \node[above=0cm of upper_brain] (stage) {Stage 2};
    % the balls
    
    \end{tikzpicture}
  %\end{subfigure}
  % next subfigure
  \hspace{-1.5em}
  ~
  %\begin{subfigure}[b]{0.15\textwidth}
    \begin{tikzpicture}[scale=\snapScale,auto,swap]

    % the two brain figures on top
    \node (upper_brain) at (0,1.5) { \includegraphics*[scale=\scaleBrainImg,trim=0 0 240 0]{\snapLocationPER/stage_3.eps}};
    \node (lower_brain) at (0,-1.5) { \includegraphics*[scale=\scaleBrainImg,trim=240 0 0 0]{\snapLocationPER/stage_3.eps}};
    \node[above=0cm of upper_brain] (stage) {Stage 3};
    % the balls
    
    \end{tikzpicture}
  %\end{subfigure}
  % next subfigure
  \hspace{-1.5em}
  ~
  %\begin{subfigure}[b]{0.15\textwidth}
    \begin{tikzpicture}[scale=\snapScale,auto,swap]

    % the two brain figures on top
    \node (upper_brain) at (0,1.5) { \includegraphics*[scale=\scaleBrainImg,trim=0 0 240 0]{\snapLocationPER/stage_4.eps}};
    \node (lower_brain) at (0,-1.5) { \includegraphics*[scale=\scaleBrainImg,trim=240 0 0 0]{\snapLocationPER/stage_4.eps}};
    \node[above=0cm of upper_brain] (stage) {Stage 4};
    % the balls
    
    \end{tikzpicture}
  %\end{subfigure}
  % next subfigure
  \hspace{-1.5em}
  ~
  %\begin{subfigure}[b]{0.15\textwidth}
    \begin{tikzpicture}[scale=\snapScale,auto,swap]

    % the two brain figures on top
    \node (upper_brain) at (0,1.5) { \includegraphics*[scale=\scaleBrainImg,trim=0 0 240 0]{\snapLocationPER/stage_5.eps}};
    \node (lower_brain) at (0,-1.5) { \includegraphics*[scale=\scaleBrainImg,trim=240 0 0 0]{\snapLocationPER/stage_5.eps}};
    \node[above=0cm of upper_brain] (stage) {Stage 5};
    % the balls
    
    \end{tikzpicture}
  %\end{subfigure}
  % next subfigure
  \hspace{-1.5em}
  ~
  %\begin{subfigure}[b]{0.15\textwidth}
    \begin{tikzpicture}[scale=\snapScale,auto,swap]

    % the two brain figures on top
    \node (upper_brain) at (0,1.5) { \includegraphics*[scale=\scaleBrainImg,trim=0 0 240 0]{\snapLocationPER/stage_6.eps}};
    \node (lower_brain) at (0,-1.5) { \includegraphics*[scale=\scaleBrainImg,trim=240 0 0 0]{\snapLocationPER/stage_6.eps}};
    \node[above=0cm of upper_brain] (stage) {Stage 6};
    % the balls
    
    \end{tikzpicture}
  %\end{subfigure}
  % next subfigure
  \hspace{-1.5em}
  ~
  %\begin{subfigure}[b]{0.15\textwidth}
    \begin{tikzpicture}[scale=\snapScale,auto,swap]

    % the two brain figures on top
    \node (upper_brain) at (0,1.5) { \includegraphics*[scale=\scaleBrainImg,trim=0 0 240 0]{\snapLocationPER/stage_7.eps}};
    \node (lower_brain) at (0,-1.5) { \includegraphics*[scale=\scaleBrainImg,trim=240 0 0 0]{\snapLocationPER/stage_7.eps}};
    \node[above=0cm of upper_brain] (stage) {Stage 7};
    % the balls
    
    \end{tikzpicture}
  %\end{subfigure}
  % next subfigure
  \hspace{-1.5em}

  
 \begin{subfigure}{0.32\textwidth}
 \centering
 {\footnotesize \textbf{Vision}\par}
 \includegraphics[width=\textwidth]{\pcaPaperFigs/posVarianceMatSmall10EAR.png}
 \end{subfigure}
  \begin{subfigure}{0.32\textwidth}
  \centering
  {\footnotesize \textbf{Space}\par}
 \includegraphics[width=\textwidth]{\pcaPaperFigs/posVarianceMatSmall10SPA.png}
 \end{subfigure}
  \begin{subfigure}{0.32\textwidth}
  \centering
  {\footnotesize \textbf{Object}\par}
 \includegraphics[width=\textwidth]{\pcaPaperFigs/posVarianceMatSmall10PER.png}
 \end{subfigure}
 \caption[Early atrophy progression within the three cognitively-defined PCA subgroups]{Early atrophy progression within the three cognitively-defined PCA subgroups, as estimated by the EBM. The top figures shows snapshots of the atrophy patterns for the first 7 stages in the EBM, while the last row shows the uncertainty in the atrophy progression sequence. Brain pictures generated using BrainPainter \cite{marinescu2019BrainPainter}}
 \label{fig:pcaSubgrSnap}
\end{figure}

\section{Discussion}
\label{sec:pcaDis}

% summary of findings
In this work we performed one of the first longitudinal studies of atrophy progression in PCA. Results suggest that in PCA occipital and superior parietal areas are the first to become affected, followed by temporal areas. By 10 years after $t_0$, there seems to be widespread atrophy in the occipital, parietal and temporal areas, as well as ventricular expansion. In contrast, tAD seems to have significant early atrophy in the hippocampus, with subsequent temporal atrophy and ventricular expansion starting 5 years after $t_0$. 

% talk about heterogeneity 
Regarding PCA heterogeneity, our study also provided the first glimpse into the early longitudinal patterns of atrophy within three cognitively defined PCA subtypes. We found early phenotype-specific patterns of atrophy within each cognitively-defined PCA subgroup. These patterns of pathology overlap with the pathways that are hypothesised to be affected within each group: striate cortex for the vision subgroup, dorsal pathway for the space subgroup and ventral pathway for the object subgroup. Nonetheless, among the subgroups there is considerable variability in these patterns as well as spatial overlap, which might suggest that these should not necessarily be interpreted as distinct diseases, but rather that the patients lie on a continuum of phenotypical variation, as suggested by \cite{lehmann2011basic}. 

% Strengths of the EBM and DEM methodology 
Our study has several strengths. First of all, the large number of PCA subjects with longitudinal neuroimaging and cognitive data allowed us to perform a robust analysis of PCA atrophy progression. The EBM and DEM methods we used are all data-driven, don't require manual biomarker thresholds and don't rely on diagnostic classes, which are often noisy and biased. Moreover, the ability of the EBM to work with limited cross-sectional data allowed us to estimate the progression of PCA subgroups, which are small and have limited longitudinal data available.  An advantage of the DEM method is its ability to fit continuous, non-parametric biomarker trajectories based on GPs, which makes it suitable for modelling biomarkers whose trajectories have varying shapes.

% Limitations
Nevertheless, our study has several limitations that need to be addressed. First of all, since data was acquired over an extended period of time, not all subjects had CSF, molecular or pathological confirmation for Alzheimer's disease. This can be a problem, as previous studies suggested that at least half of patients who receive a  diagnosis of probable AD actually have other non-AD underlying pathologies \cite{schneider2007mixed,schneider2009neuropathology}. Follow-up studies will need to have a higher proportion of patients with pathological or molecular confirmation. Moreover, the data was acquired in three different centres using different scanners and field strengths, although we adjusted for these covariates. 

The EBM and DEM models that we employed also have several limitations that we acknowledge. First of all, both methodologies assume all subjects follow the same progression sequence. Secondly, the DEM requires longitudinal data, which prevented us from fitting the DEM to the PCA subgroups, who lacked enough longitudinal data. Another assumption made by the EBM is that the control population is well-defined, as we fit the distribution of normal biomarker values directly on the biomarker values of the control population. The EBM also assumes simplistic, step-wise biomarker trajectories that switch from a normal to an abnormal value. With respect to the DEM, the approach requires a reference timepoint, which we took it to be the threshold that best separates the controls from patients after disease staging.  

% Future studies 
There are several avenues for future research. Further molecular and pathological confirmation can be obtained for the remaining patients to ensure they all have a reliable diagnosis, which will enable an unbiased estimation of the progression sequence. The EBM and DEM methodologies can be further extended to allow random effects or to fit different progression sequences for different sub-populations in a data-driven way, such as the approach of \cite{young2015multiple}. Information about the rate and extent of atrophy in the PCA subgroups can also be computed after enough data has been acquired. A well-defined control population for the EBM can also be defined by selecting only amyloid-negative subjects or by other types of stratification. The EBM model can be extended to model more complex trajectory shapes, while the DEM can be further extended to a multivariate approach that inherently aligns the biomarker trajectories.

Finally, one of the key directions of future research is to understand the disease mechanisms underlying PCA. To this end, several methods can be used to estimate these mechanisms, such as those based on propagation of pathogenic proteins \cite{raj2012network, georgiadis2018computational} or the architecture of brain networks \cite{zhou2012predicting}. The influence of genetic factors such as Alipoprotein E (APOE) status \cite{schott2006apolipoprotein, snowden2007cognitive} and other factors recently identified \cite{schott2016genetic, schott2006apolipoprotein} from genome-wide association studies also need to be understood. This research will lead the way towards drug development in PCA clinical trials and will allow the selection of robust outcome measures and fine-grained patient stratification in clinical trials in PCA. 

\section{Conclusion}

In this work I performed a statistical analysis of the neuroimaging data from PCA and tAD subjects from the DRC, HUVR and UCSF centres. I pre-processed all the MRI images and applied the event-based model and the differential equation model on the PCA and tAD cohorts, as well as on three cognitively-defined PCA subgroups. The analysis I made gives the first glimpse into the longitudinal progression of atrophy in PCA subjects, and into the early longitudinal patterns of atrophy in the vision, space and object subgroups.

In the following chapter, I will present some novel extensions to the EBM and DEM models that will enable better estimation of the parameters for the EBM and alignment of the biomarker trajectories for the DEM. These improvements can provide a more accurate disease signature, and remove the need for ad-hoc methods of estimating parameters.































\chapter[Novel Extensions to the EBM and DEM]{Novel Extensions to the Event-based Model and Differential Equation Model}
\label{chapter:perf}

\section{Contributions}

In this work I present methodological extensions to the event-based model (EBM) and differential equation model (DEM) and I evaluate their performance compared to the standard implementations. In order to assess differences between these methods more accurately, I also propose novel performance measures based on disease staging consistency and prediction of time elapsed between visits. I formulated and implemented the novel methodologies, and performed their evaluation. I also pre-processed the DRC MRI scans. My colleague Alexandra Young pre-processed the ADNI data. 

\section{Introduction}

% validation of DPMs
% DPMs have limitations/overfitting -> need to test them -> but no ground truth -> aim is to eval. perf. w/o ground truth -> perf. metrics -> test them on EBM and DEM -> impact
Many data-driven disease progression models (DPMs) that have been presented in chapter \ref{chapter:bckDpm} make assumptions about the biomarker data and the model parameters, which limit their usefulness on practical applications. For example, the differential equation model by \cite{villemagne2013amyloid} is univariate, hence it assumes independence across different biomarkers. In order to place biomarker trajectories on the same time frame, in the previous chapter we used a post-hoc anchoring process (see section \ref{sec:pcaStaMetDEM}). This anchoring is inaccurate, as it relies on setting the reference time $t_0$ using biomarker values of a clinical group (i.e. controls or AD). This anchoring process is challenging because of singularities arising from flat trajectories\footnote{The alignment is performed by setting $t_0=0$ so that $f(t_0) = mean(patients)$. However, if the trajectory is flat then there are many points  $t_0$ that match the mean of patients. Even is the trajectory is not fully flat, the measurement noise is amplified by the low slope of $f$.}, and the fact that subjects are at different stages along the disease. Another limitation of some DPMs is that the fitting algorithm assumes independence between different sets of parameters. While this is done in order to ensure computational tractability, this yields inaccurate parameter estimates. In particular, the event-based model parameter estimation procedure proposed by \cite{fonteijn2012event} and \cite{young2014data} assumes that the parameters of the likelihood models for normal and abnormal values are independent of the abnormality sequence. Some better parameter estimation procedures are therefore needed, which can ensure a robust data fit.

The evaluation of the performance of disease progression models is another open problem that has not been addressed so far. While previous studies used accuracy of clinical status predictions\cite{young2014data}, clinical diagnosis is often not reliable without neuropathological confirmation -- one study reported that a clinical diagnosis of \emph{probable AD} has between 70.9--87.3\% accuracy and between 44.3\%--70.8\% specificity. Therefore, performance metrics based on the prediction of clinical diagnosis might not be sufficiently sensitive to differences in the performance of such algorithms. While \cite{fonteijn2012event} computed the number of subjects with increased staging over time -- a performance measure that doesn't rely on clinical diagnosis -- it does not take model uncertainty of staging into account and it is specific to discrete models such as the event-based model. 

In this chapter we suggest novel extensions in the event-based model and differential equation model and propose four novel performance measures for evaluating disease progression models that don't rely on clinical diagnosis. For the event-based model, we devise two novel fitting procedures that perform joint optimisation of the parameters of the normal and abnormal likelihood models, as well as the abnormality sequence. For the differential equation model, we devise a novel data-driven way to align the biomarker trajectories to a common axis by estimating trajectory-specific and subject-specific time shifts. The novel performance measures that we propose exploit uncertainty in the estimated stages and are also suitable for evaluating continuous trajectory models. Using data from the Alzheimer's Disease Neuroimaging Initiative (ADNI) and the Dementia Research Centre (DRC), UK, we show that the novel models generally have better or equal performance compared to standard models. Moreover, we also show that the novel performance measures that we proposed are more sensitive to changes in models than standard measures based on the prediction of diagnosis or conversion status. 

\section{Methods}

\subsection{EBM Extensions}

In this section we outline two novel methods of parameter fitting for the event-based model: a blocked MCMC sampling of the distribution parameters and event ordering (section \ref{sec:simultSampling}), and an Expectation-Maximisation approach (section \ref{sec:ebmEM}). Furthermore, we also present a novel methodology performing a data-driven temporal alignment of the differential equation model trajectories (section \ref{sec:demOptim}). 

\subsubsection{EBM -- Joint MCMC Sampling}
\label{sec:simultSampling}

We present a novel method for fitting the event-based model that jointly optimises the parameters of the normal and abnormal likelihood models using MCMC sampling. The full EBM likelihood model, already been described in Eq. \ref{eq:ebm4}, is:

\begin{equation}
\label{ap:ebmMain}
 p(X|S) = \prod_{i=1}^P \left[ \sum_{k=0}^N p(k) \left( \prod_{j=1}^k p\left(x_{i,s(j)} | E_{s(j)} \right) \prod_{j=k+1}^N p\left(x_{i,s(j)} | \neg E_{s(j)}\right) \right) \right]
\end{equation}
where $x_{ij}$ represents the value of biomarker $j$ from subject $i$  and is informative of event $E_j$ in subject $i$, $P$ is the number of subjects and $N$ is the number of biomarkers. The abnormality sequence $S = [S(1), \dots, S(N)]$ describes the order in which events $E_1, E_2, \dots , E_N$ become abnormal, and models disease progression. 

We now reformulate the EBM likelihood model to explicitly take into account the parameters of the likelihood models for the normal and abnormal biomarker values. This will allow joint optimisation of these parameters, along with sequence parameter $S$. We assume the likelihood models for normal and abnormal biomarker values are Gaussian distributions, i.e. $p(x|E_j) \sim N(\mu^a_j, \sigma^a_j)$ and $p(x|\neg E_j) \sim N(\mu^n_j, \sigma^n_j)$ where $\mu^a_j$ and $\sigma^a_j$ model the distribution of abnormal values for biomarker $j$ (i.e. event $E_j$ occurred), while $\mu^n_j$ and $\sigma^n_j$ model the distribution of normal values for biomarker $j$ (event $E_j$ did not occur). Thus, the full set of parameters that need to be modelled is $\theta = \left[ [\mu^n_j, \sigma^n_j, \mu^a_j, \sigma^a_j]^{j=1 \dots N}, S \right]$. Therefore, the likelihood in equation \ref{ap:ebmMain} can be explicitly written as:


\begin{multline}
\label{ap:ebmExplicit}
 p(X|S, [\mu^n_j, \sigma^n_j, \mu^a_j, \sigma^a_j]^{j=1 \dots N}) = \\ \prod_{i=1}^P \left[ \sum_{k=0}^N p(k) \left( \prod_{j=1}^k p\left(x_{i,S(j)} | \mu^a_{S(j)}, \sigma^a_{S(j)} \right) \prod_{j=k+1}^N p\left(x_{i,S(j)} | \mu^n_{S(j)}, \sigma^n_{S(j)} \right) \right) \right]
\end{multline}


We maximise this likelihood using blocked MCMC sampling, where at each step we only propose parameters for biomarker $j$, i.e. $[\mu^n_j, \sigma^n_j, \mu^a_j, \sigma^a_j]$ along with a new sequence $S_j^{new}$ where only event $E_j$ changed its position. The distribution parameters for the other biomarkers and the ordering of the other events $i \neq j$ are kept the same. This blocked approach can lead to faster convergence because there is strong dependence between parameters corresponding to the same biomarker and between the position of the corresponding event in the sequence. The covariance matrix of the proposal distribution is estimated by taking 100 bootstraps of the dataset and computing the covariance of $[\mu^c, \sigma^c, \mu^p, \sigma^p]$, where $\mu^c$, $\sigma^c$ are the mean and standard deviation of the control group while $\mu^p$, $\sigma^p$ are the mean and standard deviation of the patient group.

\subsubsection{EBM -- Expectation Maximisation}
\label{sec:ebmEM}

The blocked MCMCM approach from the previous section can be challenging to implement and execute, due to the difficulty of sampling in a high-dimensional space. The user needs to further tune the covariance matrix in order to get a good acceptance rate, and ensure enough samples are taken in order to exhaustively explore the space of the distribution. To mitigate these issues, we further propose a novel parameter estimation procedure for the EBM based on the Expectation Maximisation (EM) framework \cite{bishop2007pattern}. The EM framework is suitable for estimating parameters of models with discrete latent variables, such as the EBM which has discrete latent variables $k$ representing the subject-specific stages. The EM framework tries to find the parameters $\theta^*$ that maximise the expected log-likelihood of the complete data $\theta^* = \argmax_{\theta} Q(\theta | \theta^{old}) = \argmax_{\theta} \mathbb{E}_{Z|X,\theta^{old}}[log\ p(X,Z|\theta)]$. The key observation to make is that the joint likelihood over the latent variables $Z = [Z_1, ..., Z_N]$ and $X = [X_1, ..., X_N]$ factorises, giving the following form:

\begin{equation}
Q(\theta | \theta^{old}) = \sum_{i=1}^P \sum_{z_i} p(Z_i = z_i|X_i, \theta^{old}) \left[ \sum_{j=1}^{z_i} log\ p(x_{ij}|E_{S(j)}) + \sum_{j=z_i + 1}^N log\ p(x_{ij}| \neg E_{S(j)}) \right]
\end{equation}

We find the maximum for $\mu_k^n$, the mean of $p(x|\neg E_k)$, by solving $\frac{d}{d\mu_k^n}Q(\theta | \theta^{old}) = 0 $. This gives the following update equation for $\mu_k^n$:
\begin{equation}
 \mu_k^n = \sum_{i=1}^P x_{ik} w_i^n
\end{equation}
with weights $w_i^n$ defined as:
\begin{equation}
w_i^n = \frac{p(S^{-1}(k) > Z_i | X, \theta^{old})}{\sum_{i=1}^P \ p(S^{-1}(k) > Z_i | X, \theta^{old})}
\end{equation}
and $p(S^{-1}(k) > Z_i | X, \theta^{old}) = \sum_{l=S^{-1}(k)+1}^{K} p(Z_i = l | X, \theta^{old})$. The full derivation is given in section \ref{sec:appEbmEm}. Similar update rules are derived in the appendix section \ref{sec:appEbmEm} for the other parameters: $\sigma_k^n$, $\mu_k^a$, $\sigma_k^a$. 

Optimising the sequence $S$ in the M-step is intractable, so we use MCMC sampling where at each step of the sampling process we propose a new sequence $S^{new}$, find the optimal distribution parameters for each biomarker given $S^{new}$ using the closed-form EM update rules, and then evaluate the likelihood $Q(\theta | \theta^{old})$. We keep performing a greedy ascent as performed by \cite{fonteijn2012event} until convergence. Although this approach might not guarantee that we truly find the optimal parameters, it still results in an increase of $Q(\theta | \theta^{old})$. This approach, called generalised EM, guarantees that the method will converge to a local maxima \cite{bishop2007pattern}. For parameter initialisation, we use the mean and standard deviation of the control and patient populations. 

\subsection{DEM -- Optimised Trajectory Alignment}
\label{sec:demOptim}

We present a novel extension to the DEM (see section \ref{sec:bckDem}) that aims to place the estimated biomarker trajectories on the same temporal axis, in a data-driven way. The main idea is to use the subjects' data to find optimal subject-specific and trajectory-specific time shifts. Since we are mostly interested in estimating population-level trajectories and to align them on a common time-axis, we do not currently add subject-specific progression speeds and random-effect deviations from the average trajectories.

Let us denote by $X$ our dataset, where $x_{pb}$ is the measurement of biomarker $b$ in patient $p$ and $f_b$ is the shape of the trajectory for biomarker $b$. For every biomarker $b$, we aim to estimate a temporal shift $t_b$ of the trajectory and a measurement noise $\sigma_b$. At the same time. The log-likelihood for the data $X_p$ from patient $p$ can be expressed as:

\beq
p(X_p| t_1, \dots, t_B, \sigma_1, \dots , \sigma_B, z_p ) = \prod_{b=1}^{B}  N(x_{pb}|f_b(z_p-t_b), \sigma_b)
\eeq

where $z_p$ is a latent parameter representing the time-shift of subject $p$. Multiplying by the prior on $z_p$ and summing over all the possible values of $z_p$ we get the marginal:
\beq
p(X_p| t_1, \dots, t_B, \sigma_1, \dots , \sigma_B) = \sum_{z_p} p(Z_p = z_p) \prod_{b=1}^{B}  N(x_{pb}|f_b(z_p-t_b), \sigma_b)
\eeq

Assuming the data from each subject is conditionally independent given $z_p$, we get the full likelihood:
\beq
p(X| t_1, \dots, t_B, \sigma_1, \dots , \sigma_B) = \prod_{p=1}^{P} \sum_{z_p} p(Z_p = z_p) \prod_{b=1}^{B}  N(x_{pb}|f_b(z_p-t_b), \sigma_b)
\eeq

This likelihood can be optimised with any method of choice such as MCMC sampling or gradient methods. We chose to optimise the model using an iterative approach, where for each biomarker $b$ we optimise it's trajectory shift $t_b$ conditioned on all the other parameters (Markov blanket), and then estimate it's measurement noise $\sigma_b$.


\subsection{Performance Evaluation}
\label{sec:perfEvalMethods}

We compare the performance of the extended EBM and DEM methods against the standard implementations, using novel performance metrics that we propose. In section \ref{sec:stagingConsist}, we present metrics which test staging consistency, i.e. that follow-up stages are greater or equal to baseline stages. We then generalise this concept in section \ref{sec:timeLapse} to test the accuracy of models in predicting the time lapse between two visits of a subject.

\subsubsection{Staging Consistency}
\label{sec:stagingConsist}

The staging consistency metrics test whether subjects' stages at follow-up visits are greater than or equal to the stages at baseline. While such a metric is simple to compute in cases of no uncertainty, we also define a more complex metric that takes staging uncertainty into account. 

Let us consider a set of random variables $z_t^i$ representing the stage of subject $i$ at timepoint $t$, where $i \in [1 \dots N], t \in [1 \dots T_i]$, $N$ being the number of subjects and $T_i$ the number of time points for subject $i$. For most disease progression models, the EBM and DEM included, we can find the posterior $p(z_t^i|X_i, \theta)$, which we will call the staging probabilities. Moreover, let $M^i_t = \argmax_s p(z_t^i = s)$ be the maximum likelihood stage for subject $i$ at time point $t$. The \emph{hard staging consistency} $C_h$ counts the proportion of stages from consecutive visits of every subject where the stage at the later visit must be greater than the stage at the earlier visit. We define $C_h$ as follows:

\begin{equation}
 C_h = \frac{1}{-N +\sum_{i=1}^N T_i} \sum_{i=1}^N \sum_{t=2}^{T_i} \mathbb{I}[M^i_t > M^i_{t-1}] 
\end{equation}
where the element $-N +\sum_{i=1}^N T_i$ is a normalising constant that represents the number of pairs of consecutive visits from all subjects and time points in the dataset. The $C_h$ metric ranges from 0 (no consistent pairs of stages) to 1 (all pairs of stages are consistent).


We further seek to generalise the \emph{hard staging consistency}, in order to make use of the full staging probabilities, instead of using only the maximum likelihood stages. We define the \emph{soft staging consistency} $C_s^i(t_1,t_2)$ for subject $i$ given two time points $t_1$ and $t_2$ as:

\begin{equation}
C_s^i(t_1,t_2) = p(z^i_{t_1} \leq z^i_{t_2}) = \sum_{s \in S} p(z^i_{t_2} = s) p(z^i_{t_1} \leq s) 
\end{equation}
where $S$ is the set of possible stages in the disease progression model. We then define the \emph{soft staging consistency} for the whole population as the mean of subject-specific consistencies for consecutive timepoints:

\begin{equation}
C_s = \frac{1}{-N +\sum_{i=1}^N T_i} \sum_{i=1}^N \sum_{t=2}^{T_i} C_s^i(t_1,t_2) 
\end{equation}

% time-difference consistency
\subsubsection{Time-lapse Prediction}
\label{sec:timeLapse}

Time-lapse prediction is a generalisation of the staging consistency, where the disease progression model needs to predict how much time passed between two visits of the same subject, which is the compared against the true time elapsed. This is only possible for models that estimate continuous latent stage variables, such as the DEM. We define the \emph{hard time-lapse} metric $D_h$ as follows:

\begin{equation}
D_h = \frac{1}{-N +\sum_{i=1}^N T_i} \sum_{i=1}^N \sum_{t=2}^{T_i} \left| \tau(M^i_t) - \tau(M^i_{t-1}) - (a^i_t - a^i_{t-1}) \right|
\end{equation}
where $a^i_t$ is the age of subject $i$ at timepoint $t$, $M^i_t$ is the maximum likelihood stage for subject $i$ at timepoint $t$ and $\tau(M^i_t)$ is the estimated time from onset associated with stage $M^i_t$. The equivalent \emph{soft time-lapse} metric $D_s$, which uses probabilistic staging variables $z_t^i$, is defined as:
\begin{equation}
D_s = \frac{1}{-N +\sum_{i=1}^N T_i} \sum_{i=1}^N \sum_{t=2}^{T_i} \left| \mathbb{E}[\tau(z^i_t) - \tau(z^i_{t-1})] - (a^i_t - a^i_{t-1}) \right|
\end{equation}

\subsection{Data Preprocessing}

\subsection{The Dementia Research Centre Cohort}
\label{sec:dataPrep}

% DRC - data summary
The Dementia Research Centre (DRC), UK cohort contains 89 controls, 74 PCA and 67 tAD subjects that have undergone 1.5T/3T T1-weighted MRI scans,  with an average of 2-3 visits per subject. The demographics of the DRC dataset is given in table \ref{tab:drcDemographics}. More details on the cohort can be found in the PCA longitudinal study from section \ref{sec:pcaParticipants}\footnote{This cohort is a subset of the cohort used in the PCA longitudinal study.}.

\begin{table}[ht]
\centering
 \begin{tabular}{c | c c c}
  Demographics & CN & PCA & AD\\
  \hline
  Number & 89 & 74 & 67\\
  Sex M/F & 33/56 & 28/46 & 35/32 \\
  Age (years) & 61 $\pm$ 11 & 63 $\pm$ 7 & 66 $\pm$ 9\\
  Years from onset & - & 4.5 $\pm$ 2.8 & 4.8 $\pm$ 2.6\\
  Number of visits & 2.8 $\pm$ 2.5 & 2.5 $\pm$ 1.7 & 3.0 $\pm$ 2.7\\
 \end{tabular}
 \caption[Baseline population demographics for DRC data]{Baseline population demographics for the DRC cohort.}
 \label{tab:drcDemographics}
\end{table}

\subsubsection{Image Processing}
The MRI scans were segmented using the Geodesic Information Flows (GIF) algorithm by \cite{cardoso2015geodesic}, which is available as a service at \url{http://cmictig.cs.ucl.ac.uk/niftyweb/}. The atlas that has been used for segmentation is the Neuromorphometrics atlas (provided by Neuromorphometrics, Inc.), which produced 146 different brain ROIs across the left and right hemisphere. All brain volumes have been corrected for total intracranial volume (TIV), age and gender using a general linear model. We summed left and right brain regions together and further selected a subset of 25 ROIs: (a) whole brain; (b) ventricles; (c) 2 subcortical regions: hippocampus and amygdala; (d) 5 occipital regions: inferior, middle and superior occipital and the occipital fusiform and lingual; (e) 5 parietal regions: superior parietal, angular, precuneus, supramarginal and postcentral; (f) 4 temporal regions: inferior, middle and superior temporal along with fusiform; (g) 4 frontal regions: superior, middle and inferior frontal along with precentral; and (h) 3 limbic regions: entorhinal, parahippocampal and posterior cingulate.

\subsection{The Alzheimer's Disease Neuroimaging Initiative Cohort}
% ADNI - data summary

In this study we also used the ADNI dataset to evaluate our disease progression models. We used the same biomarker data as previously used by \cite{young2014data}, which included all 285 subjects (Controls, MCI and AD) that had a CSF examination at baseline, cognitive assessment at baseline and MRI scans at baseline and 1 year follow-up. The demographics of the selected subjects is shown in table \ref{tab:adniDemographics}. We also used follow-up imaging, clinical and CSF data at 12- and 24-months after baseline visit. Clinical diagnoses at baseline, 12-, 24- and 36-months were also used for evaluating performance on diagnosis prediction and conversion prediction. The CSF total tau and phosphorylated tau were log-transformed to improve normality. 

\begin{table}[ht]
\centering
 \begin{tabular}{c | c c c}
  Demographics & CN & MCI & AD\\
  \hline
  Number & 92 & 129 & 64\\
  Sex M/F & 48/44 & 82/47 & 34/30\\
  Age (years) & 75 $\pm$ 5 & 73 $\pm$ 7 & 75 $\pm$ 8\\
  Education (years) & 15.6 $\pm$ 2.9 & 15.9 $\pm$ 3 & 15 $\pm$ 3\\
  APOE +/- & 22/70 & 72/57 & 45/19\\
 \end{tabular}
 \caption[Baseline population demographics for the ADNI cohort.]{Baseline population demographics for the ADNI cohort.}
 \label{tab:adniDemographics}
\end{table}

% ADNI - data preprocessing
\subsubsection{Image Processing}

FreeSurfer Version 4.3 was used to compute regional volumes of the hippocampus, entorhinal cortex, middle temporal gyrus, fusiform gyrus, ventricles, whole brain and total intracranial volume (TIV) at baseline, 12- and 24-month follow-up. All regional volumes were normalised for each subject by dividing by TIV. Atrophy rates for the whole brain and hippocampus were estimated using the Boundary Shift Integral (BSI) (\cite{freeborough1997boundary}) using the scans at baseline and 12-months follow-up. In particular, volume change for the whole brain was measured using the KN-BSI method (\cite{leung2010robust}) and for hippocampus using the MAPS-HBSI method (\cite{leung2010automated}). 

We used the same biomarker set as the one used by \cite{young2014data}, which included 14 biomarkers in total: (a) three CSF biomarkers: amyloid-$\beta_{1-42}$, phosphorylated tau and total tau; (b) 3 cognitive tests: Alzheimer's Disease Assessment Scale - Cognitive Subscale (ADAS-Cog), Rey Auditory Verbal Learning Test (RAVLT) and the Mini-Mental State Examination (MMSE); (c) six regional brain volumes: whole brain, ventricles, hippocampus, entorhinal, middle temporal gyrus and fusiform gyrus; (d) rates of atrophy for two ROIs: hippocampus and whole brain.

\section{Results}
\label{sec:perfRes}

% fitting methods tested
We tested all novel EBM and DEM methods, along with their standard implementations. We evaluated each model using the staging consistency and time-lapse metrics, using data from the DRC and ADNI datasets. On the DRC dataset, we also evaluated the models with respect to diagnosis prediction, while on ADNI we evaluated them based on prediction of conversion from healthy controls to mild cognitive impairment (MCI) and from MCI to Alzheimer's disease.

\subsection{DRC Results}
\label{sec:perfResDrc}

We ran all the standard and novel methods for the EBM and DEM described above. For the EBM joint MCMC sampling method, we took 100,000 samples who had an acceptance rate in the range of 0.29-0.33 (min-max) across all cross-validation folds, which suggested a good mixing, while the sample autocorrelation was in the range of 0.86-0.93. In order to get the acceptance rate to this reasonable rate, we estimated the covariance matrix  of the MCMC proposal distribution based on the covariance of maximum likelihood parameter estimates on bootstrapped subsets of the full dataset. For the EBM-EM method, we performed 20 iterations as we noticed that the method converged after a maximum of 3-4 iterations.

% staging - PCA cohort
In table \ref{tab:drcStagingResPCA} we show the staging-based metrics for the PCA cohort. Each entry shows the mean and standard deviation of the metric calculated over 10 cross-validation folds. Similar results are shown for the AD cohort in table \ref{tab:drcStagingResAD}. In table \ref{tab:drcDiagRes} we show the models' balanced accuracy in diagnosis prediction on the DRC cohort. For reference, we also show similar results using a standard Support Vector Machine (SVM) classifier \cite{vapnik2006estimation}. The SVM classifier was trained with a linear kernel trained using sequential minimal optimisation \cite{platt1998sequential} and a box-constraint parameter $C=1$. In each entry we show the mean and standard deviation of the balanced classification accuracy across the 10 cross-validation folds. 

% staging metrics - PCA
\begin{table}[ht]
\centering
 \begin{tabular}{c | c | c | c | c}
  Model & \multicolumn{2}{c |}{Staging Consistency} & \multicolumn{2}{c}{Time-lapse}\\
  & Hard & Soft & Hard & Soft\\
%     \multicolumn{5}{c }{\textbf{}}\\
  \multicolumn{5}{c }{Event-based Model}\\
  \hline
  EBM - Standard & 0.88 $\pm$ 0.12 & 0.66 $\pm$ 0.09 & - & -\\ 
  EBM - MCMC & 0.96 $\pm$ 0.06 & 0.70 $\pm$ 0.06  & - & -\\
  EBM - EM & 0.95 $\pm$ 0.10 & 0.68 $\pm$ 0.11 & - & -\\
  \multicolumn{5}{c }{\textbf{}}\\
  \multicolumn{5}{c }{Differential Equation Model}\\
  \hline
  DEM - Standard & 0.94 $\pm$ 0.06 & 0.95 $\pm$ 0.05 & 0.54 $\pm$ 0.31 & 0.52 $\pm$ 0.29\\
  DEM - Optimised & 0.95 $\pm$ 0.05 & 0.95 $\pm$ 0.04 & 0.56 $\pm$ 0.28 & 0.52 $\pm$ 0.27\\
  
 \end{tabular}
 \caption[Model performance according to staging-based metrics on PCA subjects from the DRC cohort]{Model performance according to staging-based metrics on PCA subjects from the DRC cohort. The mean and standard deviations are calculated for each testing set in 10-fold cross-validation.}
 \label{tab:drcStagingResPCA}
\end{table}

% staging metrics - AD
\begin{table}[ht]
\centering
 \begin{tabular}{c | c | c | c | c}
  Model & \multicolumn{2}{c |}{Staging Consistency} & \multicolumn{2}{c}{Time-lapse}\\
  & Hard & Soft & Hard & Soft\\
  \multicolumn{5}{c }{Event-based Model}\\
  \hline
  EBM - Standard & 0.91 $\pm$ 0.16 & 0.71 $\pm$ 0.07 & - & -\\
  EBM - MCMC & 0.96 $\pm$ 0.07 & 0.76 $\pm$ 0.10 & - & -\\
  EBM - EM & 0.99 $\pm$ 0.01 & 0.72 $\pm$ 0.07 & - & -\\
  \multicolumn{5}{c }{\textbf{}}\\
  \multicolumn{5}{c }{Differential Equation Model}\\
  \hline
  DEM - Standard & 0.87 $\pm$ 0.10 & 0.88 $\pm$ 0.08 & 0.72 $\pm$ 0.91 & 0.67 $\pm$ 0.92\\
  DEM - Optimised & 0.87 $\pm$ 0.10 & 0.88 $\pm$ 0.08 & 0.74 $\pm$ 0.92 & 0.69 $\pm$ 0.92\\
  
 \end{tabular}
 \caption[Model performance according to staging-based metrics on typical AD subjects from the DRC cohort.]{Model performance according to staging-based metrics on typical AD subjects from the DRC cohort. The mean and standard deviations are calculated for each testing set in 10-fold cross-validation.}
 \label{tab:drcStagingResAD}
\end{table}

% diagnosis  
\begin{table}[H]
\centering
 \begin{tabular}{c | c c c}
  Model & PCA vs AD &  Controls vs PCA & Controls vs AD\\
  \multicolumn{4}{c }{Event-based Model}\\
  \hline
  EBM - Standard & 0.72 $\pm$ 0.13 & 0.95 $\pm$ 0.05 & 0.90 $\pm$ 0.06\\
  EBM - MCMC & 0.79 $\pm$ 0.09 & 0.94 $\pm$ 0.06 & 0.90 $\pm$ 0.05\\
  EBM - EM & 0.80 $\pm$ 0.07 & 0.95 $\pm$ 0.05 & 0.87 $\pm$ 0.05\\
  \multicolumn{4}{c }{\textbf{}}\\
  \multicolumn{4}{c }{Differential Equation Model}\\
  \hline
  DEM - Standard & 0.81 $\pm$ 0.07 & 0.95 $\pm$ 0.05 & 0.90 $\pm$ 0.11\\
  DEM - Optimised & 0.82 $\pm$ 0.09 & 0.93 $\pm$ 0.06 & 0.88 $\pm$ 0.14\\
  \multicolumn{4}{c }{\textbf{}}\\
  \multicolumn{4}{c }{Support Vector Machine}\\
  \hline
  SVM & 0.79 $\pm$ 0.14 & 0.91 $\pm$ 0.06 & 0.88 $\pm$ 0.07\\
  
 \end{tabular}
 \caption[Model performance at diagnosis prediction on DRC data.]{Model performance at diagnosis prediction on the DRC cohort. Each entry shows the mean and standard deviation of the balanced accuracy across the cross-validation folds. }
 \label{tab:drcDiagRes}
\end{table}

\subsection{ADNI Results}
\label{sec:perfResAdni}

In table \ref{tab:adniStagingRes} we show the staging-based performance results of the progression models on the ADNI dataset. As with the DRC results, for each metric we show its mean and standard deviation over the 10 cross-validation folds. In table \ref{tab:adniConvPredRes} we also evaluated the models on how well they predict conversion from MCI to AD at 12-months, 24-months and 36-months from baseline visit. We did not compute results for prediction of conversion status in controls due to small and very imbalanced datasets (i.e. only ).

% staging metrics
\begin{table}[H]
\centering
 \begin{tabular}{c | c c | c c}
  Model & \multicolumn{2}{c |}{Staging Consistency} & \multicolumn{2}{c}{Time-lapse}\\
  & Hard & Soft & Hard & Soft\\
%     \multicolumn{4}{c }{\textbf{}}\\
  \multicolumn{5}{c }{Event-based Model}\\
  \hline
  EBM - Standard & 0.83 $\pm$ 0.07 & 0.76 $\pm$ 0.05 & - & -\\ 
  EBM - MCMC & 0.84 $\pm$ 0.05 & 0.76 $\pm$ 0.06 & - & -\\
  EBM - EM & 0.84 $\pm$ 0.08 & 0.74 $\pm$ 0.06 & - & -\\
  \multicolumn{5}{c }{\textbf{}}\\
  \multicolumn{5}{c }{Differential Equation Model}\\
  \hline
  DEM - Standard & 0.87 $\pm$ 0.05 & 0.83 $\pm$ 0.08 & 0.85 $\pm$ 0.17 & 0.85 $\pm$ 0.16\\
  DEM - Optimised & 0.87 $\pm$ 0.05 & 0.84 $\pm$ 0.07 & 0.86 $\pm$ 0.15 & 0.86 $\pm$ 0.16\\
 \end{tabular}
 \caption{Model performance according to staging metrics on ADNI data.}
 \label{tab:adniStagingRes}
\end{table}

% prediction of conversion

\begin{table}[H]
\centering
 \begin{tabular}{c | p{2.5cm} p{2.5cm} p{2.5cm}}
  Model & \multicolumn{3}{c}{Duration between baseline and follow-up}\\
  
  & 12 months & 24 months & 36 months\\
%     \multicolumn{4}{c }{\textbf{}}\\
  \multicolumn{4}{c }{Event-based Model}\\
  \hline
  EBM - Standard & 0.69 $\pm$ 0.14 & 0.64 $\pm$ 0.11 & 0.72 $\pm$ 0.15\\
  EBM - MCMC & 0.66 $\pm$ 0.14 & 0.63 $\pm$ 0.10 & 0.74 $\pm$ 0.14\\
  EBM - EM & 0.69 $\pm$ 0.15 & 0.63 $\pm$ 0.10 & 0.76 $\pm$ 0.15\\
  \multicolumn{4}{c }{\textbf{}}\\
  \multicolumn{4}{c }{Differential Equation Model}\\
  \hline
  DEM - Standard & 0.73 $\pm$ 0.13 & 0.72 $\pm$ 0.14 & 0.70 $\pm$ 0.13\\
  DEM - Optimised & 0.64 $\pm$ 0.11 & 0.69 $\pm$ 0.12 & 0.75 $\pm$ 0.14\\
  \multicolumn{4}{c }{\textbf{}}\\
  \multicolumn{4}{c }{Support Vector Machine}\\
  \hline
  SVM & 0.68 $\pm$ 0.15 & 0.70 $\pm$ 0.10 & 0.77 $\pm$ 0.08\\
  
 \end{tabular}
 \caption{Model performance at prediction of conversion from MCI to AD on ADNI data.}
 \label{tab:adniConvPredRes}
\end{table}

\section{Discussion}
\label{sec:perfDis}

\subsection{Model Performance on DRC cohort}
\label{sec:perfDisDrc}

In the PCA cohort, we notice that the extended EBM methods show better results compared to the standard EBM method, whereas the extended DEM method has equal performance compared to the standard method. When comparing EBM vs DEM models, most EBM models perform as well as the DEM models in terms of \emph{hard staging consistency} but relatively worse in \emph{soft staging consistency}. There is also a drop in EBM staging consistency when moving from the \emph{hard} to the \emph{soft staging consistency}, which can be explained by the discrete nature of the EBM and by the simplistic biomarker trajectories, effectively modelled as step-functions, which can result in significant staging uncertainty. 


% staging - AD cohort
In the AD cohort we again find that the novel EBM methods show improvements over standard methods, while there is no significant difference between the novel and standard DEM methods. When comparing EBMs vs DEMs, we notice that the EBM models actually perform better in terms of \emph{hard-}, but worse in \emph{soft-staging consistency}. This could again be due to overly simplistic EBM trajectories that might not offer a good fit to the data.

% diagnosis prediction
In the diagnosis prediction tasks, most disease progression models have similar performance, with only the Standard EBM having a low performance in the PCA vs AD test. The SVM classifier has slightly worse results compared to the disease progression models for the Controls vs PCA task, but similar results for the other tasks.  

\subsection{Model Performance on ADNI cohort}
\label{sec:perfDisAdni}

In the ADNI cohort, we notice that the extended EBM and DEM methods have similar performance to the standard methods. There is again a drop in EBM performance on the soft consistency metric as compared to the hard consistency. The fact that there is no improvement in ADNI data between the novel methods and the standard methods suggests that the standard methods already offered a good fit on this dataset, and further that the ADNI dataset has different characteristics compared to the DRC dataset. We attribute this to the fact that the biomarkers present in the ADNI dataset were multimodal and included both early-stage mollecular markers as well as late-stage cognitive tests, which enabled even the standard models to robustly estimate the subjects' disease stages. 


The results on conversion prediction in ADNI show that all models have a broadly similar performance at this task. However, a few clear differences can be noticed in some models. The model with the best performance at 12-months and 24-months conversion prediction is the DEM with standard trajectory alignment, while at 36-month conversion the SVM and the novel EBM and DEM methods perform the best. The fact that different models have different performance at different durations-of-conversion suggests different models have better fits on certain time-frames of the disease time course.


\subsection{Staging-based Metrics}

The staging-based performance measures can pick up differences in the performance of the models in both DRC and ADNI datasets, in particular between different classes of models such as EBM vs DEM. This is most clear with the soft staging consistency, which penalises the EBM more than the DEM. This might be the case because the DEM constructs continuous non-parametric biomarker trajectories which might give a better fit than the simplistic step-wise trajectories of the EBM. 

The staging consistency metrics can also pick up differences across fitting procedures within the same model, such as in the case of EBMs. On the other hand, there are generally no statistically significant differences in the DEM between the standard versus optimised alignment, probably due to the fact that the standard alignment is already good enough for the two datasets that we tested them on.

% hard vs soft staging consistency
There are some differences between the results on the \emph{hard-} vs \emph{soft-staging consistency} metrics. In particular, the \emph{soft-staging consistency} penalised the EBM models more than the DEM models, probably due to increased staging uncertainty in the EBM models. In terms of time-lapse, there were no significant differences between the \emph{hard} and the \emph{soft} versions of this metric.

\subsection{Diagnosis Prediction Metrics}

We found that the performance metrics which are based on diagnosis or conversion prediction are less able to discriminate between different types of models or fitting procedures. Moreover, there was a lot of variability in the values of these performance metrics across folds and also in the model rankings across experiments, especially in the ADNI cohort, which made it hard to identify an overall best model. The variability of the diagnosis prediction metrics can be attributed to inaccuracies and biases in the diagnostic labels, and to the heterogeneity present in these diseases.

\section{Summary}
\label{sec:perfSum}

In this work we presented several extensions of the EBM and the DEM. We further devised performance metrics that measure the accuracy of the predicted subject stages and clinical diagnosis. We evaluated the new methodologies on data from two distinct diseases (PCA vs tAD), and on two independent datasets (ADNI and DRC). Our results show that in many situations the novel EBM and DEM fitting methods show improvements with respect to our performance metrics compared to the standard versions. 

\subsection{Limitations and Future Work}
\label{sec:perfSumLim}

% staging consistency is still a consistency metric, can easily get 100% if a model assigns the same stage to everyone
The performance metrics we used for evaluation have certain inherent limitations which might limit their use for some disease progression models. For example, the staging consistency metrics are prone to cheating using a specially crafted model that can get perfect consistency if it simply assigns the same stage to all subjects. However, the time-difference metric does not suffer from this problem, due to the fact that the model needs to predict precisely the time that passed between different visits to the clinic. 

% assume monotonically decreasing biomk: not the case for some diseases such as MS
The staging consistency metrics are based on the idea that the biomarkers evolve monotonically as the disease progresses. However, this is not the case with some neurological disorders such as multiple sclerosis (MS), where a patient can have an attack (relapse) followed by a period of steady recovery (remission). For such ``non-monotonic`` diseases, other performance measures based on biomarker predictions would be more suitable, such as the ones used in the TADPOLE Challenge \cite{marinescu2018tadpole}. 

% time difference is better but cannot be estimated for discreete models like EBM or Markov chains
One limitation of the time difference metrics is that they require the disease progression model to estimate the time from onset for every stage. This is not normally modelled in discrete models such as the EBM or other methods based on Markov chains (e.g. \cite{sukkar2012disease}). However, it might be possible to extend these discrete models in order to estimate time since onset for each of the states.

While we have tested these models only on the DRC and ADNI datasets, their performance might be different on other datasets with different types of neurodegenerative diseases and biomarker data. Future work should include validation on other datasets, including well-phenotyped datasets of autosomal-dominant Alzheimer's disease or genetic frontotemporal dementia. Moreover, the models should be tested also on other types of biomarkers, such as non-MRI imaging biomarkers, molecular measurements from cerebro-spinal fluid or cognitive tests.

Future work can also include performance evaluation of these models on simulated datasets, in presence of ground truth. This will enable the detection of more subtle differences in these methodologies, which might not be detectable in patient datasets due to inherent measurement noise and disease heterogeneity.

\section{Conclusion}
\label{sec:perfCon}

In this chapter I presented methodological extensions in the EBM and DEM, and evaluated their performance based on a set of performance measures, some of which I proposed. Future work will focus on evaluating other types of disease progression models presented in chapter \ref{chapter:bckDpm}, or on devising more sensitive performance metrics, for evaluation on both simulated data as well as patient datasets.

In the next chapter, I will present DIVE: a novel disease progression model that can estimate fine-grained spatial patterns of brain pathology, and estimate latent subject-specific time-shifts. Such a model overcomes a some limitations of the EBM and DEM models, which do not take spatial correlation into account and assume a pre-defined ROI atlas. DIVE can also help us better understand underlying disease mechanisms by studying the overlap between spatial patterns of pathology and brain connectomes.



\chapter[DIVE: A Spatiotemporal Progression Model of Brain Pathology]{DIVE: A Spatiotemporal Progression Model of Brain Pathology in Neurodegenerative Disorders}
\label{chapter:dive}

In this chapter I present DIVE, a novel spatiotemporal model of disease progression that I estimates fine-grained spatial patterns of atrophy in the brain. I did the entire work: model development, mathematical derivations (Supplementary section \ref{sec:diveAppendix}), image pre-processing and analysis and wrote the manuscripts. Arman Eshaghi showed me how to perform the image processing with Freesurfer. Daniel Alexander and Marco Lorenzi offered suggestions with modelling and experiment design. All collaborators gave me feedback on the manuscripts.

\section{Publications}
\begin{itemize}
 \item R. V. Marinescu, A. Eshaghi, M. Lorenzi, A. L. Young, N. P. Oxtoby, S. Garbarino, T. J. Shakespeare, S. J. Crutch and D. C. Alexander, A Vertex Clustering Model for Disease Progression: Application to Cortical Thickness Images, Information Processing in Medical Imaging, 2017
 \item R. V. Marinescu, A. Eshaghi, M. Lorenzi, A. L. Young, N. P. Oxtoby, S. Garbarino, S. J. Crutch, D. C. Alexander, DIVE: A spatiotemporal progression model of brain pathology in neurodegenerative disorders, NeuroImage, 2019.
\end{itemize}


\section{Introduction}
\label{sec:diveInt}

Current image-based disease progression models, such as those presented in section \ref{sec:bckSca}, estimate the evolution of the disease using a small set of biomarkers corresponding to pre-defined regions-of-interest (ROI). This ROI parcellation is usually coarse and doesn't allow one to find spatially dispersed patterns of atrophy. While spatiotemporal longitudinal models have already been demonstrated \cite{derado2010modeling, hyun2016stgp, lorenzi2015efficient}, these models regress against pre-defined sets of covariates such as age, time since baseline or clinical markers. This is problematic because, age-based alignment of subjects assumes all subjects have the same age of disease onset, while for time since baseline, its relationship with disease onset is unknown. Similarly, clinical markers are noisy, biased, suffer from floor/ceiling and training effects, are not sensitive in pre-symptomatic phases, and have low test-retest reliability \cite{johnson2012brain}.  Recently, some spatiotemporal models that estimate subject-specific time-shifts have been developed \cite{bilgel2016multivariate,koval2017statistical}. However, these models generally cannot recover dispersed and disconnected pathological patterns, because they assume voxel measurements correlate based on spatial distance, either through a distance function or distance from control points. However, spatially dispersed pathological patterns have been observed in AD and related dementias and are hypothesised to appear due to the interaction of pathology with brain networks \cite{seeley2009neurodegenerative}. Discovering such fine-grained patterns could allow one to understand underlying mechanisms of pathology propagation along these networks. However, a spatiotemporal disease progression model that allows recovery of dispersed and disconnected atrophy patterns present in AD, is not currently available. 

In this work, we present DIVE: Data-driven Inference of Vertexwise Evolution. DIVE is a novel disease progression model with single vertex resolution that makes only weak assumptions on spatial correlation. In contrast to approaches which model temporal trajectories for a small set of biomarker measures based on a priori defined ROIs, DIVE models temporal trajectories for each vertex on the cortical surface. DIVE combines unsupervised learning and disease progression modelling to identify clusters of vertices on the cortical surface that show a similar trajectory of brain pathology over a particular patient cohort. This formulation enables us to estimate a fine-grained spatial distribution of pathology and also provides a novel parcellation of the brain based on temporal change. 

We first test DIVE on synthetic data and show that the model can recover known biomarker trajectories and time-shifts. We then demonstrate the model on both MRI and PET data from two cohorts: the Alzheimer's Disease Neuroimaging Initiative (ADNI) and the Dementia Research Centre (DRC), UK. We use the model to reveal spatiotemporal patterns of pathology to a much finer resolution than previous models and demonstrate the ability to assign subjects to stages that predict progression. Finally, we validate DIVE in terms of how robust are the estimated pathology patterns and how well the disease progression scores correlate with cognitive tests. Code for DIVE is available online: \url{https://github.com/mrazvan22/dive}.


\section{Methods}
\label{sec:diveMet}

In this section we describe the mathematical formulation of DIVE (section \ref{sec:diveMod}), then we show how to fit the model using Expectation Maximisation (section \ref{sec:diveFit}) and we describe further implementation details of the algorithm (section \ref{sec:diveImplem}). Afterwards, we outline the synthetic data-generation process (section \ref{sec:diveSimulations}) for testing the model in the presence of ground truth, as well as the pipeline for pre-processing the ADNI and DRC datasets (section \ref{sec:diveDataAcquis}).

\begin{figure}
 \centering
 \includegraphics[width=\textwidth]{images/vwdpm_diagram.pdf}
%  \includepdf{../voxelwiseDPM/journal_paper/figures/vwdpm_diagram.png}
 \caption[Diagram of the proposed DIVE model.]{Diagram of the proposed DIVE model. DIVE assumes that biomarkers of pathology (e.g. cortical thinning) can be measured at many vertices (i.e. locations) on the cortical surface (A), where each vertex has a distinct trajectory of change during disease progression (B). In (B), each individual has measurements for vertex 1 at three visits. DIVE assigns to every cortical vertex one of a small set of temporal trajectories describing the change in some image-based measurement (e.g. cortical thickness, amyloid PET, DTI fractional anisotropy measures) from beginning to end of the disease progression. The estimation process simultaneously estimates the set of clusters, the trajectory defining each cluster, and the position of each subject along the trajectories, which are defined on a common timeline. The process iterates assignment of each vertex to clusters (red, green and blue in this diagram) (C), estimation of the trajectory in each cluster (D) and estimation of the disease progression score (location along trajectory) for  each subject (E), all within an Expectation-Maximisation framework, until convergence. In particular, (E) shows how the disease progression score, which is initially set to the individual's age, converges to the disease stage of the subject. Diagram made by me. }
 \label{fig:diveDiagram}
\end{figure}

\subsection{DIVE Model}
\label{sec:diveMod}

Figure \ref{fig:diveDiagram} illustrates the DIVE aims and implementation. DIVE input measures are vertexwise or voxelwise biomarker measures in the brain (Fig \ref{fig:diveDiagram}\textcolor{blue4}{A}), such as cortical thickness or amyloid load. A vertex is a location on the cortical surface at which a biomarker of pathology is quantifiable (e.g. cortical thickness). For each vertex on the cortical surface (or voxel in the 3D brain volume), we estimate a unique trajectory along the disease progression timeline (Fig \ref{fig:diveDiagram}\textcolor{blue4}{B}), while also estimating subject/visit-specific disease progression scores (i.e. disease stages). We do that by grouping vertices with similar biomarker trajectories into clusters (Fig \ref{fig:diveDiagram}\textcolor{blue4}{C}), and we estimate a representative trajectory for every cluster (Fig \ref{fig:diveDiagram}\textcolor{blue4}{D}). Each trajectory is a function of subject-/visit-specific disease progression scores (DPS) (Fig \ref{fig:diveDiagram}\textcolor{blue4}{E}). The DPS depends linearly on the time since baseline visit, but with subject-specific slope and intercept.

\subsection{Modelling Subject-specific Parameters}

The disease progression score $s_{ij}$ for subject $i$ at visit $j$ is a latent variable denoting the current disease stage of the subject at this visit. It  defined as a linear transformation of time since baseline measurement $t_{ij}$ (in years):

\begin{equation}
\label{eq:dps_vwdpm}
 s_{ij} = \alpha_i t_{ij} + \beta_i
\end{equation}
where $\alpha_i$ and $\beta_i$ represent the speed of progression and time shift (i.e. disease onset) of subject $i$ respectively. 

\subsection{Modelling Biomarker Trajectory for a Single Vertex}

DIVE assumes that the biomarker measure at each vertex on the cortical surface follows a sigmoidal trajectory $f(s ; \theta)$ over the disease progression score $s$ and with parameters $\theta$. We choose a parametric sigmoid function because it is a parsimonious parametric model that offers better fit compared to linear models, is monotonic, and can account for floor and ceiling effects \cite{caroli2010dynamics, sabuncu2011dynamics}. We also assume that vertices are grouped into $K$ clusters and we model a unique trajectory for each cluster $k \in {1, ... , K}$, which will be referred to as cluster trajectories. The sigmoidal function $f(s; \theta_k)$ for cluster $k$ is defined as: 

\begin{equation}
\label{eq:dps_vwdpm2}
 f(s;\theta_k) = \frac{a_k}{1+exp(-b_k(s-c_k))} + d_k
\end{equation}
where $s$ is the disease progression score from Eq. \ref{eq:dps_vwdpm} and $\theta_k = [a_k, b_k, c_k, d_k]$ are parameters controlling the shape of the trajectory -- $d_k$ and $d_k + a_k$ represent the lower and upper limits of the sigmoidal function, $c_k$ represents the inflection point and $a_k b_k/4$ represents the slope at the inflection point. 

For a given subject $i$ at visit $j$, the value $V_l^{ij}$ of its biomarker measurement at vertex $l$ is a random variable that has an associated discrete latent variable $Z_l \in [1, ... , K]$ denoting the cluster it was generated from. The value of $V_l^{ij}$ given that it was generated from cluster $Z_l$ can be modelled as:

% model for one voxel and label
\begin{equation}
\label{eq:dps_vwdpm3}
 p(V_l^{ij} | \alpha_i, \beta_i, \theta_{Z_l}, \sigma_{Z_l}, Z_l) = N(V_l^{ij} | f(\alpha_i t_{ij} + \beta_i | \theta_{Z_l}), \sigma_{Z_l})
\end{equation}
where $N(V_l^{ij} | f(\alpha_i t_{ij} + \beta_i | \theta_{Z_l}), \sigma_{Z_l})$ represents the probability density function (pdf) of the normal distribution that models the measurement noise along the sigmoidal trajectory of cluster $Z_l$, having variance $\sigma_{Z_l}$. Next, we assume the measurements from different subjects are independent, while the measurements from the same subject $i$ at different visits $j$ are linked using the disease progression score from equation \ref{eq:dps_vwdpm}. Moreover, we also assume a uniform prior on $Z_l$. This gives the following model:

% all subj are independent
\begin{equation}
\label{eq:dps_vwdpm4}
 p(V_l, Z_l | \alpha, \beta, \theta, \sigma) = \prod_{(i,j) \in I} N(V_l^{ij} | f(\alpha_i t_{ij} + \beta_i | \theta_{Z_l}), \sigma_{Z_l})
\end{equation}
where $I = {(i,j)}$ represents the set of all the subjects $i$ and their corresponding visits $j$. Furthermore, $V_l = [V_l^{ij} | (i,j) \in I]$ is the 1D array of all the values for vertex $l$ across every subject and corresponding visit. Vectors $\alpha = [\alpha_1, \dots, \alpha_S]$ and $\beta = [\beta_1, \dots, \beta_S]$, where $S$ is the number of subjects, denote the stacked parameters for the subject shifts. If a subject $i$ has multiple visits, these visits share the same parameters $\alpha_i$ and $\beta_i$. Vectors $\theta = [\theta_1, \dots, \theta_K]$ and $\sigma = [\sigma_1, \dots, \sigma_K]$, with $K$ being the number of clusters, represent the stacked parameters for the sigmoidal trajectories and measurement noise specific to each cluster.

Due to our main motivation of modelling population trajectories and in order to ensure robustness and identifiability, we did not add random effects to the trajectories of specific subjects.

\subsection{Modelling Biomarker Trajectories for all Vertices}

So far we have a model for only one vertex on the brain surface. We therefore extend the formulation to all the vertices by assuming all these vertex measurements are spatially independent, giving the complete data likelihood:

% all vertices are independent
\begin{equation}
\label{eq:dps_vwdpm5}
 p(V, Z | \alpha, \beta, \theta, \sigma) = \prod_l^L \prod_{(i,j) \in I} N(V_l^{ij} | f(\alpha_i t_{ij} + \beta_i | \theta_{Z_l}), \sigma_{Z_l})
\end{equation}
where $V = [V_1, \dots, V_L]$, $Z = [Z_1, \dots, Z_L]$, $L$ being the total number of vertices on the cortical surface. The formulation so far assumes spatial independence between measurements in different vertices, but in section \ref{sec:diveSpatialCorr} the model is extended to capture spatial correlations. The full joint distribution is given by:

\begin{equation} \label{eq21}
  p(V, Z, \alpha, \beta, \theta, \sigma) = p(V, Z | \alpha, \beta, \theta, \sigma) p(\alpha, \beta, \theta, \sigma)
\end{equation}
where $p(\alpha, \beta, \theta, \sigma)$ is an informative prior on the model parameters defined as follows: 
\begin{equation} \label{eq1}
\begin{split}
 p(V, Z, \alpha, \beta, \theta, \sigma) & = \prod_{i} p(\alpha_i) p(\beta_i)\\
 p(\alpha_i) & \sim \Gamma(\alpha_{shape}, \alpha_{rate})\\
 p(\beta_i) & \sim N(\beta_{mean}, \beta_{std})\\
\end{split}
\end{equation}
where $\alpha_{shape}$, $\alpha_{rate}$, $\beta_{mean}$, $\beta_{std}$ are a-priori defined hyperparameters. The informative priors on the subject-specific parameters help ensure model identifiability, as the model otherwise has two extra degrees of freedom. Such informative priors on $\alpha_i$ and $\beta_i$ also help deal with singularities in the objective functions of $\alpha_i$ and $\beta_i$ when the biomarker trajectories are flat.

We get the final model log likelihood for incomplete data by marginalising over the latent variables $Z$:
\begin{equation}
\label{eq:dps_vwdpm6}
 p(V|\alpha, \beta, \theta, \sigma) = \prod_{l=1}^L \sum_{k=1}^K p(Z_l = k) \prod_{(i,j) \in I} N(V_l^{ij} | f(\alpha_i t_{ij} + \beta_i | \theta_k), \sigma_k)
\end{equation}

Throughout the article, we will use the shorthand $z_{lk} = p(Z_l = k)$.

\subsection{Modelling Spatial Correlation}
\label{sec:diveSpatialCorr}

The version of the model presented so far assumes spatial independence between vertex measurements. However, the regional organisation of the cortex suggests we would expect spatial correlation\footnote{By correlation here we mean that these vertex measurements are not statistically independent} of the vertex measurements. More precisely, measures of cortical thickness or other modalities are often similar in neighbouring vertices on the cortical surface and likely belong to the same cluster. DIVE can be easily extended to include mild spatial constraints on the correlation of vertex measurements via a Markov Random Field (MRF), which encourages neighbouring vertices to have the same corresponding cluster. We hypothesise that incorporating such constraints should reduce the effects of noise and produce a more stable clustering. However, this does not model correlation between the actual vertex values, but only between the latent variables $Z_l$, i.e. the cluster membership of each vertex. The MRF thus has the advantage of not requiring the use of huge covariance matrices, which are otherwise needed if we want to model correlation of vertex values directly. Moreover, in contrast to previous methods that use correlation based on spatial distance \cite{bilgel2016multivariate,koval2017statistical}, we use neighbourhood correlations, which allow us to estimate fine-grained spatial patterns of pathology. With the MRF, the full-data likelihood function of the model now becomes:

\begin{equation}
 p(V, Z | \alpha, \beta, \theta, \sigma, \lambda) = \prod_l^L \left[ \prod_{(i,j) \in I} N(V_l^{ij} | f(\alpha_i t_{ij} + \beta_i | \theta_{Z_l}), \sigma_{Z_l}) \prod_{l_2 \in N_l} \Psi (Z_{l}, Z_{l_2}) \right]
\end{equation}

where $\Psi(Z_l, Z_{l2})$ is a clique term representing the likelihood of a neighbouring vertex $l_2$ to have similar label with vertex $l$. The formula for the clique term is:

\begin{equation}
 \Psi (Z_{l}=k, Z_{l_2}=k_2) = 
 \begin{cases}
  exp(g(\lambda)) & \text{if } k = k_2\\
  exp(-h(\lambda)) & \text{otherwise}
 \end{cases}
\end{equation} 

where $\lambda$ is a parameter controlling how much to penalise neighbouring vertices that belong to distinct clusters, and $g$ and $h$ are positive, monotonic functions over the $\lambda>0$ range. We choose $g(\lambda)=\lambda$ and $h(\lambda)=\lambda^2$, which results in a concave objective function for $\lambda$, ensuring that it can later be optimised (see M-step).

Therefore, the model parameters that need to be estimated are $M = [\alpha, \beta, \theta, \sigma, \lambda]$ where $\alpha$ and $\beta$ are the subject specific shifting parameters, $\theta$ and $\sigma$ are the cluster specific trajectory and noise parameters and $\lambda$ is the clique parameter denoting the penalisation of spatially non-smooth assignments of latent variables $Z$. 

\subsection{Fitting the Model using Generalised Expectation-Maximisation}
\label{sec:diveFit}

We choose to fit our model using Expectation-Maximisation (EM), because it offers a fast convergence given the large number of parameters that need to be estimated and the huge dimensionality of relevant datasets (e.g. 1973 subjects x 163,842 vertices in ADNI). In the next two sections we outline the E-step and M-step. While both of these steps have no closed-form solution, we will solve them using numerical optimisation, which only results in an increase in the objective function at each iteration. However, the EM algorithm is still guaranteed to converge, and this approach is called Generalised EM \cite{bishop2007pattern}.

Algorithm \ref{fig:algo_vwdpm} shows the model fitting procedure using the EM algorithm. The procedure first initialises (line 1) some parameters required to start the EM algorithm: the subject parameters $\alpha$ and $\beta$ and the latent parameters $z_{lk}$ which represent the assignment of vertices to clusters. In the M-step, the method updates the trajectories of each cluster (lines 4-6), the subjects-specific parameters (line 9) and the clique penalty term $\lambda$ (line 17). In the E-step, the method computes $z_{lk}$ (line 18) using previously defined functions that compute $z_{lk}$ given a fixed $\lambda$ (line 14).

\begin{figure}
\begin{algorithm}[H]
 Initialise $\alpha^{(0)}$, $\beta^{(0)}$, $z_{lk}^{(0)}$ \\
  \While{$\theta$, $\sigma$, $\alpha$, $\beta$ or $z_{lk}$ not converged}{
   \tcp*[l]{M-step 1: For each cluster, optimise its trajectory}
    \For{$k=1$ to $K$}{
      ${\theta_k^{(u)} = \argmin_{\theta_k} \sum_{l=1}^L z_{lk}^{(u-1)} \sum_{(i,j) \in I} (V_l^{ij} - f(\alpha_i^{(u-1)} t_{ij} + \beta_i^{(u-1)} | \theta_k))^2  - log\ p(\theta_k)}$\\
      $\theta_k^{(u)} = \mbox{make\_identifiable}(\theta_k^{(u)})$\\
      ${ \left(\sigma_k^{(u)}\right)^2 = \frac{1}{|I|} \sum_{l=1}^L z_{lk}^{(u-1)} \sum_{(i,j) \in I} (V_l^{ij} - f(\alpha_i^{(u-1)} t_{ij} + \beta_i^{(u-1)} | \theta_k^{(u)}))^2 - log\ p(\sigma_k)}$
    }
     \tcp*[l]{M-step 2: For each subject, optimise its time shift $\alpha_i$ and progression speed $\beta_i$}
    \For{$i=1$ to $S$}{
    \footnotesize
      ${\alpha_i^{(u)}, \beta_i^{(u)} = \argmin_{\alpha_i, \beta_i}  \left[ \sum_{l=1}^L \sum_{k=1}^K \frac{z_{lk}^{(u-1)}}{2 \left(\sigma_k^{(u)}\right)^2 } \sum_{j \in I_i} (V_l^{ij} - f(\alpha_i t_{ij} + \beta_i | \theta_k^{(u)}))^2\right] - log\ p(\alpha_i, \beta_i)}$
    \normalfont
    }
   \tcp*[l]{E-step 1: Define functions $\zeta_{lk}(\lambda)$ computing $z_{lk}$, the probability of vertex $l$ being assigned to cluster $k$, given fixed $\lambda$}
  \For{$l = 1$ to $L$}{
    \For{$k = 1$ to $K$}{
      \tcp*[l]{Pre-compute data fit terms $D_{lk}$}
      $D_{lk} = -\frac{1}{2}log\ (2 \pi \left(\sigma_k^{(u)}\right)^2) |I| - \frac{1}{2\left(\sigma_k^{(u)}\right)^2} \sum_{i,j \in I} (V_l^{ij} - f(\alpha_i^{(u)} t_{ij} + \beta_i^{(u)} | \theta_k^{(u)}))^2$\\
%       \tcp*[l]{Function $\zeta_{lk}(\lambda)$ computes $z_{lk}$ for a given $\lambda$}
      $ \zeta_{lk}(\lambda) \approx exp \left( D_{lk} +   \sum_{l_2 \in N_l} log\ \left[ exp(-\lambda^2) + z_{l_2k}^{(u-1)} (exp(\lambda) - exp(-\lambda^2)) \right] \right) $

    }
  }
%   $\Omega(\lambda) = $
  \tcp*[l]{M-step 3: optimise clique term $\lambda$ using above definitions in E-step 1}
%   $ \lambda^{(u)} = \argmax_{\lambda}\ \sum_{l=1}^L \sum_{k=1}^K \zeta_{lk}(\lambda) D_{lk} \  + \lambda \sum_{l = 1}^{L} \sum_{k}^K \sum_{l_2 \in N_l}  \zeta_{lk}(\lambda) \zeta_{l_2 k}(\lambda)\  + (-\lambda^2) \sum_{l = 1}^{L} \sum_{k}^K \sum_{l_2 \in N_l} \zeta_{lk}(\lambda) (1- \zeta_{l_2 k}(\lambda))   $\\
$ \lambda^{(u)} = \argmax_{\lambda}\ \sum_{l=1}^L \sum_{k=1}^K \zeta_{lk}(\lambda) \left[  D_{lk} \  + \lambda \sum_{l_2 \in N_l}  \zeta_{l_2 k}(\lambda)\  -\lambda^2 \sum_{l_2 \in N_l} (1- \zeta_{l_2 k}(\lambda))  \right]  $\\
  
  \tcp*[l]{E-step 2: Compute next $z_{lk}$ using the best $\lambda$}
  $z_{lk}^{(u)} = \zeta_{lk}(\lambda^{(u)})$
  }

  $\alpha_i^{(u)} = \frac{\alpha_i^{(u)}}{\sigma_N}, \beta_i^{(u)} = \frac{\beta_i^{(u)} - \mu_N}{\sigma_N}$   \tcp*[l]{Re-scale subject shifts}
%  }
\end{algorithm}
\caption[The DIVE parameter estimation algorithm.]{The DIVE parameter estimation algorithm. The algorithm, based on Expectation-Maximisation, iteratively optimises the assignment of vertices to clusters (E-step) and the parameters for the biomarker trajectories and subject time-shifts (M-step). }
\label{fig:algo_vwdpm}
\end{figure}
\subsubsection{E-step}

In the Expectation step, at iteration $u$ we seek an estimate of $p(Z | V, M^{(u-1)})$, given the current estimates of the parameters $M^{(u-1)} =[ \theta_k^{(u-1)}, \sigma_k^{(u-1)}, \alpha_i^{(u-1)}, \beta_i^{(u-1)}, \lambda_i^{(u-1)}]$. We perform this using Iterated Conditional Modes \cite{bishop2007pattern}, which performs coordinate-wise gradient ascent. This works by conditioning the clique terms Z on the values of Z from the previous iterations.  This approximation gives the following factorisable likelihood:

\begin{equation}
\label{eq:e_approx}
 p(Z | V, \Mu^{(u-1)}) \approx \prod_l^L \mathbb{E}_{Z_{N_l}^{(u-1)}|V_l, M} \left[ p(Z_l|V_l, \Mu, Z_{N_l}^{(u-1)}) \right]
\end{equation}

The factorised form allows for tractable computation and memory storage of $p(Z)$. Let $z_{lk}(u) = p(Z_l = k | V_l, M^{(u-1)},Z^{(u-1)})$. After simplifications we reach the following update rule:

\begin{equation}
\label{eq:e-step}
\begin{split}
 log\ z_{lk}^{(u)} \propto D_{lk}+ \left[ \sum_{l_2 \in N_l} log\ \left[ exp(-\lambda^2) + z_{l_2k}^{(u-1)} (exp(\lambda) - exp(-\lambda^2)) \right] \right]
\end{split}
\end{equation}
where the data-fit term $D_{lk}$ has the following form:

\begin{equation}
\label{eq:e-step_Dlk}
D_{lk} = -\frac{log\ (2 \pi \sigma_k^2) |I|}{2} - \sum_{i,j \in I}  \frac{1}{2\sigma_k^2}(V_l^{ij} - f(\alpha_i t_{ij} + \beta_i | \theta_k))^2 
\end{equation}

The full derivation is given in Supplementary section \ref{sec:diveEmDerivAppendix}. In order to enable optimisation over $\lambda$, a final modification of this step is performed, by considering $z_{lk}$ to be functions $\zeta_{lk}(\lambda)$ over $\lambda$. This results in the update equation from Alg. \ref{fig:algo_vwdpm}, line 18 which is based on pre-defined terms on lines 13-14. 


\subsubsection{M-step}

In the Maximisation step we try to estimate the model parameters $M = (\alpha, \beta, \theta, \sigma, \lambda)$ that maximise $E_{Z|V,M^{(u-1)}}[log\ p(V,Z|M)]$. We cannot simultaneously optimise all 5 sets of parameters, so we optimise them independently. In order to get the update rule for the trajectory parameters $\theta_k$ corresponding to cluster $k$ we need to maximise the expected log likelihood with respect to $\theta_k$. The key observation here is that if we assume fixed $\alpha$, $\beta$ and $Z$, then the trajectory parameters $\theta_k$ for every cluster $k$ are conditionally independent, i.e. $\theta_k \ci \theta_m | (Z, \alpha, \beta, \sigma)\ \forall\ (k, m)$, $k \neq m$. This allows us to maximise every $\theta_k$ independently using the following equation:

\begin{equation}
 \theta_k = \argmax_{\theta_k} \sum_{z_1,\dots, z_L}^K p(Z | V, \Mu^{(u-1)})\ log \left[ \prod_{l=1}^L \prod_{(i,j) \in I} N(V_l^{ij} | f(\alpha_i t_{ij} + \beta_i | \theta_{z_l}), \sigma_{z_l}) \right] + log\ p(\theta_k)
\end{equation}

A similar observation of conditional independence can also be observed for the latent variables $Z$. This allows us to decompose the joint distribution over $Z$, and after expanding the noise model we reach the optimisation problem from Alg. \ref{fig:algo_vwdpm}, line 4. See Supplementary section \ref{sec:diveEmDerivAppendix} for full derivation. This does not have a closed-form solution, so we use numerical optimisation for finding $\theta_k$ that maximises the equation from Alg. \ref{fig:algo_vwdpm}, line 4.

A similar equation, yet in closed form, is also obtained for $\sigma_k$ (Alg. \ref{fig:algo_vwdpm}, line 6).  After estimating $\theta$ and $\sigma$ for every cluster, we use the new values to estimate the subject specific parameters $\alpha$ and $\beta$. For every subject $i$, we maximise the expected log likelihood with respect to $\alpha_i$, $\beta_i$ independently, and after simplifications we obtain the update rule from Alg. \ref{fig:algo_vwdpm}, line 9, which is again solved using numerical optimisation. For the numerical optimisation of $\theta$ we used the Nelder-Mead method for its robustness, while for $\alpha$ and $\beta$ we used the second-order Broyden--Fletcher--Goldfarb--Shanno algorithm due to fast convergence. 

The large dimensionality of the dataset (around 163,428 vertices x 400 subjects x 4 timepoints each) makes model fitting extremely difficult from a computational perspective. Initial optimisation on a smaller subset of around 100 ADNI subjects took around 30h. However, we achieved a significant speed-up in the evaluation of objective functions by computing a $z_{lk}$-weighted average of vertex measurements within each cluster (see Appendix section \ref{sec:appDivFas}). This resulted in a final convergence time of around 4-6h depending on the size of the dataset, using an Intel Xeon E3-1271 @ 3.60GHz CPU. Regarding memory requirements, loading into memory around 1600 and fitting the model required around 12GB of RAM. However, we dropped it down by a factor of x4 by using small 16-bit floating representations for the vertexwise biomarkers.

For optimising $\lambda$, we again try to optimise in the M-step the expected full data likelihood under the $Z$ estimates from the previous iteration: 

\begin{equation}
\lambda^{(u)} = \argmax_{\lambda} E_{p(Z|V, M^{(u-1)}, \lambda, Z^{(u-1)})}[log\ p(V,Z|M^{(u-1)})]
\end{equation}

We simplify the above equation by expanding the likelihood model and approximating the joint over $Z$ with the product of the marginals $z_{lk}$ over all vertices $l$. This results in the update equation from Alg. \ref{fig:algo_vwdpm} line 17 -- see appendix for full derivation. In this final equation we also replaced $z_{lk}$ with a function $\zeta_{lk}(\lambda)$ over $\lambda$, which updates $z_{lk}$ based on the current value of $\lambda$ being evaluated. This is done to increase convergence, as latent variables $z_{lk}$ are highly coupled with the value of $\lambda$ being evaluated.

\subsection{Implementation Details}
\label{sec:diveImplem}

\subsubsection{Parameter Initialisation and Priors}

Before starting the fitting process, we need to initialise $\alpha$, $\beta$ and the clustering probabilities $z_{lk}$ (Alg. \ref{fig:algo_vwdpm}, line 1). We set $\alpha_i$ and $\beta_i$ to be 1 and 0 respectively for each subject, which sets the initial disease progression score to the time since baseline of the subject at the clinical visit. We initialise $z_{lk}$ using k-means clustering of the vectors $V_l$. We also initialise hyperparameters $\alpha_{shape}=16e4$, $\alpha_{rate}=16e4$, $\beta_{mean} = 0$ $\beta_{std} = 0.1$, which work well in practice as they result in realistic ranges for $\alpha_i$ and $\beta_i$ of around [0.3, 3] and [-15,15] respectively. The reason why we need to give such large numbers of 16e4 is because there are many vertex measurements (>100,000) that each drag the subject to an extremity if most values are above/below the population curve. This can be avoided in the future by adding subject-specific random effects to the population trajectory.

As already explained in \cite{jedynak2012computational}, the sigmoid parameters $\theta_k$ are not identifiable because $f(t;a_k,b_k, c_k, d_k) = f(t;-a_k,-b_k, c_k, a_k + d_k)$. We thus need to apply the following transformation on line 5 of Alg. \ref{fig:algo_vwdpm}: if $b_k^{(u)} < 0$ then $a_k^{(u)} = - a_k^{(u)}; b_k^{(u)} = - b_k^{(u)}; d_k^{(u)} = d_k^{(u)} - a_k^{(u)}$. This ensures model identifiability and is performed at every iteration. 

\subsubsection{Estimating the Optimal Number of Clusters}

The EM procedure needs to specify a-priori the number of clusters to fit on the data. We optimise the number of clusters $K$ using Akaike Information Criterion (AIC), which we found to better agree with ground truth in simulations than other information criteria such as the Bayesian Information Criteria (BIC). The number of parameters of the fitted model is 5$K$+2$S$+1, where $S$ is the number of subjects. Note that $z_{lk}$ are not included as parameters of the model because they are latent variables that are marginalised (see Eq. \ref{eq:dps_vwdpm6}). We repeat the fitting procedure for each $K$ from 2 to 100 clusters and select the $K$ that minimises the AIC.


\subsection{Simulation Experiments}
\label{sec:diveSimulations}

\subsubsection{Motivation}

Initial assessment of DIVE performance uses synthetic data, where we know the ground truth. The aim is to explore how accurately we can recover ground truth parameters as the problem becomes harder in three different scenarios:
\begin{itemize}
 \item Scenario 1: as the number of clusters increases, evaluate how well DIVE can estimate the correct number of clusters using AIC and BIC
 \item Scenario 2: as the trajectories become more similar, test how well we can recover the assignment of vertices to clusters and the DIVE parameters
 \item Scenario 3: same as Scenario 2, but for decreasing number of subjects
\end{itemize}

\subsubsection{Synthetic Data Generation}

We first designed a basic simulation, which the model should be able to fit well since the trajectories were designed to be well separated and enough subject data was generated along the disease time course. The data in the basic simulation was generated as follows: 

\begin{enumerate}
 \item Sampled baseline age $a_{i1}$ and shift parameters $\alpha_i$, $\beta_i$ for 300 subjects with 4 timepoints (each timepoint 1 year apart), with $a_{i1} \sim U(40,80)$, $\alpha_i \sim \Gamma(6.25, 6.25)$, $\beta_i \sim N(0, 10)$. Time since baseline has been obtained for every visit $j$ of subject $i$ as follows: $t_{ij} = a_{ij} - a_{i1}$.
 \item Generated three sigmoids with different (slope, centre) parameters: [(-0.1, -15), (-0.1, 2.5), (-0.1, 20)] (Fig. \ref{diveResSynthA}, red lines). Upper and lower limits have been set to 1 and 0 respectively.
 \item randomly assign every vertex $l \in \{1, \dots, L\}$, where $L = 1000$, to a cluster $a[l] \in \{1,2,3\}$
 \item Sampled a set of $L$ perturbed trajectories $\theta_l$ from each of the original trajectories, one for each vertex (Fig. \ref{diveResSynthA}, gray lines) using covariance matrix $C_{\theta} = diag([0, 2b_k/15,  11.6, 0])$.
 \item Sampled subject data for every vertex $l$ from its corresponding perturbed trajectory $\theta_l$ with noise standard deviation $\sigma_l = 1$
\end{enumerate}

From the basic simulation, we generated synthetic data for each of the three scenarios by varying one parameter at a time and kept the other parameters constant, having the same values as in the basic simulation. We varied the following parameters:
\begin{itemize}
 \item Scenario 1: number of clusters - 2, 3, 5, 10, 15, 20, 30 and 40. The cluster centres were spread evenly across a fixed total DPS range where the data was available. 
 \item Scenario 2: distance between trajectory centres (as proportion of total DPS range sampled) -- 0.33, 0.30, 0.23, 0.17, 0.10, 0.07, 0.03 and 0.02 
 \item Scenario 3: number of subjects - 300, 200, 100, 50, 35, 20, 10 and 5
\end{itemize}


\subsubsection{Model Fitting and Evaluation}

Since there was no spatial information in the data generation procedure, we used DIVE without the MRF extension. For Scenario 1, we estimated using AIC and BIC the optimal number of clusters. For Scenarios 2 and 3, after fitting the parameters of DIVE, we calculated the agreement between the final clustering probabilities $p(Z_l)$ and the true clustering assignments $a[l]$. This agreement, which we will call the clustering agreement, is defined as $\aleph = max_{\tau} (1/L) \sum_{l=1}^L p(Z_l = \tau(a[l]))$, where $\tau$ is any permutation of cluster labels. We also computed the error in the DPS estimation (sum of squared differences, SSD) and trajectory estimation (SSD between predicted trajectory and true trajectory at DPS points of every subject visit). 

\subsection{Data Acquisition and Pre-processing}
\label{sec:diveDataAcquis}

Data used in this work were obtained from the Alzheimer's Disease Neuroimaging Initiative (ADNI) database (\url{adni.loni.usc.edu}) and from the Dementia Research Centre, UK. For ADNI, we downloaded all T1 MR images that have undergone gradient warping, intensity correction, and scaling for gradient drift. We included subjects that had at least 3 scans, to ensure we get a robust estimate of the subject specific parameters. This resulted in 138 healthy controls, 235 subjects with mild cognitive impairment (MCI) and 81 subjects with Alzheimer's disease. 

We also downloaded all AV45 PET images from ADNI that were fully pre-processed, having the following tag: \emph{Co-reg, Avg, Std Img and Vox Siz, Uniform Resolution}. This meant that the images were co-registered, averaged across the 6 five-minute frames, standardised with respect to the orientation and voxel size and smoothed to produce a uniform resolution of 8mm full-width/half-max (FWHM). 

The DRC dataset consisted of T1 MRI scans from 31 healthy controls, 32 PCA and 23 typical AD subjects with at least 3 scans each and an average of 5.26 scans per subject. All PCA patients fulfilled both Tang-Wai \cite{tang2004clinical} and Mendez \cite{mendez2002posterior} criteria based on clinical review. The typical AD patients all met the criteria for probable Alzheimer's disease \cite{dubois2007research,dubois2010revising}. 

Given that the ADNI and DRC datasets contained subjects with different modalities or diseases, we ran DIVE independently on the following four cohorts (see Table \ref{tab:divePcaDemogr} for demographics): 
\begin{enumerate}
 \item ADNI MRI: controls, MCI and tAD subjects from ADNI (cortical thickness data) 
 \item DRC tAD: tAD subjects and controls from the DRC dataset (cortical thickness data)
 \item DRC PCA: PCA subjects and controls from the DRC dataset (cortical thickness data)
 \item ADNI PET: AV45 scans from ADNI containing subjects with following diagnoses: healthy controls, subjective memory complaints, early MCI, late MCI and Alzheimer's disease.
\end{enumerate}


\begin{table}
  \centering
  \begin{tabular}{c | c | c | C{3.5cm} | C{3.5cm} } 
  \textbf{Cohort} & \textbf{Diagnosis} & \textbf{Number of Subjects} & \textbf{Number of Scans} & \textbf{Age at baseline (years)}\\
  \hline
  ADNI & Controls & 138 & 4.3 & 76.3\\ 
  MRI & MCI & 235 & 4.6 & 74.8\\ 
  & AD & 81 & 3.5 & 75.8\\ 
  \hline
  DRC & Controls & 31 & 5.0 & 66.3\\ 
  tAD & AD & 24 & 5.4 & 71.2\\ 
  \hline
  DRC & Controls & 31 & 5.0 & 66.3\\ 
  PCA & PCA & 32 & 4.1 & 62.6\\ 
  \hline
  & Controls & 141 & 2.4 & 85.5\\ 
  ADNI & SMC & 27 & 2.0 & 86.1\\ 
  PET & EMCI & 149 & 2.4 & 85.6\\
  & LMCI & 104 & 2.4 & 86.0\\
  & AD & 12 & 2.0 & 87.3\\
  \end{tabular}
  \caption[Demographics of the four cohorts from ADNI and DRC]{Demographics of the four cohorts used in our analysis. ADNI MRI and the DRC cohorts were used for the cortical thickness analysis, while ADNI PET was used for the PET AV45 analysis. MCI -- mild cognitive impairment, SMC - subjective memory complaints, EMCI -- early MCI, LMCI -- late MCI.}
 \label{tab:divePcaDemogr}
\end{table}


\subsubsection{MRI Preprocessing}

On both datasets, in order to extract reliable cortical thickness measures, we ran the Freesurfer longitudinal pipeline \cite{reuter2012within}, which first registers the MR scans to an unbiased within-subject template space using inverse-consistent registration. The longitudinally registered images were then registered to the average Freesurfer template. No further smoothing was performed on these images (FWHM level of zero mm). From these template-registered volumetric images, cortical thickness measurements were computed at each vertex (i.e. point) on an average 2D cortical surface manifold. For each vertex we averaged the thickness levels from both hemispheres in order to later ease visualisation and to obtain a smaller representation of the input data. Each of the final images had a resolution of 163,842 vertices on the cortical surface. 

Finally, we standardised the data by computing Z-scores for each vertex with respect to the values of that vertex in the control population. This normalisation step ensures that the model will not be affected by different thicknesses of the cortex at various locations on the cortical surface. This step is specific for MRI cortical thickness data, and might not be necessary for other modalities (e.g. PET). 

\subsubsection{PET Preprocessing}

We computed amyloid standardised uptake value ratio (SUVR) levels using the PetSurfer pipeline \cite{greve2014cortical,greve2016different}, which is available with Freesurfer version 6. The PetSurfer pipeline first registers the PET image with the corresponding MRI scan, then applies Partial Volume Correction, and finally resamples the voxelwise SUVR values onto the cortical surface. While the final images also had a resolution of 163,842 vertices, the PET data we obtained from ADNI was inherently more smooth than the MRI cortical thickness data (8mm FWHM). We did not standardise the SUVR values like we did for cortical thickness, due to the fact that we did not observe different uptake based on anatomy within the control population.

\subsubsection{The MRF Neighbourhood Graph}

We estimated the MRF neighbourhood graph based on a Freesurfer triangular mesh for the fsaverage template. Each vertex was a triangle on the brain surface estimated with Freesurfer, and we connected the vertices if the corresponding triangles had a shared edge. For the MRF neighbourhood graph, we used a 3rd degree neighbourhood structure, meaning that two vertices were considered neighbours if the shortest path between them was not higher than 3. 



\section{Results}
\label{sec:diveResults}

\subsection{Results on Synthetic Data}
\label{sec:diveResultsSynth}

In the basic simulation, we obtained a clustering agreement $\aleph$ of 0.97, which suggests that almost all vertices were assigned to the correct cluster. Fig. \ref{diveResSynthA} shows the original trajectories and the recovered trajectories using our model, plotted against the disease progression score on the x-axis and the vertex value on the y-axis. In Fig. \ref{diveResSynthB} we plotted the recovered DPS of each subject along with the true DPS. The results for the three scenarios are shown in Figs. \ref{diveResSynthC}-\ref{diveResSynthE}. In Fig. \ref{diveResSynthC}, we show for Scenario 1 the estimated number of clusters against the true number of clusters using both AIC and BIC criteria. In Figs. \ref{diveResSynthD}-\ref{diveResSynthE} we show the distributions for $\aleph$ in Scenarios 2 and 3 as the problem becomes harder in each successive step. 


The results show that, in a simple experiment where the trajectories are well separated, DIVE can very accurately estimate which clusters generated each vertex. Moreover, the recovered trajectories and DPS scores are close to the true values. The results of Scenario 1 also suggest that both AIC and BIC are effective at estimating the correct number of known clusters, with AIC having slightly better performance than BIC for larger numbers of clusters. On the other hand, the results of the stress test scenarios 2 and 3 show that performance measure $\aleph$ drops when the trajectories become very similar with each other or when the number of subjects decreases. This happens because small differences in trajectories are hard to detect in the presence of measurement noise, while a small number of subjects doesn't provide enough data to accurately estimate the parameters. Similar decreases in performance for scenarios 2 and 3 are observed also for other measures, such as the error in recovered trajectories or DPS scores (Supplementary Fig \ref{diveTrajError}).


\begin{figure}
\centering
\begin{subfigure}[b]{0.7\textwidth}
\includegraphics[width=1\textwidth,trim=30 0 60 0,clip]{images/initk-meansCl3Pr1Ra1_VWDPMMean/synThetaRes_initk-meansCl3Pr1Ra1_VWDPMMean.png}
\caption{}
\label{diveResSynthA}
\end{subfigure}
\begin{subfigure}[b]{0.25\textwidth}
\includegraphics[width=1\textwidth,trim=0 0 0 0,clip]{images/initk-meansCl3Pr1Ra1_VWDPMMean/synShiftsRes_initk-meansCl3Pr1Ra1_VWDPMMean.png}
\vspace{0.4em}
\caption{}
\label{diveResSynthB}
\end{subfigure}
\begin{subfigure}[b]{0.32\textwidth}
\includegraphics[width=1\textwidth]{images/nrClust.png}
\caption{}
\label{diveResSynthC}
\end{subfigure}
\begin{subfigure}[b]{0.32\textwidth}
\includegraphics[width=1\textwidth]{images/correctVertices_trajCent.png}
\caption{}
\label{diveResSynthD}
\end{subfigure}
\begin{subfigure}[b]{0.32\textwidth}
\includegraphics[width=1\textwidth]{images/correctVertices_nrSubj.png}
\caption{}
\label{diveResSynthE}
\end{subfigure}
\caption[DIVE Simulation Results]{(a-b) Results for the basic simulation, where trajectories are relatively well separated. (a) Reconstructed temporal trajectories (blue) plotted against the true trajectories (red). The x-axis shows the disease progression score (DPS), while the y-axis shows the biomarker values of the vertices. (b) Estimated subject-specific DPS scores compared to the true scores. (C-E) Simulation results for the three scenarios: (c) increasing number of clusters, (d) trajectories becoming similar and (e) decreasing number of subjects. On the x-axis we show the variable that was changing within the scenario (e.g. number of clusters), while on the y-axis we show the agreement measure $\aleph$, representing the percentage of vertices that were assigned to the correct cluster.}
\label{diveResSynth}
\end{figure}


\subsection{Results with ADNI and DRC Datasets}
\label{sec:diveResAdniDrc}

\subsubsection{Initial Hypotheses}

Using ADNI and DRC datasets, we aim to recover the spatial distribution of cortical atrophy and amyloid pathology, as well as the rate and timing of these pathological processes. In particular, we hypothesise that these spatial patterns of pathology and their evolution will be: 
\begin{itemize}
 \item similar on two independent typical AD datasets: ADNI and DRC
 \item different on distinct diseases: tAD vs PCA 
 \item different in distinct modalities: cortical thickness from MRI vs amyloid load from AV45 PET.
\end{itemize}

\subsubsection{Results}

The optimal number of clusters, as estimated with AIC, was three for the ADNI MRI dataset, three for the DRC tAD dataset, five for the DRC PCA dataset and eighteen for the ADNI PET dataset. Fig. \ref{diveClustAdniMri} (left) shows the results from the ADNI MRI dataset, where in the left image we coloured the vertices on the cortical surface according to the cluster they most likely belong to. We assigned a colour for each cluster (both the brain figures on the left and the trajectory figures on the right) according to the extent of pathology of its corresponding trajectory at a DPS score of 1. The cluster colours range from red (severe pathology) to blue (moderate pathology). In Fig. \ref{diveClustAdniMri} (right), we show the resulting cluster trajectories with samples from the posterior distribution of each $\theta_k$. Similar results are shown for the other three datasets: the DRC tAD dataset (Fig. \ref{diveClustDrcAd}), DRC PCA dataset (Fig. \ref{diveClustDrcPca}) and the ADNI PET dataset (Fig. \ref{diveClustAdniPet}).


We notice that in tAD subjects using the ADNI datasets (Fig. \ref{diveClustAdniMri}), there is more severe cortical thinning mainly in the inferior temporal lobe (red cluster), with disperse atrophy also in parietal and frontal regions (green cluster), with relative sparing of the inferior frontal and occipital lobes. In tAD subjects from the DRC dataset (Fig. \ref{diveClustDrcAd}), we see a relatively similar pattern, however with more pronounced atrophy in the supramarginal cortex (red cluster) compared to ADNI. This could be due to the younger ages of controls and tAD subjects in the DRC dataset as compared to ADNI. The spatial distribution of cortical thinning found with DIVE resembles results from previous longitudinal studies such as \cite{dickerson2008cortical,singh2006spatial}. However, in contrast to these approaches, our model gives insight into the timing and rate of atrophy and is also able to stage subjects across the disease time course. We also find that the cluster trajectories in the DRC tAD dataset have similar dynamics to the ADNI MRI dataset, although they show a clearer separation between each other.

In the PCA subjects (Fig. \ref{diveClustDrcPca}), we find that atrophy is mainly focused on the posterior part of the brain, with limited spread in the motor cortex, anterior temporal and frontal areas. This posterior-focused pattern of atrophy is different from the one found in the tAD datasets, and agrees with previous findings in the literature \cite{crutch2012posterior,lehmann2011cortical}.  However, as opposed to the results from \cite{lehmann2011cortical} which showed posterior regions uniformly affected, we notice that there are two clusters within the posterior region with different pathology dynamics, with the superior parietal and supramarginal areas affected more that the remaining posterior regions. This might be attributable to DIVE's ability to model subjects' disease onset and progression speed, along with non-linear cortical thinning dynamics, other differences due to the different subjects analysed, and the merging of left and right hemispheres could also give such differences.

In ADNI PET (Fig. \ref{diveClustAdniPet}) we see that the regions with the highest amyloid uptake are more spatially continuous, comprising the precuneus and anterior frontal areas. On the other hand, the anterior-superior temporal gyrus shows the least uptake of amyloid. This result closely matches the result by \cite{bilgel2016multivariate}, which used a completely different dataset and modelling technique. These results using AV45 PET are also noticeably different from results using cortical thickness (e.g. Fig. \ref{diveClustAdniMri}), which have more high-frequency patterns and only give 3-5 optimal clusters instead of 20. The “layers of clusters” starting from the precuneus and frontal lobes, which range from severe to less severe atrophy, suggest a continuum of variation in vertex trajectories in the case of the PET dataset (Fig \ref{diveClustAdniPet}, right). These trajectories all start with a low amyloid SUVR, between 0 and 0.25, but in late stages the trajectories for some clusters such as cluster 0 can reach an SUVR of 1.5. The reason for seeing this continuum might be because the PET images have a much lower resolution than MR images and were smoothed by ADNI during the pre-processing steps. 

\newcommand{\scalingFactor}{1.2}
\newcommand{\scalingFactorSubfigBrain}{0.35}

\newcommand{\gradLimLeft}{-1.6}
\newcommand{\gradLimRight}{1.6}

% \newcommand{\scalingFactorLeftFig}{1.2}
\newcommand{\scalingFactorBrains}{0.75}
\newcommand{\scalingFactorTraj}{1.05}

\newcommand{\typeOfBrainColoring}{atrophyExtent}

\definecolor{barGreen}{rgb}{0.4,1,0.4}

% FWHM0 avg thickness map MCI & AD
\begin{figure}
  \centering
  \vspace{-1em}

  % do the legend colorbar
  \begin{subfigure}[b]{0.45\textwidth}
   \centering
  \begin{tikzpicture}[scale=\scalingFactor]
    \shade[left color=red,right color=yellow] (\gradLimLeft,2.5) rectangle (-0.8,2.75);
    \shade[left color=yellow,right color=barGreen] (-0.8,2.5) rectangle (0,2.75);
    \shade[left color=barGreen,right color=cyan] (0,2.5) rectangle (0.8,2.75);	
    \shade[left color=cyan,right color=blue] (0.8,2.5) rectangle (\gradLimRight,2.75);   
%     \node[inner sep=0] (corr_text) at (\gradLimLeft,2.25) {cluster 0};
%     \node[inner sep=0] (corr_text) at (0,2.25) {cluster 1};
%     \node[inner sep=0] (corr_text) at (\gradLimRight,2.25) {cluster 2};
    \node[inner sep=0] (corr_text) at (\gradLimLeft,3) {severe pathology};
    \node[inner sep=0] (corr_text) at (\gradLimRight,3) {moderate pathology};
  \end{tikzpicture}
%     \caption{}
%       \label{fig:adniClust}
  \vspace{1em}
  \end{subfigure}
  
  %%%%%%%%%%%%%%%%%%% BRAINS %%%%%%%%%%%%%%%%%%%%
  

  \begin{subfigure}[b]{\textwidth}
   \centering
  \includegraphics[width=\scalingFactorSubfigBrain \textwidth,trim=0 0 0 20,clip]{images/atrophyExtent24_adniThInitk-meansCl3Pr1Ra1_VDPM_MRF.png} \includegraphics[width=\scalingFactorSubfigBrain \textwidth,trim=0 10 0 30,clip]{images/trajSamplesOneFig_adniThInitk-meansCl3Pr1Ra1_VDPM_MRF.png}
    \caption{ADNI MRI}
    \label{diveClustAdniMri}
  \end{subfigure}

  \begin{subfigure}[b]{\textwidth}
   \centering
  \includegraphics[width=\scalingFactorSubfigBrain \textwidth,trim=0 0 0 20,clip]{images/atrophyExtent24_drcThInitk-meansCl3Pr1Ra1_VDPM_MRFAD.png} \includegraphics[width=\scalingFactorSubfigBrain \textwidth,trim=0 10 0 30,clip]{images/trajSamplesOneFig_drcThInitk-meansCl3Pr1Ra1_VDPM_MRFAD.png}
    \caption{DRC tAD}
    \label{diveClustDrcAd}
  \end{subfigure}
  
  \begin{subfigure}[b]{\textwidth}
   \centering
  \includegraphics[width=\scalingFactorSubfigBrain \textwidth,trim=0 0 0 20,clip]{images/atrophyExtent24_drcThInitk-meansCl5Pr1Ra1_VDPM_MRFPCA.png} \includegraphics[width=\scalingFactorSubfigBrain \textwidth,trim=0 10 0 30,clip]{images/trajSamplesOneFig_drcThInitk-meansCl5Pr1Ra1_VDPM_MRFPCA.png}
    \caption{DRC PCA}
    \label{diveClustDrcPca}
  \end{subfigure}
  
  \begin{subfigure}[b]{\textwidth}
   \centering
  \includegraphics[width=\scalingFactorSubfigBrain \textwidth,trim=0 0 0 20,clip]{images/atrophyExtent24_adniPetInitk-meansCl18Pr1Ra1_VDPM_MRF.png} \includegraphics[width=\scalingFactorSubfigBrain \textwidth,trim=0 10 0 30,clip]{images/trajSamplesOneFig_adniPetInitk-meansCl18Pr1Ra1_VDPM_MRF.png}
    \caption{ADNI PET}
    \label{diveClustAdniPet}
  \end{subfigure}
  
  %%%%%%%%%%%%%%%%%%%%%% trajectories %%%%%%%%%%%%%%%%%%%%%%%

  \caption[DIVE Results on ADNI and DRC cohorts]{(left column) DIVE estimated clusters (left column) and corresponding disease progression trajectories (right column) on four datasets: (a) ADNI MRI (b) DRC tAD (c) DRC PCA and (d) ADNI PET. We coloured each cluster according to the extent of pathology (cortical thickness or amyloid uptake) at DPS=1.}
  \label{diveClustTrajAll}
\end{figure}


\subsection{Model Evaluation}
\label{sec:diveEval}

\subsubsection{Motivation}
\label{sec:diveEvalMotiv}

We further tested the robustness and validity of the model as follows: 
\begin{itemize}
 \item Robustness in parameter estimation: test whether similar spatial clustering is estimated for different subsets of the data
 \item Clinical validity of DPS scores: test whether the subject disease progression scores, based purely on MRI or PET data, correlate with cognitive tests such as Clinical Dementia Rating Scale - Sum of Boxes (CDRSOB), Alzheimer's Disease Assessment Scale - Cognitive (ADAS-COG), Mini-Mental State Examination (MMSE) and Rey Auditory and Verbal Learning Test (RAVLT).
 \item Comparison with other models: to evaluate the benefit of estimating fine-grained patterns of pathology in DIVE, as well as latent time shifting of subjects, we compared the performance of DIVE with a region-of-interest based method \cite{jedynak2012computational} and a no-staging method that doesn't estimate subject time shifts. See Supplementary Section \ref{sec:diveCompAppendix} for precise specifications.
\end{itemize}


\subsubsection{Evaluation Procedure}

For all scenarios, we ran 10-fold cross-validation (CV) on the ADNI MRI dataset. At each fold we fit the model using 3 clusters, since this was the optimal number of clusters found previously on the entire dataset. The trained model was then used to estimate the DPS of the test subjects. 

For the performance comparison of DIVE with other models, we compute two performance metrics: (1) between-subject correlation of the models' estimated DPS values with cognitive tests; we estimated a unique DPS for every subject and every visit, which we then matched with the corresponding cognitive tests at that subject's visit and (2) prediction root mean squared error (RMSE) between the predicted vertex-wise values and actual measurements, averaged over all subjects and all locations on the brain; to evaluate these predictions, for every subject we use the first n-1 scans for training and the last scan for testing the prediction.


\subsubsection{Evaluation Results}


\newcommand{\outFoldADNICVbrains}{images/vwdpm/crossvalid/adniThavgFWHM0Initk-meansCl3Pr0Ra1_VWDPMMean}
\newcommand{\outFoldADNIPetCVbrains}{figures/validAdniPET/brainAtrophyExtent}

\newcommand{\adniThickCVExpName}{adniThInitk-meansCl3Pr1Ra1_VDPM_MRF}
\newcommand{\adniThickCVFolder}{images/\adniThickCVExpName}
\newcommand{\trimModelValidTop}{0}

\begin{figure}
 \centering
\foreach \f in {0,1,2,3,4,5,6,7,8,9}
{
\begin{subfigure}[b]{0.185\textwidth}
\centering
  \FPeval{\faddOne}{clip(\f+1)}
  f = \faddOne \\
  \includegraphics[width=\textwidth, trim=0 0 0 \trimModelValidTop]{\adniThickCVFolder/f\f/atrophyExtent_cogCorr_\adniThickCVExpName_f\f.png}
\end{subfigure}
}
\vspace{1em}

 \begin{python}
for f in range(10): 
  if f == 5:
    print '\n'
  if f == 0 or f ==5:
    #print r''
    print r'\begin{subfigure}[b]{0.223\textwidth}'
    print r'\centering'
    print 'f=%d' % (f+1)
    print r'\includegraphics[width=\linewidth, trim=22 0 35 20,clip]{\adniThickCVFolder/'+ 'f%d/trajSamplesOneFig_cogCorr_\\adniThickCVExpName_f%d.png}' % (f, f)
    
  else:
    print r'\begin{subfigure}[b]{0.185\textwidth}'
    print r'\centering'
    print 'f=%d' % (f+1)
    print r'\includegraphics[width=\linewidth, trim=75 0 35 20,clip]{\adniThickCVFolder/'+ 'f%d/trajSamplesOneFig_cogCorr_\\adniThickCVExpName_f%d.png}' % (f, f)
  
  print r'\end{subfigure}%    <-- % added here'
  print r'\hfill'
\end{python}
\caption[DIVE estimated clusters and trajectories over the 10 cross-validation folds]{(top) Clusters estimated from 10-fold cross-validation training sets on the ADNI MRI dataset. (bottom) Estimated trajectories for each fold. }
\label{fig:diveClustTrajCV}
\end{figure}

Fig. \ref{fig:diveClustTrajCV} shows the brain clusters and corresponding trajectories, estimated for all the cross-validation folds after fitting the model on the training data. The clusters have been coloured using a similar colour scheme as in Fig. \ref{diveClustTrajAll}. In Fig \ref{fig:diveCogCorr} we show scatter plots of the DPS scores with clinical measures such as CDRSOB, ADAS-COG, MMSE and RAVLT.

\newcommand{\figFont}{\normalfont}
\newcommand{\pValFont}{\footnotesize}

\newcommand{\cogCorrScatterFold}{images}

\begin{figure}[h]
  \begin{subfigure}{0.245\textwidth}
    \centering
    \hspace{1.5em}\figFont{CDRSOB}\\ 
    \hspace{1.5em}\pValFont{($\rho = 0.37$, $p < 1e-65$)}
    \includegraphics[width=1.1\textwidth]{\cogCorrScatterFold/stagingCogTestsScatterPlot_adniThInitk-meansCl3Pr1Ra1_VDPM_MRF_CDRSOB.png}
  \end{subfigure}
  \begin{subfigure}{0.245\textwidth}
    \centering
    \hspace{1.5em}\figFont{ADAS-COG}\\ 
    \hspace{1.5em}\pValFont{($\rho = 0.37$, $p < 1e-64$)}
    \includegraphics[width=1.1\textwidth]{\cogCorrScatterFold/stagingCogTestsScatterPlot_adniThInitk-meansCl3Pr1Ra1_VDPM_MRF_ADAS13.png}
  \end{subfigure}
    \begin{subfigure}{0.245\textwidth}
    \centering
    \hspace{1.4em}\figFont{MMSE}\\ 
    \hspace{1.4em}\pValFont{($\rho = -0.36$, $p < 1e-63$)}
    \includegraphics[width=1.1\textwidth]{\cogCorrScatterFold/stagingCogTestsScatterPlot_adniThInitk-meansCl3Pr1Ra1_VDPM_MRF_MMSE.png}
  \end{subfigure}
    \begin{subfigure}{0.245\textwidth}
    \centering
    \hspace{1.4em}\figFont{RAVLT}\\ 
    \hspace{1.4em}\pValFont{($\rho = -0.32$, $p < 1e-49$)}
    \includegraphics[width=1.1\textwidth]{\cogCorrScatterFold/stagingCogTestsScatterPlot_adniThInitk-meansCl3Pr1Ra1_VDPM_MRF_RAVLT.png}
  \end{subfigure}
  \caption[Scatter plot of DIVE-derived DPS scores vs cognitive tests]{Scatter plots of the DPS scores estimated from the ADNI MRI dataset, plotted against four cognitive tests: CDRSOB, ADAS-COG, MMSE and RAVLT. For each cognitive test we also report the Pearson correlation coefficient and p-value. The disease progression scores, computed only based on MRI cortical thickness data, correlate with these cognitive measures, suggesting that the DPS scores are clinically meaningful. }
  \label{fig:diveCogCorr}
\end{figure}



\begin{table}[H]
\centering
\begin{footnotesize}
 \begin{tabular}{c | c c c c | c}
  Model & CDRSOB ($\rho$) & ADAS13 ($\rho$) & MMSE ($\rho$) & RAVLT ($\rho$) & Prediction (RMSE)\\
  \hline 
DIVE & 0.37 $\pm$ 0.09 & 0.37 $\pm$ 0.10 & 0.36 $\pm$ 0.11 & 0.32 $\pm$ 0.12 & 1.021 $\pm$ 0.008 \\
ROI-based model & 0.36 $\pm$ 0.10 & 0.35 $\pm$ 0.11 & 0.34 $\pm$ 0.13 & 0.30 $\pm$ 0.13 & 1.019 $\pm$ 0.010 \\
No-staging model & *0.09 $\pm$ 0.06 & *0.03 $\pm$ 0.09 & *0.05 $\pm$ 0.06 & *0.02 $\pm$ 0.06 & *1.062 $\pm$ 0.024 \\

 \end{tabular}
 \end{footnotesize}
 \caption[Performance evaluation of DIVE and two simplified models on the ADNI MRI dataset]{Performance evaluation of DIVE and two simplified models on the ADNI MRI dataset using 10-fold cross-validation. In the middle four columns, we show between-subject correlations between the DPS scores and several cognitive tests: CDRSOB, ADAS-Cog13, MMSE and RAVLT. The last column shows the prediction error (RMSE) of cortical thickness values from follow-up scans. (*) Statistically significant differences between the model and DIVE, Bonferroni corrected for multiple comparisons.}
 \label{tab:divePerfEval}
\end{table}

The results in Fig. \ref{fig:diveClustTrajCV} demonstrate that DIVE is robust in cross-validation, as the estimated clusters and trajectory parameters are all similar across folds. The average Dice score overlap across the 10-folds range were 0.77, 0.76 and 0.90 for clusters 0, 1 and 2 respectively. The DIVE-derived DPS scores, which were estimated purely based on MRI data, are also clinically relevant as they correlate with cognitive tests (Fig. \ref{fig:diveCogCorr}). 

The performance of DIVE in terms of subject staging and biomarker prediction also compares favourably with simpler no-staging and ROI-based models (Table \ref{tab:divePerfEval}). Results show that DIVE has comparable performance to the ROI-based model, both in terms of subject staging and cortical thickness prediction. The fact that DIVE has similar performance to a simpler model which has less parameters is evidence that the estimated patterns are meaningful. Moreover, DIVE offers qualitative insight into the fine-grained spatial patterns of pathology and their temporal progression. Furthermore, the No-staging model performs significantly worse than DIVE, both in terms of subject staging and for biomarker prediction. This suggests that, when modelling progression of AD, it is important to account for the fact that patients are at different stages along the disease time-course.


\section{Discussion}
\label{sec:diveDis}

\subsection{Summary and Key Findings}

We presented DIVE, a spatiotemporal model of disease progression that clusters vertex- or voxel-wise measures of pathology in the brain based on similar temporal dynamics. The model highlights, for the first time, groups of cortical vertices that exhibit a similar temporal trajectory over the population. The model also estimates the temporal shift and progression speed for every subject. We applied the model on cortical thickness vertex-wise data from three MRI datasets (ADNI, DRC tAD and DRC PCA), as well as an amyloid PET dataset (ADNI). Our model found qualitatively similar patterns of cortical thinning in tAD subjects using the two independent datasets (ADNI and DRC). Moreover, it also found different patterns of pathology dynamics on two distinct diseases (tAD and PCA) and on different types of data (PET and MRI-derived cortical thickness). Finally, DIVE also provides a new way to parcellate the brain that is specific to the temporal trajectory of a particular disease, and enables staging of individuals at risk of disease, which can potentially help stratification in clinical trials.

The characteristics of the subjects' data used for training can affect the DIVE output. For instance, in cortical thinning analyses we standardised the data with respect to controls, which might have already shown cortical thinning due to early pathology. This can be mitigated through enrichment of the control population to amyloid-negative individuals. DIVE also relies on subjects spanning the entire disease progression, so inclusion of subjects in middle stages is recommended for a robust estimation of trajectories and spatial patterns. To reliably estimate the subject-specific time shift and progression speed, multiple follow-up scans are required. We mitigated this by using only subjects with at least three scans, and further placing informative priors on these parameters. 

The DIVE-estimated spatial patterns are patchier in MRI compared to PET scans, which had lower resolution and were smoothed a-priori. However, we believe MRI images should not instead smoothed a-priori, as the spatial correlation mechanism within DIVE enables it to automatically remove high-frequency patterns from MRI that are not meaningful.  Moreover, such a-priori smoothing could potentially loose dispersed patterns of pathology that arise due to underlying disruption of brain networks.



\subsection{Limitations and future work}

DIVE has some limitations that can be addressed. First, we assumed that cluster trajectories follow sigmoidal shapes, which is not the case for many types of biomarkers in ADNI which do not plateau in later stages. The assumption of sigmoidal trajectories can be avoided using non-parametric curves such as Gaussian Processes \cite{lorenzi2017disease}, which would be straightforward to incorporate into the DIVE framework. To get a reliable estimate of the subject-specific parameters, we only tested DIVE on balanced datasets, where subjects had at least three scans. However, DIVE can also be applied to less balanced datasets, by setting stronger priors on these parameters or even fixing the progression speed for every subject to 1. Another limitation of the model is that it assumes all subjects follow the same disease progression pattern, which might not be the case in heterogeneous datasets such as ADNI or DRC. This can be a concern, as there might be a pattern of pathology that occurs in a small set of subjects. However, DIVE can be extended to account for heterogeneity in the datasets by modelling subject-specific trajectories using random effects, or different progression dynamics for distinct subgroups, using unsupervised learning methods like the SuStaIn model by \cite{young2018uncovering}. While SuStaIn, just like DIVE, estimates clusters and trajectories within the dataset, the clusters in SuStaIn are made of subjects with similar disease progression, while the clusters in DIVE are made of vertices with similar progression. Future work could combine clustering along both subjects and vertices simultaneously to estimate disease subtypes with distinct spatiotemporal dynamics at the vertexwise level.

There are several potential future applications of DIVE. One of the advantages of DIVE is that it can be used to study the link between disconnected patterns of brain pathology and connectomes extracted from diffusion tractography or functional MRI (fMRI). Such an analysis would enable further understanding of the exact underlying mechanisms by which the brain is affected by the disease. Our model, which can estimate fine-grained spatial patterns of pathology, is more suitable than standard ROI-based methods for studying the link between pathology and these structural or functional connectomes, because white matter or functional connections have a fine-grained and spatially-varying distribution of endpoints on the cortex.

Apart from studying the link with brain connectomes, there are other potential applications for DIVE. While we only applied it to vertexwise data, the model can also be applied to study voxelwise data. Moreover, DIVE can be applied to other modalities or types of data, including FDG PET, tau PET, DTI or Jacobian compression maps from MRI. Moreover, the model can also be extended to cluster points on the brain surface according to a more complex disease signature, that can be made of two or more biomarkers. For example, using our cortical thickness and amyloid PET datasets from ADNI, we could have clustered points on the brain based on both modalities simultaneously. Such complex disease signatures can offer important insights into the relationships between different modalities and underlying disease mechanisms.

DIVE is a spatiotemporal model that can be used for accurately predicting and staging patients across the progression timeline of neurodegenerative diseases. The spatial patterns of pathology can also be used to test mechanistic hypotheses which consider AD as a network vulnerability disorder. All these avenues can help towards disease understanding, patient prognosis, as well as clinical-trials for assessing efficacy of a putative treatment for slowing down cognitive decline.

\section{Conclusion}
\label{sec:diveConclusion}

In this chapter I developed DIVE, a spatiotemporal model of disease progression that estimates fine-grained spatial patterns of brain pathology, while simultaneously placing subjects optimally on a disease time axis. I applied it to two typical AD MRI datasets (ADNI and DRC), one dataset of PCA patients, and one typical AD PET dataset. I also tested the robustness of the method in simulations, under cross-validation, and I've also compared its performance to simpler feature-based models.

In the next chapter, I will present another model, DKT, that can transfer information across different types of dementias in order to estimate the progression of rare dementias from limited, unimodal datasets. 


\chapter{Disease Knowledge Transfer across Neurodegenerative Diseases}
\label{chapter:dkt}

\newcommand{\expFld}{images/}
\newcommand{\jmdFld}{../jointModellingDisease}


\section{Contributions}

In this chapter I present Disease Knowledge Transfer (DKT), a novel method for transferring biomarker information between related neurodegenerative diseases. I performed the mathematical modelling, implementation of the DKT method, data pre-processing, statistical analysis and model validation. The TADPOLE dataset has been assembled by myself and Neil Oxtoby, with suggestions from the EuroPOND team. The PCA dataset was acquired by the Dementia Research Centre, UK. 

While the original DKT implementation relied on a non-parametric GP disease progression model by Marco Lorenzi \cite{lorenzi2017disease} as a building block, for this thesis I chose a simpler parametric model, due to the complexity of fitting hierarchical, non-parametric, latent-space models.

\section{Publications}
\begin{itemize}
 \item R. V. Marinescu, M. Lorenzi, S. B. Blumberg, P. Planell-Morell, A. L. Young, N. P. Oxtoby,  A. Eshaghi, K. X. X. Yong, S. Crutch, D. C. Alexander, arXiv, 2019.
\end{itemize}


\section{Introduction}
\label{sec:dktInt}

The estimation of accurate biomarker signatures in Alzheimer's disease (AD) and related neurodegenerative diseases is crucial for understanding underlying disease mechanisms, predicting subjects' progressions, and selecting the right subjects in clinical trials. As a result, data-driven disease progression models (chapter \ref{chapter:bckDpm}) were proposed that reconstruct long term biomarker signatures from collections of short term individual measurements. When applied to large datasets of typical AD, disease progression models have shown important benefits in understanding the earliest events in the Alzheimer's disease cascade \cite{iturria2016early, young2014data}, the heterogeneity of AD \cite{young2018uncovering}, helped discover novel genes involved in AD \cite{scelsi2018genetic} and they showed improved predictions over standard approaches \cite{oxtoby2018}. However, by necessity these models require large datasets -- in addition they must be both multimodal and longitudinal. Such data is not available in rare neurodegenerative diseases. In particular, most datasets for rare neurodegenerative diseases come from local clinical centres, are unimodal (e.g. MRI only) and limited both cross-sectionally and longitudinally -- this makes the application of disease progression models extremely difficult.  Moreover, such a model estimated from common diseases such as typical AD may not generalise to specific variants. For example, in Posterior Cortical Atrophy -- a neurodegenerative syndrome causing visual disruption -- posterior regions such as the occipital lobe and superior parietal regions are affected early, instead of the hippocampus and temporal regions that are affected early in typical AD. 

The problem of limited data in medical imaging has so far been addressed through transfer learning methods. Such techniques have been successfully used to improve the accuracy of AD diagnosis \cite{hon2017towards, cheng2017multi} or prediction of MCI conversion \cite{cheng2015domain}, but have two key limitations. First, they use deep learning or other machine learning methods, which are not interpretable and don't allow us to understand underlying disease mechanisms that are either specific to rare diseases, or shared across related diseases. Secondly, these models cannot be used to forecast the future evolution of subjects at risk of dementia, which is important for selecting the right subjects in clinical trials. 

We propose Disease Knowledge Transfer (DKT), a generative joint model that estimates continuous multimodal biomarker progressions for multiple neurodegenerative diseases simultaneously -- including rare neurodegenerative diseases -- and which inherently performs transfer learning between the modelled phenotypes. This is achieved by exploiting biomarker relationships that are shared across diseases, whilst accounting for differences in the spatial distribution of brain pathology. DKT is interpretable, which allows us to understand underlying disease mechanisms, and can also predict the future evolution of subjects at risk of diseases. We apply DKT on Alzheimer's variants and demonstrate its ability to predict non-MRI trajectories for patients with Posterior Cortical Atrophy, in lack of such data. This is done by fitting DKT to two datasets simultaneously: (1) the TADPOLE Challenge \cite{marinescu2018tadpole} dataset containing subjects from the Alzheimer's Disease Neuroimaging Initiative (ADNI) with MRI, FDG-PET, DTI, AV45 and AV1451 scans and (2) MRI scans from patients with Posterior Cortical Atrophy from the Dementia Research Centre (DRC), UK. We first show that the estimated non-MRI trajectories for PCA subjects are plausible as they agree with previous literature findings. We finally validate DKT on three datasets: 1) simulated data with known ground truth, 2) TADPOLE sub-populations with different progressions and 3) 20 DTI scans from controls and PCA patients from the DRC, showing it yields favourable performance compared to standard approaches. Code for DKT is available online: \url{https://github.com/mrazvan22/dkt}.


\begin{figure}[h]
 \centering
 \includegraphics[width=1\textwidth,trim=0 0 0 0,clip]{images/disease_knowledge_transfer.pdf}
 \caption[Diagram of the proposed framework for joint modelling of multiple diseases.]{Diagram of the proposed framework for joint modelling of multiple diseases. We assume that each disease can be modelled as the evolution of abstract dysfunctionality scores (Y-axis, top row), each one related to different brain regions. Each region-specific dysfunctionality score then further models (X-axis, bottom row) the progression of several modality-specific biomarkers within that same region. For instance, the temporal dysfunction, modelled as a biomarker in the disease specific model (top row), is the X-axis in the disease agnostic model (temporal unit, bottom row), which aggregates together abnormality from amyloid, tau and MR imaging within the temporal lobe. The biomarker correlations within the bottom units are assumed to be disease agnostic and shared across all diseases modelled. Disease knowledge transfer can then be achieved via the disease-agnostic units.}
 \label{fig:diagram}
\end{figure}

\section{Methods}
\label{sec:dktMet}


\subsection{DKT Framework}
\label{sec:dktMethFramework}
\newcommand{\lp}{\lambda_{d_i}^{\psi(k)}}
\newcommand{\lpuu}{\lambda_{d_i}^{\psi(k),(u)}}
\newcommand{\lpum}{\lambda_{d_i}^{\psi(k),(u-1)}}


Fig. \ref{fig:diagram} shows the overall diagram of our proposed framework for joint modelling of diseases. We assume that the progression of each disease (X-axis, top row) can be modelled as the evolution of abstract dysfunctionality scores, each one related to different brain regions (top row). Each dysfunctionality score is then modelled as the progression of several biomarkers within that same region, but acquired using different noninvasive imaging modalities (bottom row). Each group of biomarkers in the bottom row will be called a \emph{functional unit}, because the correlations between biomarkers are related through common "function" in a disease--agnostic way, since they are related to the same underlying brain region. Biomarker groupings into functional units are defined a-priori. We choose to model the correlations within each unit using the disease progression model (DPM) by Jedynak et al. \cite{jedynak2012computational}, but any other DPM can also be used. The DPM allows us to reconstruct unit-specific \emph{dysfunction} progression manifolds (bottom row, X axis), which can be used for staging subjects. Finally, we use the same DPM to express the progression within each disease (Figure 1, top) in terms of the dysfunction scores estimated within each functional unit. More precisely, the X-axis dysfunction scores from the functional units become Y-axis measurements in the disease specific models.

The model has a concise mathematical formulation. We assume a set of given biomarkers measurements $Y = [y_{ijk} | (i,j,k) \in \Omega]$ for subject $i$ at visit $j$ in biomarker $k$, where $\Omega$ is defined as the set of available biomarker measurements, since subjects can have missing data at various visits. We assume that each subject $i$ at each visit $j$ has an underlying disease stage $s_{ij} = \beta_i + m_{ij}$, where $m_{ij}$ represents the months since baseline visit for subject $i$ at visit $j$ and $\beta_i$ represents the time shift of subject $i$. We further denote by $\theta_k$ the parameters used to represent the trajectory for biomarker $k \in K$ within its functional unit $\psi(k)$, where $\psi$: \{1, ..., K\} $ \rightarrow \Lambda$ is a function that maps each biomarker $k$ to a unique functional unit $l \in \Lambda$, where $\Lambda$ is the set of functional units. Moreover, we denote by $\lambda_d^l$ the parameters for the trajectory of the dysfunction score corresponding to functional unit $l \in \Lambda$ in the space of disease $d$. These definitions allow us to formulate the likelihood for a single measurement $y_{ijk}$ as follows:

\begin{equation}
 p(y_{ijk}|\theta_k, \lp, \beta_i, \epsilon_k) = N(y_{ijk}| g(f(\beta_i + m_{ij}); \lp; \theta_k), \epsilon_k)
\end{equation}
where $g(\ .\ ; \theta_k)$ represents the trajectory of biomarker $k$ within functional unit $\psi(k)$ and $f(\ .\ ; \lambda_{d_i}^{\psi(k)})$ represents the trajectory of the functional unit $\psi(k)$ within the space of disease $d_i$. To be precise, $d_i \in \mathbb{D}$ represents the index of the disease space where subject $i$ belongs, where $\mathbb{D}$ is the set of all diseases modelled. For example, MCI and tAD subjects from ADNI as well as tAD subjects from the DRC cohort can all be assigned $d_i=1$, while PCA subjects from the DRC dataset can be assigned $d_i=2$. Healthy controls can be assigned to either disease space, although a more precise assignment would take molecular biomarkers into account. Variable $\epsilon_k$ denotes the variance of measurements for biomarker $k$. 

We extend the above model to multiple subjects, visits and biomarkers to get the full model likelihood:
\begin{equation}
 p(\boldsymbol{y}|\theta, \lambda, \beta , \epsilon) = \\ \prod_{(i,j,k) \in \Omega} p(y_{ijk}|\theta_k, \lp, \beta_i) 
\end{equation}

where $\boldsymbol{y} = [y_{ijk} | \forall (i,j,k) \in \Omega ]$ is the vector of all biomarker measurements, while $\boldsymbol{\theta} = [\theta_1, ..., \theta_K]$ represents the stacked parameters for the trajectories of biomarkers in functional units, $\boldsymbol{\lambda} = [\lambda_d^{l}|l \in \Lambda, d \in \mathbb{D}]$ are the parameters of the dysfunctionality trajectories within the disease models, $\boldsymbol{\beta} =[\beta_1, ..., \beta_N]$ are the subject-specific time shifts and $\boldsymbol{\epsilon} = [\epsilon_k | k \in K]$  estimates biomarker measurement noise. Although we assumed independence across different subjects, biomarker measurements and visits are linked using the latent time-shift $\beta_i$ for each subject. The parameters of the model that need to be estimated are $[\boldsymbol{\theta}, \boldsymbol{\lambda}, \boldsymbol{\beta}, \boldsymbol{\epsilon}]$. For model simplicity, we did not account for inter-individual variability other than that expressed by the time-shift $\beta_i$, although this could be extended in future work.

\subsection{Modelling Biomarker Trajectories}
\label{sec:dktBiomkTraj}

So far we defined the DKT framework using generic models $g(\ .\ ; \theta_k)$ and $f(\ .\ ; \lp)$ for the biomarker trajectories within the functional units and the disease models. Now we choose to implement the $f$ and $g$ models as parametric sigmoidal curves, to enable a robust optimisation and because these models account for the floor and ceiling effects normally observed in AD biomarkers \cite{sabuncu2011dynamics,caroli2010dynamics}. The sigmoidal model for $f$ is defined as:

\begin{equation}
 f(s;\theta_k) = \frac{a_k}{1+exp(-b_k(s-c_k))} + d_k
\end{equation}

where $s$ is the disease progression score of a subject and $\theta_k = [a_k, b_k, c_k, d_k]$ are parameters controlling the shape of the trajectory for biomarker $k$: $d_k$ and $d_k + a_k$ represent the lower and upper limits of the sigmoidal function, $c_k$ represents the inflection point and $a_k b_k/4$ represents the slope at the inflection point. A similar model is used also for $g$. 

\subsection{Parameter Estimation}

\newcommand{\uu}{^{(u)}}
\newcommand{\um}{^{(u-1)}}


We estimate the model parameters using a two-stage approach. In the first stage, we perform belief propagation within each functional unit and then within each disease model. Each functional unit and disease model is assumed to be an independent disease progression model that we fit by alternatively optimising the fit of biomarker trajectories and subject-specific time-shifts, using the approach described in \cite{jedynak2012computational}. At this stage we assume the existence of a latent variable $\beta_i^{\psi(k)} = f(\beta_i + m_{ij}; \lp)$ representing the dysfunctionality score of subject $i$ within the functional unit $\psi(k)$, which represents a time-shift within that functional unit.

In the second stage we jointly optimise across all functional units and disease models using loopy belief propagation. An overview of the algorithm is given in Figure \ref{fig:dktAlgo}. Given the initial parameters estimated from the first stage (line 1), the algorithm continuously updates the biomarker trajectories within the functional units (lines 4-5), dysfunctionality trajectories (line 9) and subject-specific time shifts (line 13) until convergence. The cost function for all parameters is nearly identical, the main difference being the measurements $(i,j,k)$ over subjects $i$, visits $j$ and biomarkers $k$ that are selected for computing the measurement error. For estimating the trajectory of biomarker $k$ within functional unit $\psi(k)$, measurements are taken from $\Omega_k$ representing all measurements of biomarker $k$ from all subjects and visits. For estimating the dysfunctionality trajectories,  $\Omega_{d,l}$ represents the measurement indices from all subjects with disease $d$ (i.e. $d_i = d$) and all biomarkers $k$ that belong to functional unit $l$ (i.e. $\psi(k) = l$). Finally, $\Omega_i$ (line 13) represents all measurements from subject $i$, for all biomarkers and visits. 

The algorithm we proposed in Figure \ref{fig:dktAlgo} has a complexity of $O(I*S)$, where $S$ is the number of subjects in the dataset and $I$ is the number of iterations until convergence. In practice, convergence is achieved after around 10-15 iterations, which takes around 1h on a Xeon CPU E5-2680 @ 2.5GHz.


\begin{figure}
\begin{algorithm}[H]
 Initialise $\boldsymbol{\theta}^{(0)}$, $\boldsymbol{\lambda}^{(0)}$, $\boldsymbol{\beta}^{(0)}$\\
  \While{$\boldsymbol{\theta}$, $\boldsymbol{\lambda}$, $\boldsymbol{\beta}$ not converged}{
   \tcp*[l]{Estimate biomarker trajectories (disease agnostic)}
    \For{$k=1$ to $K$}{
      ${\theta_k\uu = \argmin_{\theta_k} \sum_{(i,j) \in \Omega_k} \left[y_{ijk} - g\left(f(\beta_i\um + m_{ij}; \lpum) ; \theta_k\right) \right]^2  - log\ p(\theta_k)}$\\
      ${\epsilon_k\uu = \frac{1}{|\Omega_k|} \sum_{(i,j) \in \Omega_k}    \left[y_{ijk} - g\left(f(\beta_i\um + m_{ij}; \lpum) ; \theta_k\uu \right) \right]^2 }$\\
    }
     \tcp*[l]{Estimate dysfunctionality trajectories (disease specific)} 
    \For{$d=1 \in \mathbb{D}$}{
      \For{$l=1 \in \Lambda$}{
        ${\lambda_{d}^{l, (u)} = \argmin_{\lambda_{d}^{l}} \sum_{(i,j,k) \in \Omega_{d,l}} \left[y_{ijk} - g\left(f(\beta_i\um + m_{ij}; \lambda_{d}^{l}) ; \theta_k\uu 
        \right) \right]^2  - log\ p(\lambda_{d}^{l})}$\\
      }
    }
    \tcp*[l]{Estimate subject-specific time shifts} 
    \For{$i=1 \in [1, \dots, S]$}{
      ${\beta_i\uu = \argmin_{\beta_i} \sum_{(j,k) \in \Omega_i} \left[y_{ijk} - g\left(f(\beta_i + m_{ij}; \lpuu) ; \theta_k\uu
      \right) \right]^2  - log\ p(\beta_i)}$\\
    }
}
\end{algorithm}
\caption[The algorithm for estimating the DKT parameters]{The algorithm for estimating the DKT parameters. The algorithm successively updates the biomarker trajectories within the functional units (disease agnostic models), dysfunctionality trajectories (disease specific) and subject-specific time shifts until convergence.}
\label{fig:dktAlgo}
\end{figure}

\subsection{Synthetic Experiment}
\label{sec:dktMetSyn}

We first test DKT on synthetic data, in order to assess the performance when ground truth is known. We generate synthetic data from two diseases as follows:
\begin{itemize}
 \item[] \textbf{Disease model}
 \item We define two functional units $l_0$ and $l_1$ and 6 biomarkers $k_0-k_5$, which we allocate to functional units as follows: $l_0:\{k_0, k_2, k_4\}$, $l_1: \{k_1, k_3, k_5\}$. Within their units, we define the trajectory of each biomarker as a sigmoidal curves with the following $\theta_k$ parameters:
 \begin{itemize}
  \item functional unit $l_0$: $\theta_0 = (1,5,0.2,0)$, $\theta_2 = (1,5,0.55,0)$ and $\theta_4 = (1,5,0.9,0)$ 
  \item functional unit $l_1$: $\theta_1 = (1,10,0.2,0)$, $\theta_3 = (1,10,0.55,0)$ and $\theta_5 = (1,10,0.9,0)$ 
 \end{itemize}
 \item We define two synthetic diseases, "synthetic AD" ($d=0$) and "synthetic PCA" ($d=1$). For each disease $d$, each functional unit $l$ has a distinct dysfunctionality trajectory defined as a sigmoidal curve with parameters $\lambda_d^l$ as follows: 
 \begin{itemize}
  \item "synthetic AD" disease: $\lambda_0^0 = (1, 0.3, -4, 0)$  and $\lambda_0^1 = (1, 0.2, 6, 0)$.
  \item "synthetic PCA" disease: $\lambda_1^0 = (1, 0.3, 6, 0)$ and $\lambda_1^1 = (1, 0.2, -4, 0)$.
 \end{itemize}

 \item[] \textbf{Subject model}
 \item We generated time-shifts $\beta_i$ for 100 subjects (disease $d_0$) and 50 subjects (disease $d_1$) based on a uniform distribution with ranges $(-13, 10)$ years before/after disease onset. 
 \item Within each disease, we generated the subjects' diagnosis (controls/patients) based on an exponential likelihood model with mean -4.5 (controls)/4.5 (patients) years before/after disease onset. 
 \item For each subject and each biomarker, we generated data for four consecutive visits, each visit one year apart, using a noise standard deviation of 0.05.
\end{itemize}

These trajectory and subject parameters were chosen to mimic the TADPOLE and DRC cohorts, described below. Before fitting DKT on the synthetic dataset, we discarded the data from biomarkers $k_0$, $k_1$, $k_4$ and $k_5$ for all subjects within the synthetic PCA cohort, to simulate the lack of multimodal data in these subjects. Remaining biomarkers $k_2$ and $k_3$, for which data was still available in the synthetic PCA cohort, are assumed to be of the same modality (e.g. MRI volume) but to represent measurements from different brain regions (e.g. temporal and occipital). 


\subsection{Data Acquisition and Preprocessing}

We trained DKT on ADNI data from the TADPOLE challenge \cite{marinescu2018tadpole}, since it contained a large number of multimodal biomarkers already pre-processed and aggregated into one table. From the TADPOLE dataset we selected a subset of 230 subjects which had at least one FDG PET, AV45, AV1451 or DTI scan. Most subjects also had MRI scans and cognitive tests. In order to model another disease, we further included 76 PCA subjects from the DRC in the training set, along with 67 tAD and 87 age-matched controls, all of which only had MRI scans. 

For both datasets, volumetric measures for each subject have been obtained using the Freesurfer software. For FDG, AV45 and AV1451 PET, we used already extracted SUVR measures from ADNI. For DTI, we used fractional anisotropy (FA) measures from white-matter regions adjacent to each lobe. For every lobe, we averaged the biomarker values for regions of interest within each lobe and regressed out the following covariates: age, gender, total intracranial volume (TIV) and dataset (ADNI vs DRC dataset). Finally, we normalised the biomarker values to lie within the [0,1] range. 

For validating DKT's performance at predicting missing biomarkers in PCA, we used a separate test set of DTI scans from the DRC controls and PCA subjects. As this validation set was acquired at a centre different from ADNI and on different scanners, we matched the FA mean and standard deviation of the DRC controls to be equal to the FA mean and standard deviation of the ADNI controls. No DTI data from PCA subjects was exposed to the algorithm at training time.



\section{Results}
\label{sec:dktRes}

\subsection{Synthetic Results}
\label{sec:dktResSyn}

Fig. \ref{fig:dktSynthTrajCompTrue} shows the true and estimated subject shifts and trajectories for each functional unit $l$ and biomarker $k$. In the top-left figures we show scatter plots of the true shifts (y-axis) against estimated shifts (x-axis), for the 'synthetic AD' and 'synthetic PCA' diseases. On the top-right and middle-left figures, we show the trajectories of the functional units within disease $d=0$ (synthetic AD) and $d=1$ (synthetic PCA). In the middle-right and bottom-left figures, we show the biomarker trajectories within units $l_0$ and $l_1$. In Figure \ref{fig:dktSynthTrajDrcSpace}, we show the corresponding trajectories of PCA patients, which as opposed to Fig. \ref{fig:dktSynthTrajCompTrue}, are plotted directly against the time-shifts, as it is normally done in a classical disease progression model. We also show the true trajectories and the data of the synthetic PCA cohort.

The results in Fig. \ref{fig:dktSynthTrajCompTrue} suggest that the DKT-estimated trajectories match closely (mean absolute error, MAE $<$ 0.058) with the true trajectories, for both the unit-trajectories within the disease-specific models and the biomarker trajectories within the disease-agnostic models. Moreover, the subject time-shifts are very close ($R^2$ $>$ 0.98) to the true time-shifts. When plotted directly against the disease space, the estimated PCA trajectories also match the true trajectories, even when there is a complete lack of such data (Fig. \ref{fig:dktSynthTrajDrcSpace}, biomarkers 0,1,4 and 5). There are however small errors in  biomarkers 0 and 5 which are due to measurement noise (confirmed by experiments with smaller noise level -- not shown here). The equivalent trajectories estimated for the synthetic AD cohort also show very good agreement with the true trajectories (Fig. \ref{fig:dktSynthTrajADSpace}).

% \begin{figure}
% \includegraphics[width=\textwidth]{images/dkt/plotHierData611_synth1Pen5_JMD.png}
%  \caption{(top-left) (top-left) Scatter plots of the true shifts (y-axis) against estimated shifts (x-axis), for the 'synthetic AD' (left) and 'synthetic PCA' (right) diseases. (top-right and middle-left) Trajectories of unit }
%  \label{fig:dktSynthTrajHierData}
% \end{figure}

\begin{figure}
\includegraphics[width=\textwidth]{images/compTrueParams101_synth1_JMD.pdf}
 \caption[DKT Simulation Results - Comparison between true and DKT-estimated biomarker trajectories and subject time-shifts.]{Comparison between true and DKT-estimated subject time-shifts and biomarker trajectories. (top-left) Scatter plots of the true shifts (y-axis) against estimated shifts (x-axis), for the 'synthetic AD' (left) and 'synthetic PCA' (right) diseases. We also show the DKT-estimated and true trajectories of the functional units within the 'synthetic AD' disease (top-right) and the 'synthetic PCA' disease (middle-left). For these figures, the x-axis measures the normalised disease progression score $s_i$ while the y-axis measures the dysfunctionality scores $f(s_i;\lambda_d^l)$. Finally, we also show the biomarker trajectories within unit 0 (middle-right) and unit 1 (bottom), where the x-axis represents the dysfunctionality scores $f(s_i;\lambda_d^l)$ and the y-axis represents the biomarker value.}
 \label{fig:dktSynthTrajCompTrue}
\end{figure}

\begin{figure}
\includegraphics[width=\textwidth]{images/trajDisSpaceDis1_101_synth1_JMD.pdf}
 \caption[Estimated biomarker trajectories for the "synthetic PCA" disease, plotted alongside true trajectories]{Estimated biomarker trajectories for the "synthetic PCA" disease, plotted alongside true trajectories. Estimation of the trajectories in biomarkers 0,1,4 and 5 has been done without any data from the "synthetic PCA" disease, only based on the disease-agnostic correlations with biomarkers 2 and 3.}
 \label{fig:dktSynthTrajDrcSpace}
\end{figure}


\subsection{Results on TADPOLE and DRC Datasets}
\label{sec:dktResTadDrc}

% Fig1: biomarker traj. over dysfunction scores in one functional unit -> Fig2: dysfunction trajectories over disease stage in the two disease models -> Fig3: inferred biomarker trajectories "directly" over disease stage  in PCA
Fig. \ref{fig:dktFitUnit} shows the estimated biomarker trajectories within the \emph{occipital unit} plotted over the dysfunction scores, along with samples from the model posterior and aligned subject data. The X-axis shows the dysfunctionality scores within the occipital unit, which represent estimated time-shifts, in months, from an arbitrary reference X=0, while the Y-axis shows biomarker values normalised to [0,1] range. The model shows an unbiased data fit (Fig. \ref{fig:dktFitUnit}), and we can observe most PCA subjects having abnormal occipital volumes, thus leading to high occipital dysfunctionality scores, in line with the current understanding of PCA as affecting posterior regions \cite{crutch2012posterior}. We also show the progression of dysfunctionality scores over the disease stage for typical AD (Fig \ref{fig:dktFitAD}) and PCA (Fig \ref{fig:dktFitPCA}). In typical AD, we see that hippocampal dysfunction becomes abnormal earliest, while PCA shows early hippocampal dysfunction, which is later exceeded by the dysfunction in the occipital, temporal and parietal regions, which are known to be affected in PCA \cite{crutch2012posterior,Baron2001}. 

In Fig. \ref{fig:PCAtrajByModality}, we plot the inferred biomarker trajectories for PCA directly across the disease progression. We do this for five different modalities: MRI Volumes, DTI, FDG, AV45 and AV1451. The results again recapitulate known patterns in PCA, where posterior regions are predominantly affected in all modalities. However, for MRI volumes and AV45, we also see early abnormalities, which we attribute to the models underestimating the biomarker measurement noise.


\begin{figure}
\centering
\begin{subfigure}{\textwidth}
\centering
\includegraphics[width=1\textwidth, trim=90 20 110 0, clip]{\expFld/unit1_allTraj.png}
\caption{Occipital unit}
\label{fig:dktFitUnit}
\end{subfigure}
\vspace{2em}

\begin{subfigure}{0.47\textwidth}
\centering
% typical AD\\
\includegraphics[width=1\textwidth, trim=0 0 0 20, clip]{\expFld/dis0_tAD_allTrajZeroOne.png}
\caption{tAD}
\label{fig:dktFitAD}
\end{subfigure}
\begin{subfigure}{0.47\textwidth}
\centering
% PCA\\
\includegraphics[width=1\textwidth, trim=0 0 0 20, clip]{\expFld/dis1_PCA_allTrajZeroOne.png}
\caption{PCA}
\label{fig:dktFitPCA}
\end{subfigure}
\caption[DKT results - biomarker trajectories in the occipital unit and dysfunctionality scores for tAD and PCA]{(a) DKT-estimated biomarker trajectories in the occipital functional unit. Subject data from ADNI and our local DRC cohort are also shown. The X-axis, defined as the occipital dysfunctionality score, represents the time-shifts (in months) of each subject. (b-c) Progression of DKT-estimated dysfunctionality scores for (b) typical AD and (c) PCA.}
\label{fig:pcaTadDisSpace}
\end{figure}

% estimated (hypothetical) trajectories in PCA: DTI, FDG, AV45, AV1451.Volumetric trajectories were based on PCA MRI data.
\begin{figure}
 \includegraphics[width=\textwidth, trim=0 20 0 0, clip]{\expFld/trajDisSpaceOverlap_PCA_tad-drcTinyPen5_JMD.png}
 \caption[Estimated multi-modal trajectories for the PCA cohort.]{Estimated multi-modal trajectories for the PCA cohort. The only data that were available were the MRI volumetric data. The dynamics of the other biomarkers has been inferred by the model using data from typical AD, and taking into account the different spatial distribution of pathology in PCA as compared to typical AD.}
 \label{fig:PCAtrajByModality}
\end{figure}


\section{Validation on DTI Data in PCA}
\label{sec:dktResVal}

We further validated DKT by predicting unseen DTI data from two patient datasets:
\begin{itemize}
 \item TADPOLE subjects with a different progression from the training subjects
 \item A separate test set of 20 DTI scans from controls and PCA patients from our own cohort.
\end{itemize}

To split TADPOLE into subgroups with different progression, we used the SuStaIn model by \cite{young2018uncovering}, which resulted into three subgroups: hippocampal, cortical and subcortical, with prominent early atrophy in the hippocampus, cortical and subcortical regions respectively. To evaluate prediction accuracy, we computed the rank correlation between the DKT-predicted biomarker values and the measured values in the test data. We compute the rank correlation instead of mean squared error as it is not susceptible to systemic biases of the models when predicting "unseen data" in a certain disease. We also compared the performance of DKT at predicting unseen data with four other models: 
\begin{itemize}
 \item \emph{Latent stage model}: a sigmoidal based disease progression model, as described in \cite{jedynak2012computational}. This model assumes all tAD and PCA subjects follow the same progression.
 \item \emph{Multivariate}: A multivariate Gaussian Process model with RBF kernel that predicts a DTI ROI marker from multiple MRI markers.
 \item \emph{Spline}: a univariate cubic spline regression model that predicts the DTI biomarker based on the corresponding MRI biomarker, independently for each region.
 \item \emph{Linear}: Same as above but linear model instead of spline.
\end{itemize}



Validation results are shown in Table \ref{sec:dktPerfMetrics}, for hippocampal to cortical TADPOLE subgroups, as well as PCA subjects from our DRC cohort. When predicting missing DTI markers from the TADPOLE cortical subgroup as well as PCA subjects, the DKT correlations are generally high for the cingulate, hippocampus and parietal, and lower for the frontal lobe. DKT further shows favourable performance compared to the other models, due to it's ability to disentangle the progressions of each disease separately. In particular, it shows the best results for DTI FA prediction in the parietal and temporal lobes on both datasets and similar performance to the latent-stage model on the PCA dataset for the cingulate, frontal and hippocampal (differences here are not statistically significant). Due to the challenging problem of predicting unseen data in these diseases/subtypes, notice how the models yield bad predictions for the occipital lobe (negative correlations), most likely due to overfitting.


\newcommand{\cw}{c}

\begin{table}
\centering
\fontsize{9}{12}\selectfont
\begin{tabular}{c | c c c c c c}
\textbf{Model} & \textbf{Cingulate} & \textbf{Frontal} & \textbf{Hippocam.} & \textbf{Occipital} & \textbf{Parietal} & \textbf{Temporal}\\
& \multicolumn{6}{c}{\textbf{TADPOLE subgroups: Hippocampal subgroup to Cortical subgroup}}\\
DKT (ours) &      0.56 $\pm$ 0.23 &    \textbf{0.35 $\pm$ 0.17} &        \textbf{0.58 $\pm$ 0.14} &     -0.10 $\pm$ 0.29 &     \textbf{0.71 $\pm$ 0.11} &     \textbf{0.34 $\pm$ 0.26} \\
Latent stage &      0.44 $\pm$ 0.25 &    0.34 $\pm$ 0.21 &       0.34 $\pm$ 0.24* &     \textbf{-0.07 $\pm$ 0.22} &     0.64 $\pm$ 0.16 &    0.08 $\pm$ 0.24* \\
Multivariate &      \textbf{0.60 $\pm$ 0.18} &   0.11 $\pm$ 0.22* &       0.12 $\pm$ 0.29* &     -0.22 $\pm$ 0.22 &   -0.44 $\pm$ 0.14* &   -0.32 $\pm$ 0.29* \\
Spline &    -0.24 $\pm$ 0.25* &  -0.06 $\pm$ 0.27* &        0.58 $\pm$ 0.17 &     -0.16 $\pm$ 0.27 &    0.23 $\pm$ 0.25* &    0.10 $\pm$ 0.25* \\
Linear &    -0.24 $\pm$ 0.25* &   0.20 $\pm$ 0.25* &        0.58 $\pm$ 0.17 &     -0.16 $\pm$ 0.27 &    0.23 $\pm$ 0.25* &    0.13 $\pm$ 0.23* \\
& \multicolumn{6}{c}{\textbf{typical Alzheimer's to Posterior Cortical Atrophy}}\\
DKT (ours) &    0.77 $\pm$ 0.11 &    0.39 $\pm$ 0.26 &      0.75 $\pm$ 0.09 &    0.60 $\pm$ 0.14 &    \textbf{0.55 $\pm$ 0.24} &    \textbf{0.35 $\pm$ 0.22} \\
Latent stage &    \textbf{0.80 $\pm$ 0.09} &    \textbf{0.53 $\pm$ 0.17} &      \textbf{0.80 $\pm$ 0.12} &    0.56 $\pm$ 0.18 &    0.50 $\pm$ 0.21 &    0.32 $\pm$ 0.24 \\
Multivariate &   0.73 $\pm$ 0.09 &   0.45 $\pm$ 0.22  &    0.71 $\pm$ 0.08 & -0.28 $\pm$ 0.21* &  0.53 $\pm$ 0.22  &  0.25 $\pm$ 0.23* \\
Spline &   0.52 $\pm$ 0.20* &  -0.03 $\pm$ 0.35* &     0.66 $\pm$ 0.11* &   0.09 $\pm$ 0.25* &    0.53 $\pm$ 0.20 &   0.30 $\pm$ 0.21* \\
Linear &   0.52 $\pm$ 0.20* &    0.34 $\pm$ 0.27 &     0.66 $\pm$ 0.11* &    \textbf{0.64 $\pm$ 0.17} &    0.54 $\pm$ 0.22 &   0.30 $\pm$ 0.21* \\
\end{tabular}
\vspace{0.5em}
\caption[Performance evaluation of DKT and other models]{Performance evaluation of DKT and four other statistical models of decreasing complexity. We show the rank correlation between predicted biomarkers and measured biomarkers in (top) TADPOLE subgroups -- information transfer from hippocampal subgroup to cortical subgroup -- and (bottom) PCA. (*) Statistically significant difference in the performance of DKT vs the other models, based on a two-tailed t-test, Bonferroni corrected.}
% \end{footnotesize}
\label{sec:dktPerfMetrics}
\end{table}



\section{Discussion}
\label{sec:dktDis}


We presented DKT, a framework that enables, for the first time, joint modelling of biomarker progressions in multiple neurodegenerative diseases simultaneously. The framework allows the inference of biomarker trajectories in rare diseases, for which there is not enough data to allow estimation of such trajectories, and accounts for a different spatial distribution of pathology between distinct types of dementia. This further enables us to understand the complex mechanisms of rare diseases, as well as mechanisms shared between different types of related diseases.

We provided an example implementation of DKT using specific models of the biomarker trajectories, measurement noise and link function (the disease progression score). However, DKT should be considered as a general framework for joint modelling of biomarker trajectories within different diseases simultaneously. The actual implementation of DKT can thus be extended to use non-parametric trajectories, or more complex link functions that estimate not only subject time-shifts but also progression speed or higher order terms.

While in this work we have focused on Alzheimer's variants such as tAD and PCA, DKT can also be applied to other progressive neurodegenerative diseases of non-Alzheimer's type such as tauopathies (e.g. Frontotemporal dementia), synucleinopathies (e.g. Parkinson's disease), other neurodegenerative diseases such as Huntington's disease or Multiple Sclerosis, and even the normal ageing process. Cognitive tests can also be included in the disease-specific sub-models of DKT, or even allocated in the functional units of the regions that are responsible for those tasks, based on previous voxel-based morphometry studies. However, some care needs to be exercised when selecting the biomarkers and grouping them into functional units, as in some diseases the assumption of disease agnostic dynamics might not hold for some groups of molecular biomarkers. For example, some non-Alzheimer's tauopathies such as Frontotemporal dementia might show tau abnormalities but no corresponding amyloid abnormalities within the same region. In the case of Frontotemporal dementia, we recommend including higher-level biomarkers such as glucose metabolism from FDG, white matter degeneration from DTI or volume from structural MRI, but one should exclude amyloid markers. 

Our work has several limitations: 1) DKT assumes all subjects within a disease follow the same trajectory, without considering heterogeneity within the disease population, 2) the allocation of biomarkers into functional units has to be done using \emph{a-priori} human knowledge, 3) DKT currently works only on extracted brain features, discarding important information present in the brain morphometry, 4) for validation, the synthetic experiment we ran was limited to only one setting of the parameters and 5) the validation on patient data was also done only on a small set of 20 DTI scans, due to lack of multimodal data in PCA.

There are several potential avenues for further research: 1) to account for heterogeneity, DKT can also be easily extended to include subject-specific effects; 2) improved schemes for biomarker allocation to functional units can take connectivity into account, or derive it from the data automatically; 3) to account for brain morphometry and connectivity, DKT can be extended into a fully spatio-temporal model, by estimating continuous changes in volumetric brain images -- in this case, each voxel can have an associated dysfunctionality score that is derived from measurements of various modalities from that voxel; 4-5) DKT can be further validated on more complex synthetic experiments with variable parameter settings, and on patient data from ADNI, where the population could be \emph{a-priori} split into sub-groups with different progressions. On these subgroups, DKT can be used to transfer biomarker modalities that have been left out during training.



\section{Conclusion}
\label{sec:dktCon}

In this work I presented DKT, a novel method that can empower studies of rare dementias with limited biomarker data by leveraging data from larger datasets of related dementias. When applied to synthetic data with ground truth, I showed that DKT can robustly recover biomarker trajectories in two distinct diseases and also subject-specific time-shifts. I also applied DKT to multimodal imaging biomarkers from the TADPOLE Challenge dataset, where I showed that it can estimate plausible non-MRI biomarker trajectories for Posterior Cortical Atrophy in lack of such data for this disease. I validated the performance of DKT on a test set of 20 DTI scans from PCA and controls, and showed that DKT has similar or better performance compared to simpler models.

In the next chapter, I will present the TADPOLE  Challenge, which evaluates the performance of algorithms and features at predicting the future evolution of subjects at risk of AD. As opposed to the work performed in this chapter, the TADPOLE challenge aims to evaluate a much larger set of algorithms and features, comprising regression techniques, disease progression models, machine learning techniques and even manual predictions made by clinicians.


\chapter[TADPOLE Challenge: Prediction of Evolution in Alzheimer's Disease]{TADPOLE Challenge: Prediction of Longitudinal Evolution in Alzheimer's Disease}
\label{chapter:tadpole}


\section{Contributions}

In this chapter I present the design of \emph{The Alzheimer's Disease Progression Of Longitudinal Evolution} (TADPOLE) Challenge, which aims to predict the evolution of subjects at risk of Alzheimer's disease. The challenge was organised by the European Progression of Neurodegenerative (EuroPOND) consortium, in collaboration with the Alzheimer's disease Neuroimaging Initiative (ADNI). The key organisers of the challenge were, in alphabetical order: Daniel Alexander, Frederik Barkhof, Esther Bron, Nick Fox, Stefan Klein, Razvan Marinescu (myself), Neil Oxtoby and Alexandra Young. 

I contributed with suggestions to the challenge design, helped write the website, assembled the TADPOLE D2 longitudinal dataset and the data dictionary, and wrote benchmark prediction scripts. I also build the leaderboard system which performs live evaluation of the participants' submissions. I further helped promote the competition at several medical imaging conferences, and organised two mini-competitions at the PyConUK conference and at the CMIC summer school, 2018. 

Daniel Alexander proposed the main design of the challenge, secured funding, helped write the website, and wrote simple prediction scripts. Neil Oxtoby contributed to challenge design, helped me validate the D2 dataset, built the D3 cross-sectional dataset, helped write the website, organised webinars and promoted the competition. Alexandra Young contributed to challenge design, helped write the website, performed simulations to establish which target biomarkers are most suitable and promoted the competition. Esther Bron and Stefan Klein contributed to challenge design and helped write the website. Nick Fox and Frederik Barkhof provided valuable suggestions on the challenge design. Arthur Toga and Michael Weiner offered access to the ADNI database. 

\section{Publications}

\begin{itemize}
\item  R. V. Marinescu, N. P. Oxtoby, A. L. Young, E. E. Bron, A. W. Toga, M. W. Weiner, F. Barkhof, N. C. Fox, S. Klein, D. C. Alexander and the EuroPOND Consortium, TADPOLE Challenge: Prediction of Longitudinal Evolution in Alzheimer's Disease, arXiv, 2018 

I wrote this challenge design paper based on text and diagrams from the TADPOLE website. All collaborators contributed with feedback on the manuscript. The results of the challenge will be published in a separate paper in 2019, after enough data has been collected for the final evaluation.
\end{itemize}

%\FloatBarrier
\section{Introduction}
\label{intro}

As already mentioned in section \ref{chapter:bckDpm}, early diagnosis of dementia is important in order to enable the administration of treatments in early disease stages, before the onset of cognitive decline. While such early and accurate diagnosis of dementia can be challenging, this can be aided by quantitative biomarker measurements taken from magnetic resonance imaging (MRI), positron emission tomography (PET), and cerebro-spinal fluid (CSF) samples extracted from lumbar puncture. It has been hypothesised for AD \cite{jack2010hypothetical,jack2013update,aisen2010clinical,frisoni2010clinical} that all these biomarkers become abnormal at different intervals before symptom onset, suggesting that together they can be used for accurate prediction of onset and overall disease progression in individuals. In particular, some of the early biomarkers become abnormal decades before symptom onset, and can thus facilitate early diagnosis. 

Several approaches for predicting AD-related target variables (e.g. clinical diagnosis, cognitive/imaging biomarkers) have been proposed which leverage multimodal biomarker data available in AD. Traditional longitudinal approaches based on statistical regression model the relationship of the target variables with other known variables. Examples include regression of the target variables against clinical diagnosis \cite{scahill2002mapping}, cognitive test scores \cite{yang2011quantifying, sabuncu2011dynamics}, rate of cognitive decline \cite{doody2010predicting}, and retrospectively staging subjects by time to conversion between diagnoses \cite{guerrero2016instantiated}. Another approach involves supervised machine learning techniques such as support vector machines, random forests, and artificial neural networks, which use pattern recognition to learn the relationship between the values of a set of predictors (biomarkers) and their labels (diagnoses). These approaches have been used to discriminate AD patients from cognitively normal individuals \cite{kloppel2008automatic, zhang2011multimodal}, and for discriminating at-risk individuals who convert to AD in a certain time frame from those who do not \cite{young2013accurate, mattila2011disease}. The emerging approach of disease progression modelling aims to reconstruct biomarker trajectories or other disease signatures across the disease progression timeline, without relying on clinical diagnoses or estimates of time to symptom onset. Examples include models built on a set of scalar biomarkers to produce discrete \cite{fonteijn2012event, young2014data} or continuous \cite{jedynak2012computational, donohue2014estimating, villemagne2013amyloid} biomarker trajectories; richer but less comprehensive models that leverage structure in data such as MR images \cite{durrleman2013toward, lorenzi2015disentangling, bilgel2016multivariate}; and models of disease mechanisms \cite{seeley2009neurodegenerative, zhou2012predicting, raj2012network, iturria2016early}.

These models have shown promise for predicting AD biomarker progression when using existing test data, but few have been tested on truly unseen \emph{future} data. Moreover, different investigators test these models on different datasets (including subsets of a single dataset) and use different processing pipelines. Community challenges have proved effective, in the medical image analysis field and beyond, for providing unbiased comparative evaluations of algorithms and tools designed for a particular task. Previous challenges that focused on prediction of AD progression include the \emph{CADDementia challenge} \cite{bron2015standardized}, which aimed to predict clinical diagnosis from MRI scans. A similar challenge, the "\emph{International challenge for automated prediction of MCI from MRI data}" \cite{sarica2018machine} asked participants to predict diagnosis and conversion status from extracted MRI features of subjects from the ADNI study \cite{weiner2017recent}. Yet another challenge, The Alzheimer's Disease \emph{Big Data DREAM Challenge} \cite{allen2016crowdsourced}, asked participants to predict cognitive decline from genetic and MRI data. 

The Alzheimer's Disease Prediction Of Longitudinal Evolution (TADPOLE) Challenge aims to identify the data, features and approaches that are the most predictive of AD progression. In contrast to previous challenges, our motivation is to improve future clinical trials through identification of patients most likely to benefit from an effective treatment, i.e., those at early stages of disease who are likely to progress over the short-to-medium term (1-5 years). Identifying such subjects reliably helps cohort selection by focusing on groups that highlight positive treatment effects. The challenge thus focuses on forecasting three key features: clinical status, cognitive decline, and neurodegeneration (brain atrophy), over a five-year timescale. It uses \emph{rollover}\footnote{i.e. subjects who enrolled in the previous ADNI2 study and who will continue in the third phase.} subjects from the ADNI study for whom a history of measurements is available, and who are expected to continue in the study, providing future measurements for testing. Since the test data does not exist at the time of forecast submissions, the challenge provides a completely unbiased basis for performance comparison. TADPOLE goes beyond previous challenges by drawing on a vast set of multimodal measurements from ADNI which support prediction of AD progression. 


\section{Competition Design}
\label{design}

The aim of TADPOLE is to predict future outcome measurements of subjects at-risk of AD, enrolled in the ADNI study. A history of informative measurements from ADNI (imaging, psychology, demographics, genetics, etc.) from each individual is available to inform forecasts. TADPOLE participants are required to predict future measurements from these individuals and submit their predictions before a given submission deadline.  Evaluation of these forecasts occurs post-deadline, after the measurements have been acquired. A diagram of the TADPOLE flow is shown in Fig \ref{fig:design}. 

\begin{figure}[h]
 \centering
 \includegraphics[width=0.7\textwidth]{images/challenge_design.png}
 \caption[Diagram showing the TADPOLE Challenge design]{TADPOLE Challenge design. Participants are required to train a predictive model on a training dataset (D1 and/or others) and make forecasts for different datasets (D2, D3) by the submission deadline. Evaluation will be performed on a test dataset (D4) that is acquired after the submission deadline.}
 \label{fig:design}
\end{figure}

\section{Forecasts}

Since we do not know the exact time of future data acquisitions for any given individual, TADPOLE challenge participants are required to make, for every individual, month-by-month forecasts of three key biomarkers: (1) clinical diagnosis which can be either cognitively normal (CN), mild cognitive impairment (MCI) or probable Alzheimer's disease (AD); (2) ADAS-Cog13 (ADAS13) score; and (3) ventricle volume (divided by intra-cranial volume). Evaluation is performed using forecasts at the months that correspond to data acquisition. TADPOLE forecasts are required to be probabilistic and some evaluation metrics will account for forecast probabilities provided by participants. Methods or algorithms that do not produce probabilistic estimates can still be used, by setting binary probabilities (zero or one) and default confidence intervals.

Participants are required to submit forecasts in a standardised format (see Table \ref{tab:subFormat}). For clinical status, relative likelihoods of each option (CN, MCI, and AD) for each individual should be provided. These are normalised at evaluation time; negative likelihoods are set to zero. For ADAS13 and ventricle volume, participants need to provide a best-guess value as well as a 50\% confidence interval for each individual. This 50\% confidence interval (as opposed to the more standard 95\%) was chosen to provide a more symmetric and less noisy evaluation of over- and under-estimation of the confidence interval, because similar sample sizes of data fall inside and outside the interval. 

\newcommand{\wi}{1.4cm}
\setlength\tabcolsep{3pt} % default value: 6pt


\begin{table}
\small
 \begin{tabular}{>{\centering\arraybackslash}m{0.8cm}  >{\centering\arraybackslash}m{1.2cm}  >{\centering\arraybackslash}m{1.3cm}  >{\centering\arraybackslash}m{1.0cm}  >{\centering\arraybackslash}m{1.0cm}  >{\centering\arraybackslash}m{1.0cm}  >{\centering\arraybackslash}m{0.9cm}  >{\centering\arraybackslash}m{1.2cm}  >{\centering\arraybackslash}m{1.2cm}  >{\centering\arraybackslash}m{1.3cm}  >{\centering\arraybackslash}m{1.3cm}  >{\centering\arraybackslash}m{1.3cm}}
  \textbf{RID} & \textbf{Month} & \textbf{Date} & \textbf{CN prob.} & \textbf{MCI prob.} & \textbf{AD prob.} & \textbf{ADAS} & \textbf{ADAS CI lower} & \textbf{ADAS CI upper} & \textbf{Vent.} & \textbf{Vent. CI lower} & \textbf{Vent. CI upper}\\
  \hline
  A & 1 & 2018-01 & 0 & 1 & 0 & 30 & 25 & 35 & 0.024 & 0.021 & 0.029\\
  B & 1 & 2018-01 & 3 & 2 & 0 & 25 & 21 & 26 & 0.023 & 0.021 & 0.025\\
  C & 1 & 2018-01 & 0.24 & 0.38 & 0.38 & 40 & 25 & 50 & 0.025 & 0.023 & 0.028\\
  
 \end{tabular}
  
 \caption{The format of the forecasts for three example subjects. Participants have to predict, for each subject, the probability of clinical diagnosis (CN/MCI/AD), the ADAS-Cog13 score and Ventricle volume, as well as the 50\% confidence range. RID - Roster ID is the unique identifier for ADNI subjects, ADAS - ADAS-Cog13, CI -  confidence range. Note that, even if the CN/MCI/AD probabilities don't sum to one, we will normalise them anyway.}
 \label{tab:subFormat}
\end{table}



% \FloatBarrier
\section{Data}

We provide participants with a standard ADNI-derived dataset (available via the Laboratory Of NeuroImaging: LONI) which they can use to train their algorithms, removing the need to pre-process the ADNI data themselves or merge different spreadsheets. However, participants are allowed to use a custom training set, by adding any other ADNI data or data from other studies. The software code used to generate the standard dataset is openly available in a GitHub repository\footnote{https://github.com/noxtoby/TADPOLE} and on the ADNI website, packaged with the standard dataset in the LONI ADNI database.

\subsection{ADNI Data}

Data used in the preparation of this article were obtained from the Alzheimer's Disease Neuroimaging Initiative (ADNI) database (\url{adni.loni.usc.edu}). The ADNI was launched in 2003 by the National Institute on Aging (NIA), the National Institute of Biomedical Imaging and Bioengineering (NIBIB), the Food and Drug Administration (FDA), private pharmaceutical companies and non-profit organisations, as a \$60 million, 5-year public-private partnership. The initial goal of ADNI was to recruit 800 subjects but ADNI has been followed by ADNI-GO and ADNI-2. To date these three protocols have recruited over 1500 adults, aged 55 to 90, to participate in the research, consisting of cognitively normal older individuals, people with early or late MCI, and people with early AD. The general ADNI inclusion criteria has been described in \cite{petersen2010alzheimer}. 

The data we used from ADNI consists of: (1) CSF markers of amyloid-beta and tau deposition; (2) various imaging modalities such as magnetic resonance imaging (MRI), positron emission tomography (PET) using several tracers: Fluorodeoxyglucose (FDG, hypometabolism), AV45 (amyloid), AV1451 (tau) as well as diffusion tensor imaging (DTI); (3) cognitive assessments acquired in the presence of a clinical expert; (4) genetic information such as Alipoprotein E4 (APOE4) status extracted from DNA samples; and (5) general demographic information. Extracted features from this data were merged together into a final spreadsheet and made available on the LONI ADNI website.

\subsection{Image Preprocessing}

The imaging data has been pre-processed with standard ADNI pipelines.  For MRI scans, this included correction for gradient non-linearity, B1 non-uniformity correction and peak sharpening\footnote{see MRI analysis on ADNI website: \url{http://adni.loni.usc.edu/methods/mri-analysis/
mri-pre-processing}}. Meaningful regional features such as volume and cortical thickness were extracted using the Freesurfer cross-sectional and longitudinal pipelines \cite{reuter2012within}. Each PET image (FDG, AV45, AV1451), which consists of a series of dynamic frames, had its frames co-registered, averaged across the dynamic range, standardised with respect to the orientation and voxel size, and smoothed to produce a uniform resolution of 8mm full-width/half-max (FWHM)\footnote{see PET analysis on ADNI website: \url{http://adni.loni.usc.edu/methods/pet-analysis/pre-processing}}. Standardised uptake value ratio (SUVR) measures for relevant regions-of-interest were extracted (see \cite{jagust2010alzheimer}) after registering the PET images to corresponding MR images using the SPM5 software \cite{ashburner2009computational}. DTI scans were corrected for head motion and eddy-current distortion, skull-stripped, EPI-corrected, and finally aligned to the T1 scans using the pipeline from \cite{nir2013effectiveness}. Diffusion tensor summary measures were estimated based on the Eve white-matter atlas by \cite{oishi2009atlas}. 

% \FloatBarrier
\section{TADPOLE Datasets}
\label{datasets}


The TADPOLE Challenge involves three kinds of data sets: (a) a \emph{training data set}, which is a collection of measurements with associated outcomes that can be used to fit models or train algorithms; (b) a \emph{prediction data set}, which contains only baseline measurements (possibly longitudinal), without associated outcomes --- this is the data that algorithms, models, or experts use as input to make their forecasts of later patient status and outcome; and (c) \emph{a test data set}, which contains the patient outcomes against which we will evaluate forecasts --- in TADPOLE, this data did not exist at the time of submitting forecasts.

In order to evaluate the effect of different methodological choices, we prepared three “standard” data sets for training and prediction: 
\begin{itemize}
 \item \textbf{D1}: The TADPOLE \underline{\smash{standard training set}} draws on longitudinal data from the entire ADNI history. The data set contains a set of measurements for every individual that has provided data to ADNI in at least two separate visits (different dates) across three phases of the study: ADNI1, ADNI GO, and ADNI2. 
 \item \textbf{D2}: The TADPOLE \underline{\smash{longitudinal prediction set}} contains as much available data as we could gather from the ADNI rollover individuals for whom challenge participants are asked to provide forecasts. D2 includes all available time-points for these individuals. 
 \item \textbf{D3}: The TADPOLE \underline{\smash{cross-sectional prediction set}} contains a single (most recent) time point and a limited set of variables from each rollover individual in D2. Although we expect worse forecasts from this data set than D2, D3 represents the information typically available when selecting a cohort for a clinical trial. 
\end{itemize}


\begin{figure}
 \centering
 \includegraphics[width=0.7\textwidth]{images/datasets_venn_diagram.png}
 \caption[Venn diagram of the TADPOLE datasets derived from ADNI data.]{Venn diagram of the TADPOLE datasets derived from ADNI data, for training (D1), longitudinal prediction (D2), cross-sectional prediction (D3) and the test set (D4). D3 is a subset of D2, which in turn is a subset of D1. Other non-ADNI data can also be used for training.}
 \label{fig:venn_diagram}
\end{figure}

The forecasts will be evaluated on future data (D4 -- test set) from ADNI3 rollovers, acquired after the challenge submission deadline. In addition to the three standard datasets (D1, D2 and D3), challenge participants are allowed to use any other data sets that might serve as useful additional training data.  

Fig. \ref{fig:venn_diagram} shows a diagram highlighting the nested structure of datasets D1--D3. Table \ref{tab:biomk_data_available} shows the proportion of biomarker data available in each dataset. There are a considerable number of entries with missing data, especially for some biomarkers such as tau imaging (AV1451). We also estimated the expected number of subjects and available data for D4, using information from the ADNI3 procedures and using rollovers from previous ADNI studies (Table \ref{tab:biomk_data_available}, right-most column) -- See \ref{app:expectedD4} for more information on D4 estimates. Based on our estimates, we believe the size of D4 (around 330 subjects, 1 visit/subject) should be enough for a reliable evaluation of TADPOLE submissions.


\begin{table}
\centering
 \begin{tabular}{c | c | c c c c}
%   \multicolumn{6}{l}{}\\
 \multicolumn{2}{c|}{\textbf{Subject statistics}} & D1 & D2 & D3 & D4 \\
 \hline
 \multicolumn{2}{c|}{Nr. of subjects } & 1667 & 896 & 896 & \emph{330}\\
 \multicolumn{2}{c|}{Visits per subject }&  $\mathbin{{7.6}{\pm}{3.8}}$  & $\mathbin{{8.5}{\pm}{4.2}}$ & $\mathbin{{1.0}{\pm}{0.0}}$ & $\mathit{\mathbin{{1.0}{\pm}{0.0}}}$\\
 & CN & 31 & 38 & 45 & \emph{39} \\
 Diagnosis* (\%) & MCI & 56 & 57 & 39 & \emph{49} \\
 & AD & 13 & 5 & 16 & \emph{12} \\
%  \multicolumn{6}{c}{}\\
 \multicolumn{2}{l}{\textbf{Data availability**}}\\
 \hline
 \multicolumn{2}{c|}{Cognitive tests (\%) } & 70 & 68 & 84 & \emph{62} \\
 \multicolumn{2}{c|}{MRI (\%) } & 62 & 56 & 75 & \emph{69} \\
 \multicolumn{2}{c|}{FDG-PET (\%) } & 16 & 20 & 0 & \emph{20} \\
 \multicolumn{2}{c|}{AV45-PET (\%) } & 16 & 22 & 0 & \emph{19} \\
 \multicolumn{2}{c|}{AV1451-PET (\%) } & 0.7 & 1.1 & 0 & \emph{19} \\
 \multicolumn{2}{c|}{DTI (\%) } & 6 & 8 & 0 & \emph{15} \\
 \multicolumn{2}{c|}{CSF (\%) } & 18 & 19 & 0 & \emph{14} \\
 \end{tabular}
  
 \caption[Subject statistics and available data in the TADPOLE datasets D1, D2, D3 and D4.]{Subject statistics and available data in the TADPOLE datasets D1, D2 and D3. There is a considerable amount of missing data in some biomarkers such as AV1451. Numbers for D4 are estimated based on ADNI3 procedures (see ADNI3 procedures manual) and rollovers from previous ADNI studies. (*) Diagnosis at baseline visit. (**) Percentage of all visits (across all subjects) that have measurements for desired biomarker.}
 \label{tab:biomk_data_available}
\end{table}




\section{Submissions}
\label{submissions}

There are two kinds of submissions that challenge participants can make. A simple entry requires a minimal forecast and a description of methods; it makes participants eligible for the prizes but not co-authorship on the scientific paper documenting the results. A simple entry can use any training data or prediction sets and forecast at least one of the target outcome variables (clinical status, ADAS13 score, or ventricle volume). A full entry entitles participants for consideration as a co-author on the publication documenting the results. Such a full entry requires a complete forecast for all three outcome variables on all subjects from the D2 prediction set, along with a description of the methods. Each individual participant is limited to a maximum of three submissions. This restriction has been introduced to avoid the risk of participants “tuning” their method on the test set by submitting multiple predictions for a range of algorithm settings. Although not required for a full entry, participants are strongly encouraged to submit predictions also for D3. 

Prizes are awarded to the best submissions regardless of the choice of training sets (D1/custom) and prediction sets (D2/D3). However, the additional submissions support the key scientific aims of the challenge by allowing us to separate the influence of the choice of training data, post-processing pipelines, and modelling techniques or prediction algorithms. The target variables used for evaluation, in particular ventricle volume, will use the same post-processing pipeline as the standard data sets D1-D3.

Beyond the standard training dataset (D1), participants can include additional forecasts from "custom" (i.e. constructed by the participant) training data or custom post-processing of the raw data from subjects in the standard training set. The same applies to the prediction sets D2 and D3, which can be customised by the participants if desired, e.g. a prediction set with different features from the same individuals as in D2 and D3. Table \ref{tab:submissions} shows the twelve possible combinations of subject sets, processing and prediction sets, from which a full-entry submission must contain at least one of the first four (ID 1--4). 


\begin{table}[h]
\centering
 \begin{tabular}{c | c | c | c}
\textbf{ID} & \multicolumn{2}{c|}{\textbf{Training set}} & \textbf{Prediction set}\\
& Subject set & Post-processing & \\
\hline
1 & D1 & standard & D2\\
2 & D1 & custom & D2\\
3 & custom & standard & D2\\
4 & custom & custom & D2\\
5 & D1 & standard & D3\\
6 & D1 & custom & D3\\
7 & custom & standard & D3\\
8 & custom & custom & D3\\
9 & D1 & standard & custom\\
10 & D1 & custom & custom\\
11 & custom & standard & custom\\
12 & custom & custom & custom\\
  
\end{tabular}
\caption[Types of TADPOLE submissions that can be made by participants.]{Types of submissions that can be made by participants, for different types of training sets, prediction sets and post-processing pipelines.}
\label{tab:submissions}
\end{table}


% \FloatBarrier
\section{Forecast Evaluation}
\subsection{Clinical Status Prediction}

For evaluation of clinical status predictions, we will use similar metrics to those that proved effective in the CADDementia challenge \cite{bron2015standardized}: (i) the multiclass area under the receiver operating curve (mAUC); and (ii) the overall balanced classification accuracy (BCA). The mAUC is independent of the group sizes and gives an overall measure of classification ability that accounts for relative likelihoods assigned to each class. The simpler BCA is also independent of group sizes, but does not exploit the probabilistic nature of the forecasts. 

\subsubsection{Multiclass Area Under the Receiver Operating Characteristic (ROC) Curve}

The multiclass Area Under the ROC Curve (mAUC) is a simple generalisation of the area under the ROC curve applicable to problems with more than two classes \cite{hand2001simple}. The AUC $\hat{A}(c_i|c_j)$ for classification of a class $c_i$ against another class $c_j$, is:
\begin{equation}
\hat{A}(c_i|c_j)=\frac{S_i-n_i(n_i+1)/2}{n_i n_j}
\end{equation}
where $n_i$ and $n_j$ are the number of points belonging to classes $i$ and $j$, respectively; while $S_i$ is the sum of the ranks of the class $i$ test points after ranking all the class $i$ and $j$ data points in increasing likelihood of belonging to class $i$. We further define the average AUC for classes $i$ and $j$ as $\hat{A}(c_i,c_j)= 0.5(\hat{A}(c_i|c_j)+\hat{A}(c_j|c_i))$. The overall mAUC is then obtained by averaging $\hat{A}(c_i,c_j)$ over all pairs of classes:
\begin{equation}
 mAUC = \frac{2}{L(L-1)}\sum_{i=2}^L\sum_{j=1}^{i}\hat{A}(c_i,c_j)
\end{equation}
where $L$ is the number of classes. The class probabilities that go into the calculation of $S_i$ in the first equation are $p_{CN}$, $p_{MCI}$ and $p_{AD}$, which are derived from the likelihoods of each ADNI subject being assigned to each diagnostic class, by normalising to have unity sum.

\subsubsection{Balanced Classification Accuracy}

The Balanced Classification Accuracy (see \cite{brodersen2010balanced}) is an extension of the classification accuracy measure that accounts for the imbalance in the numbers of datapoints belonging to each class. However, the measure is not probabilistic, so in TADPOLE the data points need to be assigned a hard classification to the class (CN, MCI, or AD) with the highest likelihood. The balanced accuracy for class $i$ is then:
\begin{equation}
 BCA_i = \frac{1}{2}\left[\frac{TP}{TP+FN}+\frac{TN}{TN+FP}\right]
\end{equation}
where TP, FP, TN, FN represent the number of true positives, false positives, true negatives and false negatives for classification as class $i$. In this case, true positives are data points with true label $i$ and correctly classified as such, while the false negatives are the data points with true label $i$ and incorrectly classified to a different class $j \ne i$. True negatives and false positives are defined similarly. The overall BCA is given by the mean of all the balanced accuracies for every class. 

\subsection{Continuous Feature Predictions}

For ADAS13 and ventricle volume, we will use three metrics: mean absolute error (MAE), weighted error score (WES) and coverage probability accuracy (CPA). The MAE focuses purely on accuracy of the best-guess prediction ignoring the confidence interval, whereas the WES incorporates confidence estimates into the error score. The CPA provides an assessment of the accuracy of the confidence estimates, irrespective of the best-guess prediction accuracy.


\subsubsection{Mean Absolute Error}

The mean absolute error (MAE) is:
\begin{equation}
 MAE = \frac{1}{N}\sum_{i=1}^{N}\left|{\tilde{M}_i-M_i}\right|
\end{equation}
where $N$ is the number of data points (forecasts) evaluated, $M_i$ is the actual biomarker value in individual $i$ in future data, and $\tilde{M}_i$ is the participant's best prediction for $M_i$.

\subsubsection{Weighted Error Score}

The weighted error score is defined as:
\begin{equation}
 WES=\frac{\sum_{i=1}^{N}\tilde{C}_i\left|\tilde{M}_i-M_i\right|}{\sum_{i=1}^{N}\tilde{C}_i}
\end{equation}
where $\tilde{C}_i$ is the participant's relative confidence in their $\tilde{M}_i$ estimate. We estimate $\tilde{C}_i$ as the inverse of the width of the 50\% confidence interval of their biomarker estimate:
\begin{equation}
\tilde{C}_i=\left(C_+-C_-\right)^{-1}
\end{equation}
where $[C-, C+]$ is the confidence interval provided by the participant.

\subsubsection{Coverage Probability Accuracy}

The coverage probability accuracy is:
\begin{equation}
CPA = |ACP - NCP| 
\end{equation}
where $NCP$ is the nominal coverage probability, the target for the confidence intervals, and $ACP$ is the actual coverage probability, defined as the proportion of measurements that fall within the corresponding confidence interval. In TADPOLE, we set $NCP$ to be 0.5, which means that ideally only 50\% of the measurements would fall inside the confidence interval. The CPA can take values between 0 and 1, and lower scores are better.

\section{Prizes}
We are extremely grateful to Alzheimer's Research UK, The Alzheimer's Society, and The Alzheimer's Association for sponsoring a prize fund of \pounds 30,000. At the time of first submission, we proposed six separate prizes, as outlined in Table \ref{tab:prizes}, but reserve the right to reallocate the prize money depending on the numbers of participants eligible for each prize. The first four are general categories (open to all challenge participants) and constitute one prize for the best forecast of each feature as well as one for overall best performance. The last two prizes are for two different student categories.


\begin{table}[h]
\centering
 \begin{tabular}{>{\centering\arraybackslash}m{1.5cm}  c  >{\centering\arraybackslash}m{3cm}  >{\centering\arraybackslash}m{3cm}}
\textbf{Prize amount} & \textbf{Outcome measure} & \textbf{Performance Metric} & \textbf{Eligibility} \\
\hline
\pounds 5,000 & Clinical status & mAUC & all \\
\pounds 5,000 & ADAS13 & MAE & all\\
\pounds 5,000 & Ventricle volume & MAE & all\\
\pounds 5,000 & Overall best & Lowest sum of ranks* & all\\
\pounds 5,000 & Clinical status & mAUC & University teams\\
\pounds 5,000 & Clinical status & mAUC & High-school teams\\
\end{tabular}
\caption[TADPOLE prize allocation scheme using funds from AD charities]{Prize allocation scheme using funds from Alzheimer's Research UK, The Alzheimer's Society and The Alzheimer's Association. There are 6 prizes awarded to different outcome measures, the last two of which are eligible only for university and high-school teams. (*) The overall best team will be the team that obtains the lowest sum of ranks in the clinical status, ADAS13 and ventricle volume categories. }
\label{tab:prizes}
\end{table}

\section{Discussion}

We have outlined the design of the TADPOLE Challenge, which aims to identify algorithms and features that can best predict the evolution of Alzheimer's disease. Challenge participants use historical data from ADNI in order to predict three key outcomes: clinical diagnosis, ADAS-Cog13 and ventricle volume. Determining which features and algorithms best predict AD evolution can aid refinement of cohorts and endpoint assessment for clinical trials, and can provide accurate prognostic information in clinical settings. 

The TADPOLE Challenge was designed to be transparent and accessible. To this end, all of our scripts are available in an open repository\footnote{TADPOLE repository: https://github.com/noxtoby/TADPOLE}. We also created a public forum\footnote{TADPOLE forum:  https://groups.google.com/forum/\#!forum/tadpolechallenge} where we answer participant questions. Finally, in order to enable participants to share algorithm performance results throughout the competition, we created a leaderboard system\footnote{Leaderboard: https://tadpole.grand-challenge.org/leaderboard/} that evaluates submissions on an existing test dataset and publishes the results live on our website.  

Going forward, we hope that by November 2018 sufficient data will be available from ADNI3 rollovers for a first meaningful evaluation of the forecasts. We plan to publish the results on the website in January 2019, and then submit a publication of the results soon after. However, we reserve the right to delay evaluation until sufficient data is available. At that time, we will also evaluate the impact and interest of the first phase of TADPOLE within the community, to guide decisions on whether to organise further submission and evaluation phases.

The fact that the D4 test set could have different properties from the training set is something that can affect the performance of certain algorithms. For example, some algorithms could perform better on different forecast time windows (short-term vs long-term) or on subjects with different properties (e.g. those with more follow-up training data vs those with less data). At the evaluation stage, we thus take into consideration doing the evaluation on different splits of the test set, in order to understand what kind of predictions algorithms perform best at. 

\section{Conclusion}

In this section I presented the TADPOLE Challenge, which aims to identify algorithms and features that best predict the evolution of subjects at risk of Alzheimer's disease. The outcomes of the challenge will be made available early in 2019, after sufficient data has been acquired. In the next chapter, I will present future work on the TADPOLE Challenge, as well as the other chapters of the thesis.


\include{conclusion}


\appendix

% these do NOT count as part of the suggested page count
% This is probably a good place to explain the models in some detail for example
\include{appendix}

\cleardoublepage

\nocite{*} % Show all Bib-entries
\bibliographystyle{unsrt}
\bibliography{citations}

%\printbibliography




\end{document}
